% Copyright 2024-2025 Kieran W Harvie. All rights reserved.

\section{Series Involving Harmonic Numbers}
I've once again wasted to much time thinking about the following series:
\[\sum_{n=1}^\infty\frac{H_n}{n^2}=2\zeta(3),\quad \sum_{n=1}^\infty\frac{H_n}{n^3}=\frac{1}{2}\zeta(2)\]
And,
while the time wasted is a nice example of how math sometimes feels more compulsive then is healthy,
I also want to write down some theory here and hope I may avoid wasting time thinking about this in future.

\subsection{Convergence}
From the standard harmonic number bounds:
\[\ln(n)+\frac{1}{n} \leq H_n \leq \ln(n)+1\]
We can see that the series $\sum_{n=1}^\infty H_nn^{-k}$ wouldn't converge for $k=0,1$ but would for $k\geq 2$.

\subsection{Properties of $T$:}
Series of the form:
\[T(s_1,s_2;s) = \sum_{n=1}^{\infty}\sum_{m=1}^{\infty}\frac{1}{n^{s_1}m^{s_2}(n+m)^s}\]
Are called Tornheim Sums and are a subcase of the more general Mordell-Tornheim Sums with a depth of $2$.
The weight of a Tornheim Sum is the sum of its parameters:
\[s_1+s_2+s\]

\subsubsection{Argument Symmetry of $T$:}
From the symmetry of $n$ and $m$ in the bounds and summand we get:
\[T(s_1,s_2;s) = T(s_2,s_1;s)\]
So by convention we simplify Tornheim Sums by writing them with $s_1\leq s_2$.

\subsubsection{Recursion:}
Consider the following expression:
\[\frac{1}{n^{s_1}m^{s_2}(n+m)^s} = \frac{1}{n^{s_1-1}m^{s_2}(n+m)^{s+1}}+\frac{1}{n^{s_1}m^{s_2-1}(n+m)^{s+1}}\]
It can be verified pretty easily by factorizing out $n^{s_1-1}m^{s_2-1}(n+m)^s$ and
double summed  over $n$ and $m$ to obtain:
\[T(s_1,s_2;s) = T(s_1-1,s_2;s+1)+T(s_1,s_2-1;s+1)\]
Note the weight of all terms is the same.

\subsubsection{Relation to $\zeta$ functions:}
When one of the arguments is zero the Tornheim Sums simplify to the (general) zeta functions:
\begin{equation*}
\begin{aligned}
	T(s_1,s_2;0) =& \sum_{n=1}^\infty\sum_{m=1}^\infty\frac{1}{n^{s_1}m^{s_2}} =\zeta(s_1)\zeta(s_2)\\
	T(0,s_2;s) =& \sum_{n=1}^\infty\sum_{m=1}^\infty\frac{1}{m^{s_2}(n+m)^s} =\zeta(s,s_2)\\
\end{aligned}
\end{equation*}

\subsubsection{Telescoping Recursion:}
Consider the last expression in more detail,
if we change how we parameterize the summation we get:
\begin{equation*}
\begin{aligned}
	T(0,s_2;s)=& \sum_{n=1}^\infty\sum_{m=1}^\infty\frac{1}{m^{s_2}(n+m)^s}\\
	=&\sum_{n=2}^\infty\frac{1}{n^s}\sum_{m=1}^{n-1}\frac{1}{m^{s_2}} \\
	=&\sum_{n=2}^\infty\frac{1}{n^s}\sum_{m=1}^n\frac{1}{m^{s_2}}-\sum_{n=2}^\infty\frac{1}{n^{s+s_2}} \\
	=&\sum_{n=1}^\infty\frac{1}{n^s}\sum_{m=1}^n\frac{1}{m^{s_2}}-\sum_{n=1}^\infty\frac{1}{n^{s+s_2}} \\
\end{aligned}
\end{equation*}
The left term is very similar to a Tornheim Sum!
All we need to do is turn the finite sum into a series.
When $s_2=1$ we can achieve this with a telescoping series,
(even though this technique is normally used to go from an infinite series to finite sum):
\begin{equation*}
\begin{aligned}
	\sum_{m=1}^{n}\frac{1}{m} =&\sum_{m=1}^\infty\left(\frac{1}{m}-\frac{1}{m+n}\right)\\
	=&n\sum_{m=1}^\infty\frac{1}{m(n+m)}\\
\end{aligned}
\end{equation*}
Substitution gives:
\begin{equation*}
\begin{aligned}
	T(0,1;s) =&\sum_{n=1}^\infty\frac{1}{n^s}\sum_{m=1}^n\frac{1}{m}-\sum_{n=1}^\infty\frac{1}{n^{s+1}} \\
	=&\sum_{n=1}^\infty\sum_{m=1}^\infty\frac{1}{n^{s-1}m(n+m)}-\sum_{n=1}^\infty\frac{1}{n^{s+1}} \\
	=&T(1,s-1;1)-\zeta(s+1)\\
\end{aligned}
\end{equation*}
Again note that the weights are the same.

\subsection{Calculating the Original Series}
With all this machinery in place we return to the original series.
Start by writing them as Tornheim Sums by using the telescoping trick again:
\begin{equation*}
\begin{aligned}
	\sum_{n=1}^\infty\frac{H_n}{n^k} =& \sum_{n=1}^\infty\frac{1}{n^k}\left(\sum_{m=1}^{n}\frac{1}{m}\right)\\
	=& \sum_{n=1}^\infty\sum_{m=1}^\infty\frac{1}{mn^{k-1}(n+m)}\\
	=& T(1,k-1;1)\\
\end{aligned}
\end{equation*}

\subsubsection{Case $k=3$:}
This case follows from a direct application of the general recursion to $T(2,2;0)$
\[T(2,2;0) = T(1,2;1)+T(2,1;1) = 2T(1,2;1)\]
Hence:
\[\sum_{n=1}^\infty\frac{H_n}{n^3} = T(1,2;1) = \frac{1}{2}T(2,2;0) = \frac{1}{2}\zeta(2)^2\]

\subsubsection{Case $k=2$:}
For this case we first use the telescoping result to give:
\[T(1,1;1) = T(0,1;2)+\zeta(3)\]
And the general recursive result to give:
\[T(1,1;1) = 2T(0,1;2)\]
Combining these results gives:
\[T(0,1;2) = \zeta(3)\]
Hence:
\[\sum_{n=1}^\infty\frac{H_n}{n^2} = T(1,1;1) = 2\zeta(3)\]

\subsection{Generalization}
The telescoping trick can be generalized.
Let $\chi_n$ be a family of functions such that:
\[\chi_n(x+n)=\chi_n(x)\]
Then:
\begin{equation*}
\begin{aligned}
	\sum_{n=1}^\infty\sum_{m=1}^\infty\frac{\chi_n(m)}{n^sm(n+m)} =& \sum_{n=1}^\infty\frac{1}{n^{s+1}}\sum_{m=1}^\infty\chi_n(m)\left(\frac{1}{m}-\frac{1}{m+n}\right)\\
	=& \sum_{n=1}^\infty\frac{1}{n^{s+1}}\sum_{k=1}^n\sum_{m=0}^\infty\chi_n(k+nm)\left(\frac{1}{k+nm}-\frac{1}{k+n(m+1)}\right)\\
	=& \sum_{n=1}^\infty\frac{1}{n^{s+1}}\sum_{k=1}^n\chi_n(k)\sum_{m=0}^\infty\left(\frac{1}{k+nm}-\frac{1}{k+n(m+1)}\right)\\
	=& \sum_{n=1}^\infty\frac{1}{n^{s+1}}\sum_{k=1}^n\frac{\chi_n(k)}{k}\\
\end{aligned}
\end{equation*}
The original result is recovered when $\chi_n(m)=1$,
another choice is the function:
\[\chi_n(m)=\begin{cases}1&\gcd(n,m)=1\\0&\text{ otherwise}\end{cases}\]
From the `Inverson Bracket and Möbius Function'\footnote{hyperref isn't working and I don't feel like fixing it.} section we have:
\[\sum_{n=1}^\infty\sum_{\substack{m=1\\\gcd(n,m)=1}}^\infty\frac{1}{n^sm(n+m)}\\= \frac{1}{\zeta(s+2)}\sum_{n=1}^\infty\sum_{m=1}^\infty\frac{1}{n^sm(n+m)}\]
Combining gives:
\begin{equation*}
\begin{aligned}
	\sum_{n=1}^\infty\frac{1}{n^{s+1}}\sum_{\substack{k=1\\\gcd(n,k)=1}}^n\frac{1}{k}
	=& \sum_{n=1}^\infty\sum_{\substack{m=1\\\gcd(n,m)=1}}^\infty\frac{1}{n^sm(n+m)}\\
	=& \frac{1}{\zeta(s+2)}\sum_{n=1}^\infty\sum_{m=1}^\infty\frac{1}{n^sm(n+m)}\\
	=& \frac{1}{\zeta(s+2)}\sum_{n=1}^\infty\frac{1}{n^{s+1}}\sum_{k=1}^n\frac{1}{k}
\end{aligned}
\end{equation*}
Which is a pretty cool expression that relates two Dirichlet generating functions.
Since the two functions are related by multiplication of a zeta function there's either more cool convolution based results or a result which trivializes all this work.

\subsubsection{Cool Convolution Results}
Lets execute on the promise of additional results and decide if it trivializes previous work or not.
Start by defining:
\[\tilde{H}_n = \sum_{\substack{k=1\\\gcd(n,k)=1}}^n\frac{1}{k}\]
Allowing us to take the above result into a more standard form.
\[ \sum_{n=1}^\infty\frac{1}{n^s}\frac{H_n}{n}=\left(\sum_{n=1}^\infty\frac{1}{n^s}\frac{1}{n^2}\right)\left(\sum_{n=1}^\infty\frac{1}{n^s}\frac{\tilde{H}_n}{n}\right)\]
Which, by the Dirichlet convolution theorem, gives:
\[\frac{H_n}{n} = \sum_{d|n}\frac{\tilde{H}_d}{d}\frac{1}{\left(\frac{n}{d}\right)^2}\]
Which after cleanup is:
\[H_n = \frac{1}{n}\sum_{d|n}d\tilde{H}_d\]
But what about triviality?
\\

\subsubsection{Triviality}
One way the result could be said to be trivialized is if it can easily achieved using elementary number theory.
To that end consider the following result,
let $n,k,m$ be integers with $m>0$ then we have:
\[\gcd(mn,mk)=m\gcd(n,k)\]
(This result intuitively follows from expressing $\gcd(n,k)$ as the minimal non-zero linear combination of $n$ and $k$).
When $d|n$ and $\gcd(nd^{-1},k)=1$ we have:
\[\gcd(n,dk) = \gcd(ndd^{-1},dk) = d\gcd(nd^{-1},k) = d\]
Hence we can reparametrize the following sum:
\[\sum_{\substack{k=1\\\gcd(nd^{-1},k)=1}}^{nd^{-1}}f(dk)=\sum_{\substack{k'=1\\\gcd(n,k')}}^nf(k')\]
Applying to our series:
\begin{equation*}
\begin{aligned}
	H_n =& \sum_{k=1}^n\frac{1}{k}\\
	=& \sum_{d|n}\sum_{\substack{k=1\\\gcd(n,k)=d}}^n\frac{1}{k}\\
	=& \sum_{d|n}\sum_{\substack{k=1\\\gcd(nd^{-1},k)=1}}^{nd^{-1}}\frac{1}{dk}\\
	=& \sum_{d|n}\frac{1}{d}\tilde{H}_{nd^{-1}}\\
\end{aligned}
\end{equation*}
Swapping the order of summation, $d\mapsto nd^{-1}$ we get the original result.
I'd say this result is elementary enough to trivialize the previous result but if we can get just a touch more advanced there's a cooler proof.
\\

It's clear that:
\[\sum_{\substack{k=1\\d|k}}^n\frac{1}{k}=\frac{1}{d}H_{nd^{-1}}\]
Hence by the Inclusion-Exclusion principle we have:
\[\tilde{H}_n = \sum_{d|n}\mu(k)\frac{H_{nd^{-1}}}{d}\]
As we start with $d=1$,
which includes all reciprocals,
then $\mu(p)=-1$ means we exclude all numbers with a given prime divisor,
then $\mu(p_0p_=)=1$ means we include all numbers with two given prime divisors,
which otherwise would have been removed twice.
So an and so forth till we get the whole sum.
Möbius inversion gives us the original result.
