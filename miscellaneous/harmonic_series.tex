% Copyright 2024 Kieran W Harvie. All rights reserved.

\section{Series Involving Harmonic Numbers}
I've once again wasted to much time thinking about the following series:
\[\sum_{n=1}^\infty\frac{H_n}{n^2}=2\zeta(3),\quad \sum_{n=1}^\infty\frac{H_n}{n^3}=\frac{1}{2}\zeta(2)\]
While the time wasted is a nice example of how math sometimes feels more compulsive then is healthy,
I also want to write down some theory here and hope I may avoid wasting time thinking about this in future.

\subsection{Convergence}
From the standard harmonic number bounds:
\[\ln(n)+\frac{1}{n} \leq H_n \leq \ln(n)+1\]
We can see that the series $\sum_{n=1}^\infty H_nn^{-k}$ wouldn't converge for $k=0,1$ but would for $k\geq 2$.

\subsection{Properties of $T$:}
Series of the form:
\[T(s_1,s_2;s) = \sum_{n=1}^{\infty}\sum_{m=1}^{\infty}\frac{1}{n^{s_1}m^{s_2}(n+m)^s}\]
Are called Tornheim Sums and is a case of the more general Mordell-Tornheim Sums with a depth of $2$.
The weight of a Tornheim Sum is the sum of its parameters, $s_1+s_2+s$.

\subsubsection{Argument Symmetry of $T$:}
From the symmetry of $n$ and $m$ in the bounds and summand we get:
\[T(s_1,s_2;s) = T(s_2,s_1;s)\]
By convention, we simplify Tornheim Sums by writing them with $s_1\leq s_2$.

\subsubsection{Recursion:}
Consider the following expression:
\[\frac{1}{n^{s_1}m^{s_2}(n+m)^s} = \frac{1}{n^{s_1-1}m^{s_2}(n+m)^{s+1}}+\frac{1}{n^{s_1}m^{s_2-1}(n+m)^{s+1}}\]
It can be verified pretty easily by factorizing out $n^{s_1-1}m^{s_2-1}(n+m)^s$,
double summing this expression over $n$ and $m$ we obtain:
\[T(s_1,s_2;s) = T(s_1-1,s_2;s+1)+T(s_1,s_2-1;s+1)\]
Note the weight of all terms is the same.

\subsubsection{Relation to $\zeta$ functions:}
When one or more of the arguments are zero the Tornheim Sums simplify to the (general) zeta functions:
\begin{equation*}
\begin{aligned}
	T(s_1,s_2;0) =& \sum_{n=1}^\infty\sum_{m=1}^\infty\frac{1}{n^{s_1}m^{s_2}} =\zeta(s_1)\zeta(s_2)\\
	T(0,s_2;s) =& \sum_{n=1}^\infty\sum_{m=1}^\infty\frac{1}{m^{s_2}(n+m)^s} =\zeta(s,s_2)\\
\end{aligned}
\end{equation*}

\subsubsection{Telescoping Recursion:}
Consider the last expression in more detail,
if we change how we parameterize the summation we get:
\begin{equation*}
\begin{aligned}
	T(0,s_2;s)=& \sum_{n=1}^\infty\sum_{m=1}^\infty\frac{1}{m^{s_2}(n+m)^s}\\
	=&\sum_{n=2}^\infty\frac{1}{n^s}\sum_{m=1}^{n-1}\frac{1}{m^{s_2}} \\
	=&\sum_{n=2}^\infty\frac{1}{n^s}\sum_{m=1}^n\frac{1}{m^{s_2}}-\sum_{n=2}^\infty\frac{1}{n^{s+s_2}} \\
	=&\sum_{n=1}^\infty\frac{1}{n^s}\sum_{m=1}^n\frac{1}{m^{s_2}}-\sum_{n=1}^\infty\frac{1}{n^{s+s_2}} \\
\end{aligned}
\end{equation*}
The left term is very similar to a Tornheim Sum!
All we need to do is turn the finite sum into a series.
When $s_2=1$ we can achieve this with a telescoping series,
(even though this technique is normally used to go from series to finite sum):
\begin{equation*}
\begin{aligned}
	\sum_{m=1}^{n}\frac{1}{m} =&\sum_{m=1}^\infty\left(\frac{1}{m}-\frac{1}{m+n}\right)\\
	=&n\sum_{m=1}^\infty\frac{1}{m(n+m)}\\
\end{aligned}
\end{equation*}
Substitution gives:
\begin{equation*}
\begin{aligned}
	T(0,1;s) =&\sum_{n=1}^\infty\frac{1}{n^s}\sum_{m=1}^n\frac{1}{m}-\sum_{n=1}^\infty\frac{1}{n^{s+1}} \\
	=&\sum_{n=1}^\infty\sum_{m=1}^\infty\frac{1}{n^{s-1}m(n+m)}-\sum_{n=1}^\infty\frac{1}{n^{s+1}} \\
	=&T(1,s-1;1)-T(0,s+1;0)\\
\end{aligned}
\end{equation*}
Again note that the weights are the same.

\subsection{Calculating the Original Series}
With all this machinery in place we return to the original series.
Start by writing them as Tornheim Sums by using the telescoping trick again:
\begin{equation*}
\begin{aligned}
	\sum_{n=1}^\infty\frac{H_n}{n^k} =& \sum_{n=1}^\infty\frac{1}{n^k}\left(\sum_{m=1}^{n}\frac{1}{m}\right)\\
	=& \sum_{n=1}^\infty\sum_{m=1}^\infty\frac{1}{mn^{k-1}(n+m)}\\
	=& T(1,k-1;1)\\
\end{aligned}
\end{equation*}

\subsubsection{Case $k=3$:}
This case is follows from a direct application of general recursion to $T(2,2;0)$
\[T(2,2;0) = T(1,2;1)+T(2,1;1) = 2T(1,2;1)\]
Hence:
\[\sum_{n=1}^\infty\frac{H_n}{n^3} = T(1,2;1) = \frac{1}{2}T(2,2;0) = \frac{1}{2}\zeta(2)^2\]

\subsubsection{Case $k=2$:}
For this case we first use the telescoping result to give:
\[T(1,1;1) = T(0,1;2)+T(0,3;0)\]
And the general recursive result to give:
\[T(1,1;1) = 2T(0,1;2)\]
Combining these results gives:
\[T(0,1;2) = T(0,3;0) = \zeta(3)\]
Hence:
\[\sum_{n=1}^\infty\frac{H_n}{n^2} = T(1,1;1) = 2\zeta(3)\]

\subsection{Generalization}
The telescoping trick can be generalized.
Let $\chi_n$ be a family of functions such that:
\[\chi_n(x+n)=\chi_n(x)\]
Then:
\begin{equation*}
\begin{aligned}
	\sum_{n=1}^\infty\sum_{m=1}^\infty\frac{\chi_n(m)}{n^sm(n+m)} =& \sum_{n=1}^\infty\frac{1}{n^{s+1}}\sum_{m=1}^\infty\chi_n(m)\left(\frac{1}{m}-\frac{1}{m+n}\right)\\
	=& \sum_{n=1}^\infty\frac{1}{n^{s+1}}\sum_{k=1}^n\sum_{m=0}^\infty\chi_n(k+nm)\left(\frac{1}{k+nm}-\frac{1}{k+n(m+1)}\right)\\
	=& \sum_{n=1}^\infty\frac{1}{n^{s+1}}\sum_{k=1}^n\chi_n(k)\sum_{m=0}^\infty\left(\frac{1}{k+nm}-\frac{1}{k+n(m+1)}\right)\\
	=& \sum_{n=1}^\infty\frac{1}{n^{s+1}}\sum_{k=1}^n\frac{\chi_n(k)}{k}\\
\end{aligned}
\end{equation*}
The original result is recovered when $\chi_n(m)=1$,
another choice is the function:
\[\chi_n(m)=\begin{cases}1&\gcd(n,m)=1\\0&\text{ otherwise}\end{cases}\]
From the `Inverson Bracket and Möbius Function'\footnote{hyperref isn't working and I don't feel like fixing it.} section we have:
\[\sum_{n=1}^\infty\sum_{\substack{m=1\\\gcd(n,m)=1}}^\infty\frac{1}{n^sm(n+m)}\\= \frac{1}{\zeta(s+2)}\sum_{n=1}^\infty\sum_{m=1}^\infty\frac{1}{n^sm(n+m)}\]
Combining gives:
\begin{equation*}
\begin{aligned}
	\sum_{n=1}^\infty\frac{1}{n^{s+1}}\sum_{\substack{k=1\\\gcd(n,k)=1}}^n\frac{1}{k}
	=& \sum_{n=1}^\infty\sum_{\substack{m=1\\\gcd(n,m)=1}}^\infty\frac{1}{n^sm(n+m)}\\
	=& \frac{1}{\zeta(s+2)}\sum_{n=1}^\infty\sum_{m=1}^\infty\frac{1}{n^sm(n+m)}\\
	=& \frac{1}{\zeta(s+2)}\sum_{n=1}^\infty\frac{1}{n^{s+1}}\sum_{k=1}^n\frac{1}{k}
\end{aligned}
\end{equation*}
Which is a pretty cool expression that relates two Dirichlet generating functions.
Since the two functions are related by multiplication of a zeta function there's either more cool convolution based results or a result which trivializes all this work.
