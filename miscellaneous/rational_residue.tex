% Copyright 2023 Kieran W Harvie. All rights reserved.

\section{Rational Residue}
I saw a cool result.
That if $P$ and $Q$ are polynomials and $Q$ has a first order root at $a$ then the then:
\[\Res\left(\frac{P}{Q},a\right) = \frac{P(a)}{Q'(a)} \]
Which can save a lot of time over factorizing, 
which is what I would have done in the past.

This follows directly from:
\[Q'(a)=\frac{d}{dz}(z-a)q(z)\bigg|_a = q(a)\]
and
\[\oint_\gamma\frac{P(z)}{Q(z)}\,dz = \oint_\gamma \frac{P(z)}{(z-a)q(z)}\,dz = 2\pi i\frac{P(a)}{q(a)}\]
\\

I thought this might be generalizable since the general Leibniz rule on $Q(z)=(z-a)^2q(z)$ gives:
\[Q''(a)=2q(a),\quad Q'''(a)=6q'(a)\]
Hence for a second order root:
\begin{equation*}
\begin{aligned}
\end{aligned}
\end{equation*}
\begin{equation*}
\begin{aligned}
	\oint_\gamma\frac{P(z)}{Q(z)}\,dz =& \oint_\gamma \frac{P(z)}{(z-a)^2q(z)}\,dz \\
	=& \pi i\frac{P'(a)q(a)-P(a)q'(a)}{q(a)^2}\\
	=& 2\pi i\frac{3P'(a)Q''(a)-P(a)Q'''(a)}{3Q'(a)^2}\\
\end{aligned}
\end{equation*}
Giving:
\[\Res\left(\frac{P}{Q},a\right)= \frac{3P'(a)Q''(a)-P(a)Q'''(a)}{3Q'(a)^2} \]

Which isn't as cool as the first order case,
and I don't feel like dealing with higher derivatives.

Still cool though.

\subsection{Actual closed form}
There actually is a way to get a closed form without dealing with higher order derivatives of reciprocals.
The idea is to use the Bézout trick to get a polynomial that acts like a reciprocal at $z=a$.
\\

Let $Q$ be a polynomial with an $n$ ordered root at $a$.
Then define $q$ as:
\[Q(z) = (z-a)^nq(z)\]
Since $\gcd(q(z),(z-a)^n) = 1$ from the Bézout's identity, there exits polynomials $r$ and $s$ such that:
\[r(z)(z-a)^n+s(z)q(z)=1\]
Hence:
\[\frac{P(z)}{(z-a)^nq(z)}(1-s(z)q(z)) = \frac{P(z)}{q(z)}r(z)\]
Has no poles at $a$ and vanishes in the contour integral.
Meaning we can split up its terms to give:
\[\oint_\gamma\frac{P(z)}{(z-a)^nq(z)}\,dz = \oint_\gamma \frac{P(z)s(z)}{(z-a)^n}\,dz\]
Resulting in the clean:
\[\Res\left(\frac{P}{Q},a\right) = \frac{1}{(n-1)!}\frac{d^{n-1}}{dz^{n-1}}P(z)s(z)\bigg|_{z=a}\]

\subsubsection{Remarks}
You can use use the standard extended Euclid algorithm to calculate $s$ from $q$, or $q$ from $s$.
But the relation:
\[r(z)(z-a)^n+s(z)q(z)=1\]
Can be derived $n-1$ and set $z=a$ to get a linear system between derivatives of $s$ and $q$.
So why bother with Bézout at all?
Well you would need to prove that the coefficients from the system converge to an analytic function,
Bézout side steps this.
\\

I'm also wondering when $s^{(n)}(a)=0$, because we loose information about $F$ in the sum when this happens.
Also working backwards, can we have any $s$.
Not $s(z) = z-a$ obviously, but we can add another power to keep it? 
$s(z) = z^2+z-a$?

\subsection{A Generalization}
Generate new solutions to the Bézout Identity in the standard way:
\[[r(z)-h(z)s(z)](z-a)^n+s(z)[q(z)+(z-a)^nh(z)]=1\]
If we choose $h$ such that $r(z)-h(z)s(z)$ is holomorphic,
easy since $r$ and $s$ are polynomials, we have, given $\gamma$ is small enough to not add new poles from $h$:
\[\oint_\gamma\frac{P(z)}{(z-a)^n[q(z)+(z-a)^nh(z)]}\,dz = \oint_\gamma \frac{P(z)s(z)}{(z-a)^n}\,dz= \frac{1}{(n-1)!}\frac{d^{n-1}}{dz^{n-1}}P(z)s(z)\bigg|_{z=a}\]

Observe that all function $q'(z)$ such $(z-a)^n | q(z)-q'(z)$ can be written as $q'(z) = q(z)+(z-a)^nh(z)$ for some holomorphic $h$ giving the pretty cool result that only the fist $n$ derivatives of $q'$ effect the value of the contour integral.
(Also observe that $r'(z) = r(z)-h(z)s(z)$ means $r'$ agrees  with $r$ on $s$ helping motivate the LHS)
\\

$q'(z)=1$ gives $s(z)=1-(z-a)^n,\,r'(z)=1$ and hence Cauchy's integral formula and hence generalizes it.

This could also  be used to extract terms from the reciprocal of a generating function.
