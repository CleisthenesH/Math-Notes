% Copyright 2023 Kieran W Harvie. All rights reserved.

\section{Vieta's Formula Applied to Prime Fields}
For a prime number $p$ we have the well known formula:
\[x^p-x \equiv \prod_{k=0}^{p-1}(x-k)\,\mod p\]
(Look up Frobenius Endomorphism for more info).

Vieta's formulas link the elementary symmetric functions of a polynomials roots to it's polynomials.
There are two nonzero coefficients the $p$-th degree doesn't tell anything and the $1$ is one-half of Wilson's Theorem:
\[(p-1)! \equiv -1\,\mod p\]

The other coefficients are zero hence the other elementary symmetric functions of $0$ to $p-1$ are divisible by $p$.
(Since terms with $0$ vanish $1$ to $p-1$ can be used).

\subsubsection{Example $p=5$:}
\begin{equation*}
\begin{aligned}
	&1\cdot2\cdot3\cdot 4 &=& 24 =&&5\cdot5-1\\
	&1\cdot2\cdot3 + 1\cdot2\cdot4+1\cdot3\cdot4+2\cdot3\cdot4&=& 50 =&& 5\cdot 10\\
	&1\cdot2+1\cdot3+1\cdot4+2\cdot3+2\cdot4+3\cdot4 &=& 35 =&&5\cdot 7\\
	&1+2+3+4 &=& 10 =&&5\cdot 2\\
\end{aligned}
\end{equation*}

\subsection{Commutative Rings}
Let $R$ be a commutative ring and let $x_0$ and $x_1$ be distinct elements of $R$.
Let $f$ be an element of $R[X]$ such that $f(x_0)=f(x_1) = 0$ then:
\[f(x) = (x-x_0)(x-x_1)q(x)+(x-x_0)r\]
For some $r\in R$ and $q \in R[X]$.
\\

\subsubsection{Proof:}
It is a corollary of Euclidean division of polynomials with monic polynomials that for a given $p\in R[X]$ and $x'\in R$ there exists a $p' \in R[X]$ such that:
\[p(x) = (x-x')p'(x)+p(x')\]
Applying this to $f$ and $x_0$ we get:
\[f(x) = (x-x_0)f'(x) + f(x_0) = (x-x_0)f(x)\]
And applying this to $f'$ and $x_1$ we get:
\[f(x) = (x-x_0)((x-x_1)f''(x)+f'(x_1))\]
Setting $q=f''$ and $r=f'(x_1)$ completes the proof

\subsubsection{Corollary:}
If $R$ is a domain, 
has no zero dividers,
then $r=0$ since $x_1-x_0\neq 0$ and:
\[0 = f(x_1) = (x_1-x_0)r\]
This is why the result main result works since $\mathbb{Z}/p\mathbb{Z}$ is a field,
and hence domain,
and why it may not work in general as:
\[2\cdot 3 \equiv 0 \mod 6\]

\subsection{Composite Numbers}
For general $n$ and relative prime $x$ we have:
\[x^{\lambda(n)}\equiv 1 \mod n\]
Which looks similar but we already see problems.
There are $\phi(n)$ totients to $n$ but $\lambda(n)$ can be less than $\phi(n)$ meaning a $\lambda(n)$ degree polynomial with $\phi(n)$ distinct roots has more roots than it's degree.
For example:
\[\lambda(8) = 2 < 4 = \phi(8)\]
So all of $\{1,3,5,7\}$ are roots of:
\[x^2-1 \mod 8\]

\subsubsection{Notes}
Considering the case $n=15$ since it's the lowest number with $2<\lambda(n)<\phi(n)$ as $\lambda(15) = 4$ and $\phi(15) = 8$.

\[x^4-1 = (x-1)(x-2)(x^2+3x+7)\]
\[x^4-1 = (x-1)(x-14)(x^2+1)\]
\[x^4-1 = (x-1)(x-8)(x^2+9x+13)\]
