% Copyright 2024 Kieran W Harvie. All rights reserved.

\section{gcd Matrix}
Let $A$, $B$, and $C$ be pairwise coprime.
Bézout's identity guaranties that there exists $m_{n,m}$ such that:
\[\begin{aligned}
	m_{2,1}A+m_{2,3}C&=1\\
	m_{1,1}B+m_{1,2}C&=1\\
	m_{3,2}A+m_{3,3}B&=1\\
\end{aligned}\]
They are written with matrix like indices because they can be used as one:
\[
	\begin{bmatrix}
		m_{1,1}&m_{1,2}&0\\
		m_{2,1}&0&m_{2,3}\\
		0&m_{3,2}&m_{3,3}\\
	\end{bmatrix}
	\begin{bmatrix}
		AB\\AC\\BC
	\end{bmatrix}
	=
	\begin{bmatrix}
		A\\B\\C
	\end{bmatrix}
\]
As well as:
\[
	\begin{bmatrix}
		m_{1,1}&m_{1,2}&0\\
		m_{2,1}&0&m_{2,3}\\
		0&m_{3,2}&m_{3,3}\\
	\end{bmatrix}
	\begin{bmatrix}
		B&A&0\\
		C&0&A\\
		0&C&B\\
	\end{bmatrix}
	=
	\begin{bmatrix}
		1&m_{1,1}A&m_{1,2}A\\
		m_{2,1}B&1&m_{2,3}B\\
		m_{3,2}C&m_{3,3}C&1\\
	\end{bmatrix}
\]
The right hand side has a cool affine diagonal form:
\[
	\begin{bmatrix}
		1&m_{1,1}A&m_{1,2}A\\
		m_{2,1}B&1&m_{2,3}B\\
		m_{3,2}C&m_{3,3}C&1\\
	\end{bmatrix}
	=
	\begin{bmatrix}
		A&0&0\\
		0&B&0\\
		0&0&C\\
	\end{bmatrix}
	\begin{bmatrix}
		0&m_{1,1}&m_{1,2}\\
		m_{2,1}&0&m_{2,3}\\
		m_{3,2}&m_{3,3}&0\\
	\end{bmatrix}
	+
	\begin{bmatrix}
		1&0&0\\
		0&1&0\\
		0&0&1\\
	\end{bmatrix}
\]
And the missing diagonal of the second factor also has a cool form:
\[
	\begin{bmatrix}
		m_{1,1}&m_{1,2}&0\\
		m_{2,1}&0&m_{2,3}\\
		0&m_{3,2}&m_{3,3}\\
	\end{bmatrix}
	\begin{bmatrix}
		0&0&x\\
		0&y&0\\
		z&0&0\\
	\end{bmatrix}
	=
	\begin{bmatrix}
		0&m_{1,2}&m_{1,1}\\
		m_{2,3}&0&m_{2,1}\\
		m_{3,3}&m_{3,2}&0\\
	\end{bmatrix}
	\begin{bmatrix}
		z&0&0\\
		0&y&0\\
		0&0&x\\
	\end{bmatrix}
\]
Substituting and combining it all gives this cool pseudo commutator identity:
\[\begin{aligned}
	&\begin{bmatrix}
		m_{1,1}&m_{1,2}&0\\
		m_{2,1}&0&m_{2,3}\\
		0&m_{3,2}&m_{3,3}\\
	\end{bmatrix}
	\begin{bmatrix}
		B&A&C\\
		C&B&A\\
		A&C&B\\
	\end{bmatrix}\\
	&=
	\begin{bmatrix}
		0&m_{1,2}&m_{1,1}\\
		m_{2,3}&0&m_{2,1}\\
		m_{3,3}&m_{3,2}&0\\
	\end{bmatrix}
	\begin{bmatrix}
		A&0&0\\
		0&B&0\\
		0&0&C\\
	\end{bmatrix}
	+\begin{bmatrix}
		A&0&0\\
		0&B&0\\
		0&0&C\\
	\end{bmatrix}
	\begin{bmatrix}
		0&m_{1,1}&m_{1,2}\\
		m_{2,1}&0&m_{2,3}\\
		m_{3,2}&m_{3,3}&0\\
	\end{bmatrix}
	+
	\begin{bmatrix}
		1&0&0\\
		0&1&0\\
		0&0&1\\
	\end{bmatrix}
\end{aligned}\]
And finally note that the original two matrix identities are related by:
\[
	\frac{1}{2}
	\begin{bmatrix}
	B&A&0\\
	C&0&A\\
	0&C&B\\
	\end{bmatrix}
	\begin{bmatrix}
	A\\B\\C
	\end{bmatrix}
	=
	\begin{bmatrix}
	AB\\AC\\BC
	\end{bmatrix}
\]
