% Copyright 2023,2025 Kieran W Harvie. All rights reserved.

\section{Golden Ratio Recursion}

\subsection{Pre-2023 Original}
The golden ratio and its negated reciprocal are given by:
\[\varphi = \frac{1+\sqrt{5}}{2},\,-\varphi^{-1} = \frac{1-\sqrt{5}}{2}\]
Notice that that we can use them in the following factorization: 
\begin{equation*}
\begin{aligned}
	x^5+1 =& (x+1)(x^4-x^3+x^2-x+1)\\
	=& (x+1)(x^2-\varphi x+1)(x^2+\varphi^{-1} x+1)\\
\end{aligned}
\end{equation*}
Using partial fractions, 
it's easier to do in two steps,
we get:
\begin{equation*}
\begin{aligned}
	\frac{1}{1+x^5}=& \frac{1}{5}\left(\frac{1}{1+x} +\frac{-x^3+2x^2-3x+4}{x^4-x^3+x^2-x+1}\right)\\
	=& \frac{1}{5}\left(\frac{1}{1+x}+\frac{\varphi^{-1}x+2}{x^2+\varphi^{-1} x+1}+\frac{-\varphi x+2}{x^2-\varphi x + 1}\right) \\
\end{aligned}
\end{equation*}
Which gives a cool recursion in the normal way.

\subsection{Pre-2023 Criticisms}
There's some criticism of the Pre-2023 section that I want to note before moving on to two new approaches.
\\

First is that I didn't actually bother to calculate the fractions in the right hand side.
Second is that there's no need to actually gives values for the golden ratio or its negated reciprocal as the factorization cleanly follows from:
\[\varphi^2=\varphi+1\]
Third is that multiplying by just $x+1$ kind of muddies the waters.
Instead directly identify the product with the $10^\text{th}$ cyclotomic polynomial:
\[x^4-x^3+x^2-x+1 = \Phi_{10}(x)\]
Where the roots can directly be identified with the primitive roots of order $10$.
\\

If you aren't comfortable with cyclotomic polynomial you should times by both $x+1$ and $x^5-1$ to get:
\[(x^5-1)(x+1)(x^4-x^3+x^2-x+1) = x^{10}-1\]
As the alternating sign means $\frac{1}{x^5+1}$ actually has a period of $x^{10}$ and not $x^5$,
as it might look at first:
\[\frac{1}{x^5+1} = 1-x^5+x^{10}-x^{15}+x^{20}-\dots\]

\subsection{Cyclotomic}
Let $\omega$ be the $10^\text{th}$ root of unity with the smallest positive argument\footnote{Is there a name for this?},
from the factorization:
\[(x^2-\varphi x +1)(x^2+\varphi^{-1}x+1)=\Phi_{10}(x)\]
There's only two pairs of conjugate primate roots of unity of order $10$,
one pair with a real component less than $0$ the other greater:
\begin{center}
\begin{tikzpicture}
	\pgfdeclarelayer{foreground}
	\pgfsetlayers{main,foreground}

	\begin{pgfonlayer}{foreground}
	\foreach \o [count=\n] in {10,5,10,5,2,5,10,5,10,1}
	{\begin{scope}[rotate=36*\n,pin distance = 5]
		\node[fill,shape=circle,scale=0.5,pin=36*\n:{$(\omega^{\n},\o)$}] (\n) at (3,0) {};
	\end{scope}}
	\end{pgfonlayer}

	\draw[red,dashed] (1) -- (9) node[midway,left] {$\varphi$};
	\draw[blue,dashed] (3) -- (7) node[midway,left] {$-\varphi^{-1}$};
	\draw[gray,dashed] (0,0) circle (3);
\end{tikzpicture}
\end{center}
Meaning we have:
\[ \varphi = \omega+\omega^9,\,-\varphi^{-1} = \omega^3+\omega^7\]
So can write the fractions as:
\[\frac{-\varphi x+2}{x^2-\varphi x + 1} = \frac{1}{1-\omega x}+\frac{1}{1-\omega^9 x}\]
And:
\[\frac{\varphi^{-1} x+2}{x^2+\varphi^{-1} x + 1} = \frac{1}{1-\omega^3 x}+\frac{1}{1-\omega^7 x}\]

\subsection{Recursive}
Let $a_n$ and $b_n$ be integers and consider the following ansatz:
\[\frac{-\varphi x+2}{x^2-\varphi x+1} = \sum_{n=0}^\infty x^n(\varphi a_n+b_n)\]
Where the idea is that the $\varphi^2=\varphi+1$ can be used to remove higher powers from the standard recursion:
\[\begin{aligned}
	&(\varphi a_n+b_n)-\varphi(\varphi a_{n-1}+b_{n-1})+(\varphi a_{n-2} + b_{n-2}) = 0\\
	\Rightarrow&\varphi(a_n-a_{n-1}-b_{n-1}+a_{n-2})+(b_n+b_{n-2}-a_{n-1})=0\\
\end{aligned}\]
And the first two values can be directly be calculated as:
\[\begin{aligned}
	\frac{-\varphi x+2}{x^2-\varphi x+1}\bigg|_{x=0}& = 2\\
	\frac{\D}{\D x}\frac{-\varphi x+2}{x^2-\varphi x+1}\bigg|_{x=0}& = \frac{-\varphi(x^2-\varphi x+1)-(-\varphi x+2)(2x-\varphi)}{(x^2-\varphi x+1)^2}\bigg|_{x=0} = \varphi\\
\end{aligned}\]
Giving the following table:
\begin{center}
\begin{tikzpicture}
	\matrix(m) [matrix of math nodes]
	{
		n&0&1&2&3&4&5&6\\
		a_n&0&1&1&-1&-1&0&-1\\
		b_n&2&0&-1&1&0&-2&0\\
	};

	\draw[dashed,rounded corners,red] (m-2-2.north west) rectangle (m-3-6.south east);
	\draw[dashed,rounded corners,blue] (m-2-7.north west) rectangle (m-3-8.south east);
\end{tikzpicture}
\end{center}
The second block is the first two rows of the first block with the signs flipped.
Since the recursion is only two deep and linear this shows that the generating function is periodic with a period of $x^{10}$ where the second half is the first with the sign flipped.
\\

Note that since $-\varphi^{-1}$ also satisfies $(-\varphi^{-1})^2 = -\varphi^{-1}+1$ the coefficients are the same for:
\[\frac{\varphi^{-1} x+2}{x^2+\varphi^{-1} x+1} = \sum_{n=0}^\infty x^n(-\varphi^{-1} a_n+b_n)\]
If you want it to be symmetric between $\varphi$ and $-\varphi^{-1}$ you can use $\varphi-\varphi^{-1}=1$ to give:
\[\begin{aligned}
	\frac{\varphi^{-1} x+2}{x^2+\varphi^{-1} x+1} &= \sum_{n=0}^\infty x^n(-\varphi^{-1} a_n+(\varphi-\varphi^{-1})b_n\\
	&= \sum_{n=0}^\infty x^n(-\varphi^{-1} (a_n-b_n)+\varphi b_n)\\
\end{aligned}\]
But it might worth symmetrizing later to cleanly get:
\[\begin{aligned}
	\frac{-\varphi x+2}{x^2-\varphi x+1} + \frac{\varphi^{-1} x+2}{x^2+\varphi^{-1} x+1} &=\sum_{n=0}^\infty x^n((\varphi-\varphi^{-1}) a_n+2b_n) \\
	&=\sum_{n=0}^\infty x^n( a_n+2b_n) \\
\end{aligned}\]
Which gives the following table.
\begin{center}
\begin{tikzpicture}
	\matrix(m) [matrix of math nodes]
	{
		n&0&1&2&3&4\\
		a_n&0&1&1&-1&-1\\
		b_n&2&0&-1&1&0\\
		a_n+2b_n&4&1&-1&1&-1\\
	};
\end{tikzpicture}
\end{center}
As expected.
\\

Also note that we have this cute matrix equation:
\[\begin{bmatrix}
	1&1&-1&0\\
	1&0&0&-1\\
	1&0&0&0\\
	0&1&0&0\\
\end{bmatrix}
\begin{bmatrix}
	a_{n-1}\\b_{n-1}\\a_{n-2}\\b_{n-2}\\
\end{bmatrix}
=
\begin{bmatrix}
	a_n\\b_n\\a_{n-1}\\b_{n-1}\\
\end{bmatrix}\]
The top left corner is the matrix used in the classic Fibonacci sequence $F_n$ equation:
\[\begin{bmatrix}
	1&1\\
	1&0\\
\end{bmatrix}
\begin{bmatrix}
	F_{n-1}\\F_{n-2}\\
\end{bmatrix}
=
\begin{bmatrix}
	F_n\\F_{n-1}\\
\end{bmatrix}\]
The bottom left corner is identity, the top right flip signs, and the bottom right is $0$.
