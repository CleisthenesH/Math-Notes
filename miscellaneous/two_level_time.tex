% Copyright 2023 Kieran W Harvie. All rights reserved.

\section{Time Evolution of a Two Dimensional Quantum System}
I was curious about the evolution of two dimensional quantum systems and decided to do a refresher. 
\\

Like all time-dependent problems start with the Schrödinger equation:
\[i\hbar \frac{\partial}{\partial t}\Psi = H\Psi\]
If $\Psi$ is an eigenvector of $H$ then we have:
\[i\hbar \frac{\partial}{\partial t}\Psi = E\Psi \Rightarrow \Psi = \exp(-i\hbar E t)\Psi_0\]
\\

A two-dimensional system where the energy levels are the same has a trivial evolution,
both gain phase at the same rate resulting in no observable change,
so assume there are two eigenvalues $\mu\pm\delta$.
\\

Let use working the orthonormal base of $H$ and define:
\[Z = \begin{bmatrix} 1&0\\0&-1\end{bmatrix},\quad E = \begin{bmatrix} \mu-\delta&0\\0&\mu+\delta\end{bmatrix}=\mu I -\delta Z\]
Then we have:
\[\Psi = \exp(-i\hbar E t)\Psi_0 = \exp(-i\hbar \mu t)\exp(i\hbar \delta Zt)\Psi_0\]
We will ignore the first factor as it's only relevant if working with a super system.

Consider the function $f(x) = \exp(i\hbar\delta t x)$ and apply the Cayley–Hamilton theorem to obtain:
\begin{equation*}
\begin{aligned}
\Psi =& \left(\exp(-i\hbar\delta t)\frac{Z+I}{2}-\exp(i\hbar\delta t)\frac{Z-I}{2}\right)\Psi_0 \\
=& \big[\cos(\hbar\delta t)I-i\sin(\hbar\delta t)Z\big]\Psi_0 \\
\end{aligned}
\end{equation*}
Observer that when $\delta t \rightarrow 0$ we have $\Psi \rightarrow \Psi_0$. 
This has two obvious sanity checks.
That as we return to the beginning, $t\rightarrow 0$, the state returns to $\Phi_0$ regardless of $\delta$.
And as the difference of the energy level shrinks, $\delta\rightarrow 0$, the state returns to $\Phi_0$ regardless of $t$.
\\

Other results include:

That for small $\delta$ we can keep the system in the $\Psi_0$ state for an inversely proportional time

That the system returns to $\Psi_0$ with a period of $\frac{\pi}{\hbar\delta}$.
Notice that the period is halved from the naïve value, 
as we only require the $\sin$ term to vanish.

\subsection{Change of Basis}
Assume the $\Psi_0$ isn't given in the orthonormal base of $H$.
Then we will need a change of basis matrix $P$ and have:
\[\Psi = P^{-1}\exp(-i\hbar E t)P\Psi_0\]
We can just substitute our worked expression for $\exp(-i\hbar Et)$ to give
\[\Psi = \big[\cos(\hbar\delta t)I-i\sin(\hbar\delta t)P^{-1}ZP\big]\Psi_0 \]
But lets pretend we didn't have the worked form,
what would we do?
\\

Well, interestingly we have:
\[P^{-1}\exp(-i\hbar E t)P\Psi_0= \exp(-P^{-1}i\hbar E tP)\Psi_0\]
So things don't change much yet.
And when it comes to applying the Cayley-Hamilton theorem we need the characteristic polynomials:
\begin{equation*}
\begin{aligned}
p_{P^{-1}MP}(x) =& \det(P^{-1}MP-xI) \\
=&\det(P^{-1}(M-xI)P) \\
=&\det(P^{-1})\det(M-xI)\det(P) \\
=&\det(M-xI) \\
\end{aligned}
\end{equation*}
Meaning the roots, eigenvalues, of one are the roots of another.
Meaning the result of Cayley-Hamilton is the same but with a different matrix subbed in.

From this point it's just algebra and left to the reader.
But what's interesting is that all the operations involved worked so well with:
\[M\mapsto P^{-1}MP\]
Which might be physically motivated.

In abstract algebra we would call this type of map a "conjugation", 
but that name already has a meaning in linear algebra,
so maybe "inner-automorphism" would be better.
