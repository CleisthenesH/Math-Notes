% Copyright 2023 Kieran W Harvie. All rights reserved.

\section{Special Relativity}
The plan:
Special Relativity is the study of affine transformations in a hyperbolic pseudo-Riemann space.
It makes sense to study hyperbolic Riemann space to get a foot hold.
In particular perspective space and the technique of homogeneous coordinates. 

Kids running at constant speed in a field (special) and hill (general)
Use electromagnetism as a testing device.

\subsection{Homogeneous coordinates}
\subsubsection{Basics}
First start with affine transforms.
Consider the composition of a rotation and translation in the Cartesian plane.
\[\begin{bmatrix} x'\\y' \end{bmatrix} = \begin{bmatrix}\cos(\theta) & -\sin(\theta)\\ \sin(\theta) & -\cos(\theta)\end{bmatrix}
	\begin{bmatrix} x\\ y\end{bmatrix}+ \begin{bmatrix}T_x\\T_y\end{bmatrix}\]
With the addition of an extra bit of information we can encode all this information in one matrix:
\[\begin{bmatrix} 1\\x'\\y' \end{bmatrix} = \begin{bmatrix}1&0&0 \\ T_x & cos(\theta) & -\sin(\theta) \\ T_y & \sin(\theta) & -\cos(\theta) \end{bmatrix}
\begin{bmatrix} 1\\x\\y \end{bmatrix}
\]
The top row stores projective information, the left column stores translational information, and the rest stores rotational information.

The benefit to this formalism is that it is also easy to do perspective calculations, that is figure out what someone standing at the origin sees.
I forget the form of the 2D projection matrix, but rest easy that it has a different top row then the rest.
\subsubsection{Lorentz force}
Consider the Lorentz force law:
\[F = q(E+v\times B)\]
Remembering that the cross product can be expressed as a matrix multiplication as:
\[v\times B = \begin{bmatrix}0 & B_z & -B_y \\ -B_z & 0 & B_x \\ B_y & -B_x & 0 \end{bmatrix}\begin{bmatrix}v_x \\ v_y \\ v_x \end{bmatrix}\]
Using homogeneous coordinates we can combine this with the electrical component to obtain.
\[ F= q\begin{bmatrix}1 & 0 & 0 & 0 \\E_x & 0 & B_z & -B_y \\ E_y & -B_z & 0 & B_x \\ E_z & B_y & -B_x & 0 \end{bmatrix}\begin{bmatrix}1 \\v_x \\ v_y \\ v_x \end{bmatrix}\]
This is remarkably close to the correct Electromagnetic tensor.

\subsection{pseudo-Riemann space}
Three main changes occurs when converting this idea to pseudo-Riemann form.
The first two are easy.
Firstly constant we chose at the start of the vector is $c$ and the whole vector is multiplied by the Lorentz factor 
Secondly the first element has a different sign then the rest.

These first two corrections are easily fixed:
\[ F= q\gamma\begin{bmatrix}1 & 0 & 0 & 0 \\-E_x/c & 0 & B_z & -B_y \\ -E_y/c & -B_z & 0 & B_x \\ -E_z/c & B_y & -B_x & 0 \end{bmatrix}\begin{bmatrix}c \\v_x \\ v_y \\ v_x \end{bmatrix}\]

The correct tensor is this:
\[ F= q\gamma\begin{bmatrix}0 & E_x/c & E_y/c & E_z/c \\-E_x/c & 0 & B_z & -B_y \\ -E_y/c & -B_z & 0 & B_x \\ -E_z/c & B_y & -B_x & 0 \end{bmatrix}\begin{bmatrix}c \\v_x \\ v_y \\ v_x \end{bmatrix}\]

And we can see that the new tensor is the same as the old one bar perspective information.
This should also be a clue that the speed of light being constant is related to perspective.

This new perspective information is because a four force the zeroth component is the change of energy with time.
Which is:
\[v\cdot E\]
Which matches the given expression, if you expand it.

\subsection{Running in a field}
Imagine people running in a flat field.
All at the same speed abut in different directions.
The runner preserves their direction at time and the perpendicular axis of time.

The runner doesn't preserves their own motion in space but only in time, at the constant speed.
This is analogous to how the world line when traveled in proper time, is always at the speed of light.

The fastest a runner can see another runner run is if that runner runs perpendicularly to the first runner.
In this case the second runner is running at the constant speed, according to the first runner.
This is analogous to there being a maximum speed of light.

Now put a hill in the field.
Imagine someone far away from the hill and a second person running up the hill such that from a birds eye perspective the runners velocities are collinear (line up).
The far runner doesn't see the deflection up the hill so from their perspective they only see the smaller birds eye view projection.
To them the hill runners local time is running slow.

If the hill runner is smart they won't run directly up the fill, they will instead by slightly deflected by it.
This is analogous to how geodesic works.

To the far runner the hill runner is experience a force.
This is analogous to a local time gradient causing a force on the hill.
