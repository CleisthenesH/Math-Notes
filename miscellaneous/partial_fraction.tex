% Copyright 2024 Kieran W Harvie. All rights reserved.

\section{Partial Fractions}
Some observations and proofs of well known facts that I want to write down.

\subsection{Bézout's Identity}
Let $P_n$ be a finitely indexed family of polynomials and define:
\begin{equation*}
\begin{aligned}
Q_n =& \prod_{k\neq n}P_k\\
G =&\gcd(\{Q_n\}_n)\\
\end{aligned}
\end{equation*}
Then there exits polynomials $q_n$ such that:
\[\left(\prod_n P_n\right)^{-1} = G^{-1}\sum_n\frac{q_n}{P_n}\]
And that this solution is "the best we can get" in the sense that if there exits $q'_n$ and $G'$ such that:
\[\left(\prod_n P_n\right)^{-1} = G'^{-1}\sum_n\frac{q'_n}{P_n}\]
Then $\deg(G') \geq \deg(G)$

\subsubsection{Existence}
From Bézout's identity on $\{Q_n\}_n$ there existences $q_n$ such that:
\[\sum_n q_n Q_n = G \,\Rightarrow\, G^{-1}\sum_n q_n Q_n = 1\]
Substitution gives:
\begin{equation*}
\begin{aligned}
\left(\prod_n P_n\right)^{-1} =& \left(\prod_n P_n\right)^{-1}\left(G^{-1}\sum_n q_n Q_n\right)\\
=& G^{-1}\left(\sum_n q_n \left(\prod_{k\neq n} P_k\right)\left(\prod_k P_k\right)^{-1}\right)\\
=& G^{-1}\sum_n\frac{q_n}{P_n}\quad \square\\
\end{aligned}
\end{equation*}

\subsubsection{Best Case}
Assume we have $q_n'$ and $g'$ such that:
\[\left(\prod_n P_n\right)^{-1} = G'^{-1}\sum_n\frac{q'_n}{P_n}\]
Then doing all the previous steps in reverse gives:
\[\sum_n q_n' Q_n = G'\]
Meaning $G | G'$ and hence $\deg(G') \geq \deg(G)\quad\square$.

\subsubsection{Sharpness}
The above inequality is sharp,
this can be seen in the standard way.
Multiply the relevant polynomials by the same element $k$ from underlying ring gives a new solution with $\deg(G') = \deg(G)$:
\begin{equation*}
\begin{aligned}
	q'_n =& kq_n\\
	G' =& kG\\
\end{aligned}
\end{equation*}
We could, however, probably get some interesting results by putting some restrictions on $q_n$ and $P_n$.

\subsubsection{Remarks}
This result matches our intuition when calculating partial fractions,
that shared factors can't easily be removed:

If a polynomial $P$ can be factorized as $P=P_0P_1$ then:
\[\frac{1}{P} = \frac{1}{\gcd(P_0,P_1)}\left(\frac{q_0}{P_0}+\frac{q_1}{P_1}\right)\]
And the "Best Case" condition means we can't get rid of the external fraction.
\\

Also note that this result suggest the application of extended euclidean algorithm to partial fractions,
which sounds like it might be a fun one-day project.

\subsection{Divisibility}
What of the reverse?
If we have:
\[\frac{1}{P} = \frac{q_0}{P_0}+\frac{q_1}{P_1}\]
What can we say of the factorization of $P$?
Well the desired answer,
that $P_0$ and $P_1$ are factors of $P$,
doesn't work:
\[\frac{1}{x-1} = \frac{1}{x-2}-\frac{1}{(x-1)(x-2)}\]
The problem seems to come from cancellation means $P$ can be of lower degree then $P_0$ or $P_1$,
so lets take $\deg(P)$ as a possible condition.
\\

Simple manipulation gives:
\[(q_1P_0+q_0P_1)P = P_0P_1 = \lcm(P_0,P_1)\gcd(P_0,P_1)\]
Since $\gcd(P_0,P_1) | (q_1P_0+q_0P_1)$ we have $P | \lcm(P_0,P_1)$ meaning if $\deg(P) \geq \deg(\lcm(P_0,P_1))$ we have:
\[P = \lcm(P_0,P_1)\]

\subsubsection{Remark}
When preforming calculations we may not know $\deg(\lcm(P_0,P_1))$,
instead we can use $\deg(P) \geq \deg(P_0P_1)$ as:
\[\deg(P_0P_1) \geq \deg(\lcm(P_0,P_1))\]

\subsection{Complex Residue}
The application of the extended euclidean algorithm to finding the complex residues of rational functions is something I've talked about before.
But my prior attempts can be improved with these observations.
\\

Assume us want to find the residues of $P/Q$ at $a$ and $b$ where $Q$ has multiplicities $n$ and $m$.
Let $q$ be the polynomial such that:
\[Q(z) = q(z)(z-a)^n(z-b)^m\]
All three factors are pairwise relatively prime meaning there exists polynomials $s$,$r$, and $t$ such that:
\[s(z)(z-a)^nq(z)+r(z)(z-b)^mq(z)+t(z)(z-a)^n(z-b)^m = 1\]
Giving:
\[\frac{P(z)}{Q(z)} = s(z)\frac{P(z)}{(z-b)^m}+r(z)\frac{P(z)}{(z-a)^n}+t(z)\frac{P(z)}{q(z)}\]
Residues naturally follow.
Interestingly,
this approach might be able to to link the residues at $a$ and $b$ together.

\subsection{Perturbation}
I found this interesting perturbation identity while working:
\begin{equation*}
\begin{aligned}
	\frac{1}{(P_0-\lambda \delta_0)(P_1+\lambda \delta_1)} =& \frac{1}{P_0\delta_1+P_1\delta_0}\left(\frac{\delta_0}{P_0-\lambda \delta_0}+\frac{\delta_1}{P_1+\lambda \delta_1}\right)\\
	=&\frac{1}{P_0\delta_1+P_1\delta_0}\sum_{n=0}^{\infty}\lambda^n\left(\left(\frac{\delta_0}{P_0}\right)^{n+1}-\left(-\frac{\delta_1}{P_1}\right)^{n+1}\right)\\
\end{aligned}
\end{equation*}

This probably only has useful results when $\delta_0$ and $\delta_1$ are chosen such that $P_0\delta_1+P_1\delta_0$ means something.
Still cool.
