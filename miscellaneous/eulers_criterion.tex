% Copyright 2024 Kieran W Harvie. All rights reserved.

\section{Euler's Criterion}
Euler's criterion is and elementary number theory result that states that for odd prime $p$ and coprime non-zero integer $a$ we have:
\[x^{\frac{p-1}{2}}\mod p = \begin{cases} 1 & \text{ if $x$ is a square}\\ -1 & \text{ otherwise.}\end{cases}\]
I proved this the `involved' way back in the day,
but it's really easy to prove and generalize with field theory.

\subsubsection{Field Theory Proof:}
Let $p$ be an odd prime and consider the polynomial $q\in\mathbb{F}_p[X]$:
\[p(x) = x^{\frac{p-1}{2}}-1\]

\textbf{Lemma, every non-zero square is a root of $p$:}
Let $a=b^2$ then:
\[p(a) = p(b^2) = b^{p-1}-1 = 0\,\, \square\]

\textbf{Lemma, there are at least $\frac{p-1}{2}$ non-zero squares in $\mathbb{F}_p$:}
Let $a\in \mathbb{F}_p$ and consider the polynomial:
\[q(x) = x^2-a\]
There are at most two distinct roots of $q$ and when $a\neq 0$ any roots are non-zero.
However there are $p-1$ distinct non-zero elements of $\mathbb{F}_p$ and the square of every non-zero element is non-zero,
but no more than two can share the same value when squared,
hence there are at least $\frac{p-1}{2}$ non-zero squares in $\mathbb{F}_p$. $\square$
\\

Combining the previous results means every root of $p$ is distinct and a square.
The only loose end is to prove that $x^\frac{p-1}{2}=-1$ when $x$ isn't a square.
\\

From the previous result when $x$ isn't a square we know $x^\frac{p-1}{2}\neq 1$.
$0$ is a square, hence $x$ is non-zero meaning:
\[0 = x^{p-1}-1=\left(x^{\frac{p-1}{2}}-1\right)\left(x^{\frac{p-1}{2}}+1\right)\]
Since the first factor isn't zero the second must be,
giving $x^{\frac{p-1}{2}}=-1$, as required.
\\

This result easily generalizes to all finite fields of odd size.
Which are fields of odd prime characteristic,
cool.

\subsubsection{Low Hanging Generalizations:}
Let $\mathbb{F}_n$ be a finite field and suppose $n-1$ is divisible by $m$ and let:
\[p(x) = x^\frac{n-1}{m}-1\]
Similar arguments show that all the roots of $p$ are distinct and are the non-zero $m$th powers.
Now it's only the loose ends of non-powers that matter.
\\

Let $h$ be an element of order $m$.
(Guaranteed since every finite field has a generator $g$ which you can raise to power $g^\frac{n-1}{m}$).
And let $q$ be:
\[q(x)=x^m-1\]
From similar arguments as those above,
you can show that the powers of $h$ are the roots of $q$:
\[q(x) = x^m-1= \prod_{k=1}^m(x-h^k)\]
Substitute $x\mapsto x^\frac{n-1}{m}$ to give:
\[\prod_{k=1}^m(x^\frac{n-1}{m}-h^k)= (x^\frac{n-1}{m})^m-1=x^{n-1}-1\]
Hence for non-zero $x$ we have:
\[\prod_{k=1}^m(x^\frac{n-1}{m}-h^k)= 0\]
Meaning $x^\frac{n-1}{m}$ is a power of $h$.
\\

We can combine these results in a nice way:
Let $m$ divide $n-1$ and let $h$ be an element of order $m$.
Define $f_m:\mathbb{F}_n^\times\rightarrow \langle h\rangle \cong C_m$ such that $f_m(x) = x^{\frac{n-1}{m}}$
$f_m$ is a group homomorphism where the kernel is the $m$th powers of $\mathbb{F}_n$

%Suppose $n-1$ is divisible by $3$ and let $h$ be an element of order $3$.
%(Guaranteed since every finite field has a generator $g$ which you can raise to power $g^\frac{n-1}{3}$).
%We have:
%\begin{equation*}
%\begin{aligned}
%&\left(x^\frac{n-1}{3}-1\right)\left(x^\frac{n-1}{3}-h\right)\left(x^\frac{n-1}{3}-h^2\right)\\
%=&x^{n-1}-x^\frac{2(n-1)}{3}\left(1+h+h^2\right) +x^\frac{n-1}{3}\left(1\cdot h+1\cdot h^2+h\cdot h^2\right)-1\cdot h\cdot h^2\\
%=&x^{n-1}-1\\
%\end{aligned}
%\end{equation*}
%Easy to see that:
%\[\img\left(x^\frac{n-1}{m}\right)=\langle h \rangle \cong C_3\]
%and
%\[\ker\left(x^\frac{n-1}{m}\right)=\{\text{$m$th powers}\}\]
%
%Unfortunately this doesn't work for $4$ and I'll have to actually think,
%will probably use:
%\[\ord(x^k) = \frac{\ord(x)}{\gcd(k,\ord(x))}\]
