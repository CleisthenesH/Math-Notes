% Copyright 2024 Kieran W Harvie. All rights reserved.

\section{Pauli Matrices}
The Pauli matrices are a collection of three traceless, unitary, Hermitian, and involutory $2\times2$ complex matrices:
\[
	\sigma_x = \begin{bmatrix}0&1\\1&0\\\end{bmatrix}\,
	\sigma_y = \begin{bmatrix}0&-i\\i&0\\\end{bmatrix}\,
	\sigma_z = \begin{bmatrix}1&0\\0&-1\\\end{bmatrix}
\]
Why these properties are important and why we chose these specific matrices,
they are not the only set that satisfy these properties,
are physically motivated but the matrices have an interesting mathematical structure by themselves.
They are useful for analysis of quaternions and $SU(2)$ for example.

\subsection{Basic Properties:}
\subsubsection{Products:}
There's only three Paul matrices so directly constructing a multiplication table is easy:
\[\begin{array}{|c|c|c|c|}
	\hline
	\rightarrow\times\downarrow&\sigma_x&\sigma_y&\sigma_z\\
	\hline
	\sigma_x&I&i\sigma_z&-i\sigma_y\\
	\sigma_y&-i\sigma_z&I&i\sigma_x\\
	\sigma_z&i\sigma_y&-i\sigma_x&I\\
	\hline
\end{array}\]
If we impose the standard order on $(x,y,z)$ we can compress this table to:
\[\sigma_j\sigma_k = \delta_{j,k}I+i\varepsilon_{j,k,l}\sigma_l\]

\subsubsection{Pauli Vectors:}
The most useful tool for Pauli matrices is a bundling called a Pauli vector:
\[
	(a_x,a_y,a_z)\cdot\sigma = 
	a_x\begin{bmatrix}0&1\\1&0\\\end{bmatrix}+
	a_y\begin{bmatrix}0&-i\\i&0\\\end{bmatrix}+
	a_z\begin{bmatrix}1&0\\0&-1\\\end{bmatrix}
	=\begin{bmatrix}a_z&a_x-ia_y\\a_x+ia_y&-a_z\end{bmatrix}
\]
By inspection all Pauli vectors are still traceless and their determinant is given by:
\[\begin{aligned}
	\det(a\cdot\sigma) &= \det\begin{bmatrix}a_z&a_x-ia_y\\a_x+ia_y&-a_z\end{bmatrix}\\
	&= -a_z^2-(a_x-ia_y)(a_x+ia_y)\\
	&= -(a_z^2+a_x^2+a_y^2)\\
\end{aligned}\]
Hence $a\cdot\sigma$ has the eigenvalues of:
\[\lambda_\pm = \pm\sqrt{a_z^2+a_x^2+a_y^2}\]
Perhaps the most useful property of Pauli vectors is their product,
which follows naturally from the product of Pauli matrices:
\[\begin{aligned}
(a\cdot \sigma)(b\cdot \sigma) &= a^j\sigma_jb^k\sigma_k\\
&= a^jb^k\delta_{j,k}I+ia^jb^k\varepsilon_{j,k,l}\sigma_l\\
&= (a\cdot b)I+i(a\times b)\cdot \sigma\\
\end{aligned}\]

\subsubsection{Bases and Convention:}
When $a$ has solely real element the Pauli vector $a\cdot\sigma$ his Hermitian.
This can be verified by inspection as the main diagonal elements are real and the minor diagonal elements are conjugates of each other.
It follows that the Pauli matrices form the basis for the real vector space of traceless $2\times2$ Hermitian matrices and that the Pauli matrices combined with $I$ form a basis for the real vector space of general $2\times2$ Hermitian matrices.
\\

It for these reasons that $a$ is often defined to have only real elements and that $I$ is sometimes called the $0^\text{th}$ Pauli matrix.
Since the abuse of notation by allowing complex numbers in Pauli vectors is somewhat common, 
and there is no universal convention on whether or not $I$ is a Pauli matrix,
I've decided to be general in previous results.

\subsubsection{Functions on Pauli vectors:}
A unit Pauli vector is one whose determinant equals $-1$ and we can construct a unit Pauli vector from a general one through: 
\[\hat{a}\cdot\sigma = \frac{a\cdot\sigma}{\sqrt{a_z^2+a_x^2+a_y^2}}\]
Observe that when $a$ is a real vector that this matches the normal usage of the word unit.
The use of unit Pauli vectors lets us write Sylvester's formula as:
\[f(t\hat{a}\cdot\sigma)=I\frac{f(t)+f(-t)}{2}+\hat{a}\cdot\sigma\frac{f(t)-f(-t)}{2}\]
We can see that $t$ correlates to length and $\hat{a}$ to direction.
A particularly interesting result is:
\[\exp(it(\hat{a}\cdot\sigma))=I\cos(t)+i(\hat{a}\cdot\sigma)\sin(t)\]
A clear generalization of Euler's formula.

\subsection{Physical Motivation:}
To understand why Hermitian matrices are important in physics understand that systems are commonly represented as vectors, $v$ and $u$, and operations as matrices $M$.
When these operations are observables their observed values are calculated as the sum of the terms like:
\[u^*Mv\quad \tr(vu^*M)\]
For pure and mixed states respectfully.
Such calculations can be difficult,
so how best to chose a basis $b_i$,
and co-basis $b^i = b_i^*$,
for calculation?
Well if all vectors are valid system states and the observed values are real numbers,
normal assumptions,
then we require:
\[b^iMb_j\quad \tr(b_ib^jM)\]
Be real as well.
It would also be convenient if the basis vectors where orthonormal:
\[b^iMb_j = b^im^k_jb_k=m^k_j\delta_{k,i}=m^i_j\]
\[\tr(b_ib^jM) = \tr(Mb_ib^j) = \tr(b_im^k_jb^k) = m^k_j\delta_{i,j}=m^i_j\]
And eigenvectors:
\[b^iMb_j=\lambda^jb^ib_j\]
\[\tr(b_ib^jM) = \tr(Mb_ib^j) = \lambda^i\tr(b_ib^j)\]
Fortunately there's and important result called the spectral theorem that means we can have both:
\begin{quote}
If $H$ is finite square Hermitian matrix on vector space $V$ there exits an orthonormal basis of $V$ consisting of eigenvectors of $H$ and each eigenvalue is real.
\end{quote}
And further more that a matrix $H$ having real eigenvalue and being diagonalizable $D$ by a pair of unitary matrices $U$:
\[ U^*HU = D \text{ where } U^*U=I\]
It the same as $H$ being Hermitian!
Meaning it's both convenient and required that observables have Hermitian matrices.
\\

For completeness sake the combined use of orthonormal eigenvectors is included below:
\[b^iMb_j = b^i\lambda^j b_j = \lambda^j\delta^i_j\]
\[\tr(b_ib^jM) = \tr(Mb_ib^j) = \lambda^i\tr(b_ib^j) = \lambda^i\delta_i^j\]

\subsubsection{Operator:}
If we consider $2$D systems and choose the base of that state vector space such that:
\[\begin{bmatrix}x\\y\end{bmatrix} = x\begin{bmatrix}1\\0\end{bmatrix}+y\begin{bmatrix}0\\1\end{bmatrix}\]
Corresponds to $x$ amount and phase of the `up' state and $y$ the `down' state.
In this case the Pauli matrices are the observables of the projection of angular momentum on the axes.
For intuition as to why calculate the inner products:
\[\begin{aligned}
	\begin{bmatrix}\bar{x}&\bar{y}\end{bmatrix}\begin{bmatrix}0&1\\1&0\end{bmatrix}\begin{bmatrix}x\\y\end{bmatrix}=&[2\re(\bar{x}y)]\\
	\begin{bmatrix}\bar{x}&\bar{y}\end{bmatrix}\begin{bmatrix}0&-i\\i&0\end{bmatrix}\begin{bmatrix}x\\y\end{bmatrix}=&[2\im(\bar{x}y)]\\
	\begin{bmatrix}\bar{x}&\bar{y}\end{bmatrix}\begin{bmatrix}1&0\\0&-1\end{bmatrix}\begin{bmatrix}x\\y\end{bmatrix}=&[|x|^2-|y|^2]\\
\end{aligned}\]
The third result is the most initiative,
norm of up minus norm of down,
but even if you can't currently see why angular momentum would be related like this you can at lest see that such simple forms would be useful.
\\

This result naturally extends to Pauli Vectors:
\[\begin{aligned}
	\begin{bmatrix}\bar{x}&\bar{y}\end{bmatrix}a\cdot\sigma\begin{bmatrix}x\\y\end{bmatrix}
	=a\cdot(2\re(\bar{x}y),2\im(\bar{x}y),|x|^2-|y|^2)\\
\end{aligned}\]
Hence when $a$ is a real vector then $\hat{a}\cdot\sigma$ is the observable for angular momentum on the $a$ axis.

\subsubsection{State:}
Let $a$ be a real vector and consider matrices of the following form:
\[\rho = \frac{1}{2}(I+a\cdot\sigma)\]
We want to show that it's a density matrix.
By inspection it has trace of $1$ so consider the following quadratic form:
\[\begin{aligned}
	\begin{bmatrix}\bar{x}&\bar{y}\end{bmatrix}\rho\begin{bmatrix}x\\y\end{bmatrix}
	=&\frac{1}{2}\left(\begin{bmatrix}\bar{x}&\bar{y}\end{bmatrix}I\begin{bmatrix}x\\y\end{bmatrix}+\begin{bmatrix}\bar{x}&\bar{y}\end{bmatrix}a\cdot\sigma\begin{bmatrix}x\\y\end{bmatrix}\right)\\
	=&\frac{1}{2}\left(|x|^2+|y|^2+a\cdot(2\re(\bar{x}y),2\im(\bar{x}y),|x|^2-|y|^2)\right)\\
\end{aligned}\]
Through standard inequalities this expression is positive when $|a|^2\leq 1$.
Hence $\rho$ is positive semi-definite a well traceless and hence is a density matrix under this condition.
But what state does it represent?
Applying the Pauli vectors as a operators gives:
\[\begin{aligned}
	\langle \hat{b}\cdot\sigma\rangle =& \tr(\rho \hat{b}\cdot\sigma)\\
	=&\tr\left(\frac{1}{2}(I+a\cdot\sigma)\hat{b}\cdot\sigma\right)\\
	=&\tr\left(\frac{1}{2}(\hat{b}\cdot\sigma+(a\cdot \hat{b})I+i(a\times \hat{b})\cdot\sigma)\right)\\
	=&\frac{1}{2}\left(\tr(\hat{b}\cdot\sigma)+\tr((a\cdot \hat{b})I)+\tr(i(a\times \hat{b})\cdot\sigma)\right)\\
	=&a\cdot \hat{b}\\
\end{aligned}\]
Hence $\rho$ is the state with angular momentum on the $a$ axis and $|a|$ relates to the purity of that state.
\\

To see this clearly we will evaluate $\det(\rho)$ and $\rho^2$.
To get the determinant we will make use of this cool,
and easy to directly algebraically verify, 
identity for $2\times2$ matrices:
\[\det(A+B) = \det(A)+\det(B) +\tr(A)\tr(B)-\tr(AB)\]
Giving:
\[\begin{aligned}
	\det(\rho)&=\det\left(\frac{1}{2}(I+a\cdot\sigma)\right) \\
	&=\frac{1}{4}\det\left(I+a\cdot\sigma\right) \\
	&=\frac{1}{4}\left(\det\left(I\right)+\det\left(a\cdot\sigma\right) +\tr\left(I\right)\tr\left(a\cdot\sigma\right)-\tr\left(a\cdot\sigma\right)\right)\\
	&=\frac{1}{4}(1-|a|^2)\\\\
	\rho^2 &= \frac{1}{4}(I+2a\cdot\sigma+(a\cdot\sigma)^2)\\
	&= \frac{1}{4}(I+2a\cdot\sigma+a^2I))\\
	&= \rho +\frac{1-a^2}{4}I\\
	&= \rho +\det(\rho)I\\
\end{aligned}\]
Hence the state is pure when $|a|^2=1$,
or equivalently the determinant is $0$,
and is a mixed state otherwise.
All these results give rise to an influential representation of $2$D states\ldots

\subsubsection{The Bloch Sphere:}
Since every $2$D state must have angular momentum along some axis there is a bijection between states and real $3$D vectors $a$ such that $|a|\leq1$.
The set of real $3$D vectors such that $|a|\leq1$ is easily interpreted as a unit ball where the sphere around this ball has $|a|=1$.
We have previously shown that states with $|a|=1$ are pure states meaning every state on this sphere is a pure state and all the states inside are mixed.
Because this sphere is so important it's named the Bloch sphere,
and similarly the general states $\rho$ with $|a|\leq 1$:
\[\rho = \frac{1}{2}(I+a\cdot\sigma)\]
Are called Bloch vectors.
Finally,
notice that we can recover the original Pauli matrices when $a$ has one $1$ element and the rest zero.
Such states are trivially on the Bloch sphere.
In particular the Pauli operator corresponding to the original `up' and `down' axis can be recovered by $(0,0,\pm1)$ making them antipodal points on the sphere and giving it a natural orientation where these points are on the vertical axis,
like the north and south poles of earth:

\begin{center}
\begin{tikzpicture}
	\draw(0,0) circle (4cm);
	\draw (-4,0) arc (180:360:4 and 1.2);
	\draw[dashed] (4,0) arc (0:180:4 and 1.2);
	\fill[fill=black] (0,4) circle (1pt) node[above] {$\begin{bmatrix}1\\0\end{bmatrix}$};
		\fill[fill=black] (0,-4) circle (1pt)node[below]{$\begin{bmatrix}0\\1\end{bmatrix}$};;
	\draw[->,red,thick] (0,0) -- (4,0) node[right] {Pure State};
	\draw[->,red,thick] (0,0) -- (0,2) node[above]{Mixed State};
\end{tikzpicture}
\end{center}
