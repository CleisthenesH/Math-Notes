% Copyright 2023 Kieran W Harvie. All rights reserved.

\section{Padé Approximant}
We wish to approximate a function $f$ by creating a rational function that agrees with $f$'s first $N$ derivatives at zero.

Let $T_N$ be the $N$th degree Maclaurin series of $f$.
Consider the steps of finding the polynomial greatest common division by the extended Euclid algorithm of $T_N$ with $x^{N+1}$ where we prematurely stop:
\begin{equation*}
\begin{aligned}
	x^{N+1} =& 1\cdot x^{N+1} +& 0\cdot T_N(x) \\
	T_N(x) =& 0\cdot x^{N+1} +& 1\cdot T_N(x) \\
	r_1(x) =& 1\cdot x^{N+1} + & -q_1(x)\cdot T_N(x) \\
	& \vdots &\\
	P(x) =& K(x)x^{N+1} + & Q(x) T_N(x) \\
\end{aligned}
\end{equation*}

By inspection we get the useful relation:
\[P(x)/Q(x) \equiv T_N(x) \mod x^{N+1}\]

Hence satisfying the original objective.
Note that we can chose decrease the degree of $P$ by simply continuing the algorithm.

\subsection{The Reverse}
Say I want to do the reverse, that I have $f = g/h$ where I have the power series for $g$ and $h$ but want to effectively find $f$
We have:

\begin{equation*}
\begin{aligned}
	x^{N+1} =& 0\cdot f(x) + &1\cdot x^{N+1}\\
	g(x) =& h(x)\cdot f(x) +& 0\cdot x^{N+1} \\
\end{aligned}
\end{equation*}

You can remove some higher term of $f$ into a residue function $K$ on $x^{N+1}$, since we don't really care about it.

\subsection{Differential}
What if $g$ and $h$ are related buy a differential equation?
We can use it like how we used the quotient-remainder equation.
The derivative is linear after all.



\subsection{Chinese Remainder Theorem}
Since I have modulo relations can I combine them with the Chinese Remainder Theorem?
The moduli will have to be pairwise coprime $(x-a_i)^n$ stand out.
