% Copyright 2024 Kieran W Harvie. All rights reserved.

\section{Conics as Determinants}
I've `known' for a long time that a conic is determined by five points.
I put known in commas because I had seen Pascal's theorem\footnote{Its old school Latin name `hexagrammum mysticum theorem' rivals `witch of Agnesi' for old math names we need to bring back.},
thought it looked legit,
and never proved or thought about it again.
But I saw the following identity that makes it so clear and almost makes me feel stupid for not seeing it before.
Given five points $(x_n,y_n)$ define the following function:
\[
f(x,y) = \begin{vmatrix}
	x^2&xy&y^2&x&y&1\\
	x_1^2&x_1y_1&y_1^2&x_1&y_1&1\\
	x_2^2&x_2y_2&y_2^2&x_2&y_2&1\\
	x_3^2&x_3y_3&y_3^2&x_3&y_3&1\\
	x_4^2&x_4y_4&y_4^2&x_4&y_4&1\\
	x_5^2&x_5y_5&y_5^2&x_5&y_5&1\\
	\end{vmatrix}
\]
Then the quadratic equation though all five given points is given by:
\[f(x,y)=0\]
The proof is straight forward as all points trivially satisfy:
\[f(x_n,y_n)=0\]
Since two rows in the determinant would be identical and hence $0$,
and the coefficients of the quadratic equation can be read of through Laplace expansion.
For example the $x^2$ coefficient is:
\[
\begin{vmatrix}
	x_1y_1&y_1^2&x_1&y_1&1\\
	x_2y_2&y_2^2&x_2&y_2&1\\
	x_3y_3&y_3^2&x_3&y_3&1\\
	x_4y_4&y_4^2&x_4&y_4&1\\
	x_5y_5&y_5^2&x_5&y_5&1\\
\end{vmatrix}
\]

\subsubsection{Computation}
Two features stuck out to me.
First is that geometry is analytic,
as opposed to synthetic,
meaning it's very easy to translate the geometry into an algorithm that acts on just the given points' coordinates. 
Second is that division isn't used which means we don't have to be concerned about zero values,
or near-zero values,
in our calculations.
\\

Sure there might be a risk of overflow by repeatably multiplying and adding values together that could be shrunk by division.
But not being forced to divide by unknown values is still preferable as near-zero values still cause overflows and we might still be able to opportunistically divide.

\subsubsection{The Constant Term}
The determinants for the coefficient of the non constant term includes a column of all $1$s and hence can be simplified further,
for example the $x^2$ could be written as:
\[\begin{vmatrix}
	x_2y_2-x_1y_1&y_2^2-y_1^2&x_2-x_1&y_2-y_1\\
	x_3y_3-x_1y_1&y_3^2-y_1^2&x_3-x_1&y_3-y_1\\
	x_4y_4-x_1y_1&y_4^2-y_1^2&x_4-x_1&y_4-y_1\\
	x_5y_5-x_1y_1&y_5^2-y_1^2&x_5-x_1&y_5-y_1\\
\end{vmatrix}\]
The constant term can be expressed without `directly' using Laplace through the following expression:
\[
f(0,0) = \begin{vmatrix}
	0&0&0&0&0&1\\
	x_1^2&x_1y_1&y_1^2&x_1&y_1&1\\
	x_2^2&x_2y_2&y_2^2&x_2&y_2&1\\
	x_3^2&x_3y_3&y_3^2&x_3&y_3&1\\
	x_4^2&x_4y_4&y_4^2&x_4&y_4&1\\
	x_5^2&x_5y_5&y_5^2&x_5&y_5&1\\
	\end{vmatrix}
= \begin{vmatrix}
	x_1^2&x_1y_1&y_1^2&x_1&y_1\\
	x_2^2&x_2y_2&y_2^2&x_2&y_2\\
	x_3^2&x_3y_3&y_3^2&x_3&y_3\\
	x_4^2&x_4y_4&y_4^2&x_4&y_4\\
	x_5^2&x_5y_5&y_5^2&x_5&y_5\\
	\end{vmatrix}
\]
Which I can't see any easy way to simplify without dividing or making another assumption about the coordinates.
So maybe a better normalization is possible but I can't see it.
And it looks like any such normalization would go against the strength of not needing to divide by arbitrary values.

\subsubsection{Level Sets}
Since we don't have to worry about division we can easily calculate $f(x,y)$ for arbitrary points and then do something with that value.
The most direct application would be to select some region of $\mathbb{R}^2$ and calculate $f(x,y)$ for each point and then colour the point based on that value,
i.e. level sets.
Such an operation maps well to shaders.
Just send $(x_n,y_n)$ to the graphics card and let it worry about invocations and visual effects based on the points.
\\

Such shading could be enhanced by considering partials to give a `flow' to the render,
animated or otherwise.

\subsection{A Tangent on Tangents}
While working on this section I was trying to calculate the tangents of:
\[Ax^2+2Bxy+Cy^2+2Dx+2Ey+F=0\]
Just directly using implicit differentiation gives:
\[Ax+By+D+(Bx+Cy+E)\frac{dy}{dx}=0\]
This equation has serious problems if $\frac{dy}{dx}$ is zero or infinity but it was good enough to give the following ansatz for the tangent at $(x_0,y_0)$:
\[(Ax_0+By_0+2)x+(Bx_0+Cy_0+E)y+Dx_0+Ey_0+F=0\]
Substitution shows that if $(x_0,y_0)$ satisfies the original quadratic then it satisfies this.
Comparison of the coefficients with the implicit derivatives shows that when they exist at $(x_0,y_0)$ they agree.
