% Copyright 2025 Kieran W Harvie. All rights reserved.

\section{Elementary Symmetric Polynomial Limit}
Consider a set of variables $\{X_k\}$.
We commonly define two families of symmetric functions on this set.
The first is the simple power sums:
\[p_n = \sum_{k}X_k^n\]
The next is the elementary symmetric polynomials.
Which are best defined as the solutions to:
\[\prod_{n=1}^{N}(\lambda -X_n) = \sum_{n=0}^{N}\lambda^{N-n}(-1)^ne_n\]
And whose general form can be understood by inspection of a few examples.
If we have set of variables $\{X_0,X_1,X_2\}$ then:
\begin{equation*}
\begin{aligned}
	e_0 =& 1\\
	e_1 =& X_1 +X_2 +X_3\\
	e_2 =& X_1X_2 +X_2X_3 +X_2X_3 \\
	e_3 =& X_1X_2X_3 \\
\end{aligned}
\end{equation*}
We can use Newton's identities to write a recursion for the elementary symmetric polynomial in terms of the power sums:
\[ne_n = \sum_{k=1}^n(-1)^{k+1}e_{n-k}p_k\]
This recursion can be used to prove an interesting limit.
If we assume the variables $\{X_k\}$ are functions of $N$ such that:
\[\frac{e_1}{N} \rightarrow 0\]
Then:
\[\frac{e_n}{N^n} \rightarrow 0\]
The proof is simple induction.
First observe that:
\[\frac{p_n}{N^n} = \sum_k\left(\frac{X_k}{N}\right)^n\rightarrow 0\,\quad\text{and}\quad\frac{e_1}{N}=\frac{1}{N}\rightarrow0\]
Then by the inductive assumption we have:
\[\frac{e_n}{N^n} = \frac{1}{n}\sum_{k=1}^n(-1)^{k+1}\frac{e_{n-k}}{N^{n-k}}\frac{p_k}{N^k}\rightarrow0\]
