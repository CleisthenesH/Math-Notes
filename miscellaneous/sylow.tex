% Copyright 2024 Kieran W Harvie. All rights reserved.

\section{Sylow}
This section follows up my group action revision with Sylow theorems revision.
I've made tweaks to definitions and proof structure to make the generalization to infinite cases smoother.
And because multiple sources have made similar non-material tweaks.

\subsection{Terminology:} 
\subsubsection{$p$-group:}
For now,
a $p$-group is a finite group whose order is a power of the prime $p$ and a $p$-subgroup of $G$ is a subgroup $G$ that is a $p$-group.
\\

I plan on adding a final section that will generalize this definition to be a group of any order so long as all elements' orders are a power of the prime $p$.
It will turn out that all such $p$-groups that are finite satisfy the original statement.

\subsubsection{Sylow $p$-subgroup:}
Similar to the first term,
we will start by defining the Sylow $p$-subgroup of a finite group $G$ as a $p$-subgroup of $G$ such that the index of the subgroup isn't divisible by $p$.
\\

Similarly we will be generalizing this to the infinite case where a Sylow $p$-subgroup of $G$ is a $p$-subgroup that isn't a proper subgroup of any $p$-subgroup of $G$.

\subsection{Lemmas:}
The lemma names given here aren't standardized,
they're just descriptions I've given them.

\subsubsection{Divisibility Lemma:}
Let $G$ act on subsets of itself by multiplication in the natural way and let the finite subset $U$ have the stabilizer $H_U$ then $|H_U|$ divides $|U|$.
\\

The main idea of this lemma is to consider the consider the natural restricted action of $H_U$ on $U$.
By definition orbits of $U$ are the right cosets of $H_U$:
\[O_u = \{h\cdot u: h\in H_U\} = H_Uu\]
Although they are the right cosets instead of the more common left cosets their property of having order of $|H_U|$ remain.
Plugging this expression into the counting theorem gives:
\[|U| = \sum_{i\in I}|H_i| = |I||H_U|\]
There lemma follows.

\subsubsection{Fixed Point Lemma:}
Let the finite $p$-group of $P$ act on the finite set $S$ and let $S_0$ be the number of elements that have an orbit of order $1$,
they are fixed by the action,
then:
\[|S_0|\equiv|S|\mod p\]

Start the proof by saying writing the order of $P$ as $p^n$ for some $n$ and use the Orbit-Stabilizer Theorem on any $s\in S$ to give:
\[|H_s||O_s|=|P|=p^n\]
For some $n$.
Use this this observation to separate terms in the Counting Theorem:
\[\begin{aligned}
	|S| &= \sum_{i\in I}|O_i|\\
	&=\sum_{\underset{|O_i|=1}{i\in I}}|O_i|+\sum_{\underset{|O_i|\neq1}{i\in I}}|O_i|\\
	&=|S_0|+\sum_{\underset{|O_i|\neq1}{i\in I}}|O_i|\\
\end{aligned}\]
Since every term in the left sum is an integer the form $p^k$ where $k\geq 1$ it is divisible by $p$,
as required.

\subsubsection{Combinatorial Lemma:}
Given a set of size $p^em$,
where $p$ doesn't divide $m$,
the number of subsets of size $p^e$ also isn't divisible by $p$.
\\

Consider the following product:
\[\binom{p^em}{p^e} = \prod_{k=0}^{p^e-1}\frac{p^em-k}{p^e-k}\]
The product can only gain a factor of $p$ when a factors' numerator has a factor of $p$,
and this can only happen when:
\[0\equiv p^em-k\equiv k\mod p \]
However, 
when this happens the denominator removes a corresponding number of factors of $p$.
To see this write relevant $k$s in the products range as $k=p^ln$,
where $p$ doesn't divide $n$ and $l<e$:
\[\frac{p^em-k}{p^e-k} = \frac{p^em-p^ln}{p^e-p^ln}=\frac{p^{e-l}m-n}{p^{e-l}-n}\]
But now the numerator has isn't divisible by $p$:
\[p^{e-l}m-n\equiv n \not\equiv 0\mod p\]
Meaning no factor can contribute a factor of $p$ to the overall product,
meaning the overall product isn't divisible by $p$.
\\

More elegant proofs using $p$-adic valuation is also possible.

\subsubsection{Subgroup Lemma}
Let $G$ be a finite group and consider a chain of subgroups $P\subseteq H\subseteq G$.
If $P$ is a Sylow $p$-subgroup of $G$ then it is a Sylow $p$-subgroup of $H$.
\\

$P$ is still a $p$-subgroup of $H$,
it has the same element,
so what's left is to show its new order $[H:P]$ isn't divisible by $p$.
Consider the Lagrange theorem:
\[[G:P] = [G:H][H:P]\]
Since if $p$ where to divide $[H_G:P]$ then it would divide $[G:P]$,
which contradicts $P$ being a Sylow $p$-subgroup of $G$.

\subsection{Theorems:}
There are three Sylow theorems and they all get results by having a finite group,
or its subgroups,
act on subsets of itself in some way.
To summarise the results:
\begin{enumerate}
	\item The first Sylow theorem shows that a Sylow $p$-subgroup exists for each prime factor.
	\item The seconds Sylow theorem shows that for any given Sylow $p$-subgroup $P$ all other $p$-subgroup have a conjugation that is a subgroup of $P$.
	\item The third Sylow theorem is three results that characterize the number of Sylow $p$-subgroups as equal to $[G:N_G(P)]$, a divider of $[G:P]$, and has a remainder of $1$ when divided by $p$.
\end{enumerate}

\subsubsection{First Theorem:}
Let the finite group $G$ have the order:
\[|G| = p^em\]
Where $p\nmid m$.
Let $S$ be the set of subsets of $G$ of order $p^e$ and let $G$ act on $S$ by multiplication in the natural way.
From the counting theorem there is $I\subseteq S$ such that:
\[|S| = \sum_{i\in I}|O_i|\]
From the combinatoric lemma we know $p$ doesn't divide $|S|$,
meaning it must not divide the sum,
meaning it must not divide at least one term $|O_i|$,
let this orbit be $O_j$.
Through the orbit-stabilizer theorem we have:
\[|O_j||H_j| = |G| = p^em\]
Now,
since $p$ doesn't divide $|O_j|$ all the $p$ factors must come from $|H_j|$,
meaning $p^e$ divides $|H_j|$.
By definition $H_j$ is the stabilizer of the element $j\in S$,
so by the divisibility lemma we know that $|H_j|$ divides $|j|=p^e$.
\\

Since $|H_j|$ divides and is divisible by $p^e$ we have $|H_j|=p^e$,
hence $H_j$ is a $p$-subgroup of $G$ and its index is:
\[[G:H_j] = \frac{|G|}{|H_j|} = \frac{p^em}{p^e} = m\]
By construction $p\nmid m$ hence $H_j$ is a Sylow $p$-subgroup.

\subsubsection{Second Theorem:}
Let $G$ be a finite group, 
$P$ be a Sylow $p$-subgroup of $G$,
and $S$ the cosets of $P$.
By definition $|S| = [G:P]$ and since $P$ is a Sylow $p$-subgroup $p\nmid |S|$.
\\

Now let $H$ be an arbitrary $p$-subgroup of $G$ and consider the action of $H$ on $S$ by multiplication in the natural way.
By the fixed point lemma we have: 
\[|S_0| \equiv |S| = [G:P]\not\equiv 0\mod p\]
Hence $|S_0|$ is non-zero,
meaning there is at least one coset $gP$ such that:
\[HgP = gP\]
This implies $g^{-1}Hg\subseteq P$ to see why expand the expression in its fully qualified glory: 
\[\begin{aligned}
	\exists g\in G,\,\forall h\in H,\,\forall p_0\in P,\, \exists p_1\in P:&& g^{-1}hgp_0=&p_1\\
	\Rightarrow\exists g\in G,\,\forall h\in H,\,\forall p_0\in P,\, \exists p_1\in P:&& g^{-1}hg=&p_1p_0^{-1}\\
	\Rightarrow\exists g\in G,\,\forall h\in H,\, \exists p_2\in P:&& g^{-1}hg=&p_2\\
\end{aligned}\]
Hence $g^{-1}Hg\subseteq P$ for all $p$-subgroups $H$.
\\

In particular if $H$ is a Sylow $p$-subgroup we would have $|H| = |P|$.
Combining with the fact the conjugation map $h\mapsto g^{-1}hg$ is injective gives:
\[g^{-1}Hg=P\]
Hence Sylow $p$-subgroups are conjugates of each other.

\subsubsection{Third Theorem:}
Let $G$ be a group, 
$S$ be the set of Sylow $p$-subgroups of $G$,
and $G$ act on $S$ by conjugation in the natural way.
This action is closed since the conjugation of a subgroup is an isomorphic subgroup,
meaning also a Sylow $p$-subgroup.
Consider some arbitrary Sylow $p$-subgroup $P\in S$,
from the second theorem the orbit of $P$ is $S$ and the stabilizer of $P$ is:
\[N_G(P) = \{g\in G\,|\,gPg^{-1} = P\}\]
This construction is already well know and is called the normalizer of $P$.
Applying the orbit-stabilizer theorem we get:
\[|S||N_G(P)|=|G|\]
The first result directly follows:
\[|S| = [G:N_G(P)]\]
For the second result we consider the Lagrange theorem:
\[[G:P] = [G:N_G(P)][N_G(P):P] = |S|[N_G(P):P]\]
Hence $|S|$ divides $[G:P]$.
\\

For the third result restrict that the action to just $P$ acting on $S$ by conjugation in the natural way.
Let $H\in S$ have an orbit size of $1$,
meaning for all $p\in P$ we have:
\[pHp^{-1}=H\]
Hence $P$ is a subgroup of $N_G(H)$.
Applying the subgroup lemma both $H$ and $P$ are Sylow $p$-subgroups of $N_G(H)$ meaning they are conjugates of each other by some elements of $N_G(H)$.
Let $n\in N_G(H)$ be an element such that:
\[nHn^{-1}=P\]
Since $n$ is in the normalizer of $H$ we have:
\[nHn^{-1}=H\]
Hence $H=P$ meaning the only element of $S$ with orbit $1$ is $P$.
Applying the fixed point theorem we get:
\[|S|\equiv |S_0| = |\{P\}| = 1 \mod p\]
The third result.
