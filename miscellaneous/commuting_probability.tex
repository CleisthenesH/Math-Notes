% Copyright 2024 Kieran W Harvie. All rights reserved.

\section{Commuting Probability}
Possibly my favourite result in elementary group theory.
Has cool lemmas,
uses many fundamental results,
and has a cool main result.

\subsubsection{Lemma: If $\Inn(G)$ is cyclic then $G$ is Abelian.}
Let $\phi_g$ represent the standard inner automorphism obtained from $g$:
\[\phi_g(x) =g^{-1}xg\]
By hypothesis there is a fixed element $g$ such that for all $h$ there exits $n$ such that:
\[\phi_h = \phi_g^n = \phi_{g^n}\]
Observe that this implies $\phi_{hg^{-n}}=\phi_1$ meaning $gh^{-n}$ is in the center of $G$,
which can be written as:
\[h=g^nz\]
The fact that two elements $h$ and $h'$ commute follow directly from powers of $g$ commuting with the center and each other:
\[\begin{aligned}
	hh'&=g^nzg^{n'}z'\\
	&=g^{n'}g^{n-n'}zg^{n'}z'\\
	&=g^{n'}zg^{n-n'}g^{n'}z'\\
	&=g^{n'}zg^{n}z'\\
	&=g^{n'}z'g^{n}z\\
	&=h'h\\
\end{aligned}\]

\subsubsection{Lemma: If $\Inn(G)$ is cyclic then $\Inn(G)$ is the trivial group.}
From the previous lemma if $\Inn(G)$ is cyclic then $G$ is Abelian but if $G$ is Abelian then $\Inn(G)$ is the trivial group.

\subsubsection{Lemma: The smallest non trivial inner automorphism group is the Klein-four Group.}
The Klein-four Group is the smallest group that isn't trivial or cyclic and by the previous lemma is the smallest possible non trivial inner automorphism group.
The dihedral group of order $8$ can also be shown to have the inner automorphism group of the Klein-four group by direct calculation.

\subsubsection{Main Result:}
The chance that two elements $g,h\in G$ commute is given by:
\[\frac{1}{|G|^2}|\{(g,h)\in G^2 | gh=hg\}|\]
This sum can be taken to a more standard form as:
\[\begin{aligned}
	|\{(g,h)\in G^2 | gh=hg\}| &= \sum_{g\in G}|{h\in G| gh=gh}|\\
	&= \sum_{g\in G}|C_G(g)|\\
\end{aligned}\]
Plugging and grinding definitions gives:
\[\begin{aligned}
	&\frac{1}{|G|^2}|\{(g,h)\in G^2 | gh=hg\}|\\
	=& \frac{1}{|G|^2}\sum_{g\in G}|C_G(g)| \\
	=& \frac{1}{|G|}\sum_{g\in G}\frac{1}{[G:C_G(g)]} \\
	=& \frac{1}{|G|}\left(\sum_{g\in Z(G)}\frac{1}{[G:C_G(g)]}+\sum_{g\in G/Z(G)}\frac{1}{[G:C_G(g)]}\right) \\
	\leq& \frac{1}{|G|}\left(\sum_{g\in Z(G)}\frac{1}{1}+\sum_{g\in G/Z(G)}\frac{1}{2}\right) \\
	=& \frac{1}{|G|}\left(|Z(G)|+\frac{1}{2}(|G|-|Z(G)|)\right) \\
	=& \frac{1}{2}\left(1+\frac{|Z(G)|}{|G|}\right) \\
	=& \frac{1}{2}\left(1+\frac{1}{|\Inn(G)|}\right) \\
\end{aligned}\]
If $\Inn(G)$ is trivial we have $|\Inn(G)|=1$ and the commuting probability is less than or equal to $1$,
as expected.
Otherwise $\Inn(G)$ isn't trivial we have $|\Inn(G)|\geq 4$ meaning the commuting probability is less than or equal to $\frac{5}{8}$,
quite a big jump!
