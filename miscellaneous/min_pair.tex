% Copyright 2025 Kieran W Harvie. All rights reserved.

\section{Minimal Chance of a Matching Pair}
As I was going to bed I had a thought:
"A uniform distribution minimizes the chance of getting a matching pair as any sample less likely to form a pair is less likely to be sampled at all".  
And it turns out this this intuition can easily be proven in the discrete and finite case through Cauchy-Schwartz inequality:
\[\sum_{k=1}^np_k^2\geq\frac{\left(\sum_{k=1}^np_k\cdot1\right)^2}{\sum_{k=1}^n1^2}=\frac{1}{n}\]
As there is equality if and only if $(p_k)_n$ is a linearly dependent to $(1)_n$,
meaning a uniform distribution.
This can be generalized to positive weights $w_k$ through Sedrakyan's lemma as:
\[\sum_{k=1}^np_k^2w_k \geq \frac{\left(\sum_{k=1}^np_k\right)^2}{\sum_{k=1}^nw_k^{-1}}=\frac{1}{n}\frac{n}{\sum_{k=1}^nw_k^{-1}}\]
Where the final right factor is the harmonic mean of the weights.
\\

Generalizing this result to the infinite case doesn't interest me as there are no uniform distributions and you can take the pair chance arbitrarily low.
\\

The continuous case is also bit awkward,
as the set of matching pairs most likely has a measure of zero,
but it's still interesting.
So let $f:X\rightarrow\mathbb{R}_{\geq0}$ be a probability density function, 
$S\subseteq X$,
and $\chi_S$ the indicator function of $S$.
From Cauchy-Schwartz we have:
\[\langle f,f\rangle \geq \frac{|\langle\chi_S,f\rangle|^2}{\langle\chi_S,\chi_S\rangle}\]
$\langle f,f\rangle$ is the autocorrelation\footnote{
Since $f$ is real the autocorrelation is just the convolution of $f(\tau)$ and $f(-\tau)$.}
at $0$ and hence a measure of the chance of getting a matching pair,
$\langle \chi_S,f\rangle$ is the probability of the sample being in $S$ and $\langle \chi_S,\chi_S\rangle$ is the size of $S$,
\\

Assume the set $S$ that is more favoured then it would be under the uniform distribution:
\[\langle f,\chi_S\rangle > \frac{|S|}{|X|}\]
Then:
\[\langle f,f\rangle > \left(\frac{|S|}{|X|}\right)^2\frac{1}{|S|} \geq \frac{1}{|X|}\]
As required.
\\

If I wanted to do the continuous case better\footnote{
Such as considering samples that are within $\epsilon$ of $0$ to better quantify the chance of a pair} 
I'd consider characteristic functions instead of density functions because if $\phi_X(t)$ is the characteristic function of $X$ then the characteristic function of the difference of two samples would be:
\[|\phi_X(t)|^2\]
Which looks promising.

\subsection{Doing the Continuous Cases Better}
Looking into this problem has been a fun return to $L^2$ as the main result is combining these three classic results:
\[\begin{aligned}
	\widehat{f*g} &= \hat{f}\hat{g} &&\text{Convolution}\\
	\widehat{f\star g} &= \hat{f}\overline{\hat{g}}&&\text{Cross-Correlation}\\
	\langle f,g\rangle &= \frac{1}{2\pi}\langle\hat{f},\hat{g}\rangle&&\text{Parseval}\\
\end{aligned}\]	
Into the aesthetically pleasing:
\[\begin{aligned}
	\langle f\star f,g\star g\rangle &=\frac{1}{2\pi}\langle \hat{f}\overline{\hat{f}},\hat{g}\overline{\hat{g}}\rangle\\
	&=\frac{1}{2\pi}\langle \hat{f}\hat{g},\hat{f}\hat{g}\rangle\\
	&=\langle f*g,f*g\rangle\\
\end{aligned}\]	
When $f$ is real $f\star f$ is the autocorrelation from earlier and when $g$ is conjugate symmetric $g\star g$ is equal to $g*g$as:
\[ \widehat{g*g} = \hat{g}\hat{g} = \hat{g}\overline{\hat{g}} = \widehat{g\star g} \]
So by considering $\langle f*g,f*g\rangle$ where we previously considered $\langle f,f\rangle$ we better quantify what a matching pair means for a continuous function as the autocorrelation $f\star f$ is weighted by $g\star g$.
This is best shown through some examples:

\subsubsection{Example 1:}
Let $g$ be the `standard' rectangular function:
\[g(x) = \chi_{\left[-\frac{1}{2},\frac{1}{2}\right]}(x)\]
This makes $g\star g$ the `standard' triangular function:
\[(g\star g)(x) = (g*g)(x) = (1-|x|)\chi_{\left[-1,1\right]}(x)\]
We can interpret $f*g$ as the sliding average of $f$ over a unit interval and $\langle f\star f,g\star g\rangle$ as the expected value of the difference of $f$ weighted and gated by the triangular function.
\\

Similar to before,
let $X$ be the domain of $f*g$ where the function is non-zero and suppose that $X$ has a finite non-zero measure.
Let there be a number $\alpha>0$ such that for all sets $S$ we have:
\[\langle \chi_S,f* g\rangle \geq \alpha\frac{|S|}{|X|}\]
How large we can make $\alpha$ is a measure of how `non-uniform' $f*g$ is.
When squared $\alpha$ gives a lower bound on the weighted chance of a matching pair:
\[\begin{aligned}
	\langle f\star f,g\star g\rangle &= \langle f*g,f*g\rangle\\
	&\geq \frac{|\langle \chi_S,f* g\rangle|^2}{\langle \chi_S,\chi_S\rangle}\\
	&\geq \alpha^2\\
\end{aligned}\]	

\subsubsection{Example 2:}
We can recover our previous analysis by letting $g=\delta$ as:
\[\langle f\star f,g\star g\rangle = \langle f\star f,\delta\rangle = (f\star f)(0)\]
And:
\[\langle f*g,f*g\rangle = \langle f*\delta,f*\delta\rangle = \langle f,f \rangle\]
Reduces our main result to:
\[(f\star f)(0) = \langle f\star f,g\star g\rangle = \langle f*g,f*g\rangle = \langle f,f\rangle\]
This is expected,
since "weighting by $\delta$" was what we were doing when using $\langle f,f\rangle$ as a measure of the chance of a matching pair.

\subsubsection{Other Results:}
While retuning to $L^2$ I was reminded of two other fun results which are easily proved with the tools we have here.
The first result is that:
\[\hat{f}(0) = \int_{\mathbb{R}}f(t)\,dt\]
Which follows from:
\[\langle f,1\rangle = \frac{1}{2\pi}\langle\hat{f},2\pi\delta\rangle = \hat{f}(0) \]
And the second is that:
\[ \int_{\mathbb{R}}(f*g)(t)\,dt = \int_{\mathbb{R}}f(t)\,dt\int_{\mathbb{R}}g(t)\,dt\]
Which follows from our previous result:
\[\begin{aligned}
	\int_{\mathbb{R}}(f*g)(t)\,dt &= \widehat{f*g}(0) \\
	&= \hat{f}(0)\hat{g}(0)\\
	&= \int_{\mathbb{R}}f(t)\,dt\int_{\mathbb{R}}g(t)\,dt\\
\end{aligned}\]
The second result is useful for proving that when $f$ and $g$ are probability density functions then so is $f*g$ as it shows that $f*g$ is normalized too.
