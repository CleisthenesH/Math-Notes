% Copyright 2025 Kieran W Harvie. All rights reserved.

\section{Minimal Chance of a Matching Pair}
As I was going to bed I had a thought:
"A uniform distribution minimizes the chance of getting a matching pair as any sample less likely to form a pair is less likely to be sampled at all".  
And it turns out this this intuition can easily be proven in the discrete and finite case through Cauchy-Schwartz inequality:
\[\sum_{k=1}^np_k^2\geq\frac{\left(\sum_{k=1}^np_k\cdot1\right)^2}{\sum_{k=1}^n1^2}=\frac{1}{n}\]
As there is equality if and only if $(p_k)_n$ is a linearly dependent to $(1)_n$,
meaning a uniform distribution.
This can be generalized to positive weights $w_k$ through Sedrakyan's lemma as:
\[\sum_{k=1}^np_k^2w_k \geq \frac{\left(\sum_{k=1}^np_k\right)^2}{\sum_{k=1}^nw_k^{-1}}=\frac{1}{n}\frac{n}{\sum_{k=1}^nw_k^{-1}}\]
Where the final right factor is the harmonic mean of the weights.
\\

Generalizing this result to the infinite case doesn't interest me as there are no uniform distributions and you can take the pair chance arbitrarily low.
\\

The continuous case is also bit awkward,
as the set of matching pairs most likely has a measure of zero,
but it's still interesting.
So let $f:X\rightarrow\mathbb{R}_{\geq0}$ be a probability density function, 
$S\subseteq X$,
and $\chi_S$ the indicator function of $S$.
From Cauchy-Schwartz we have:
\[\langle f,f\rangle \geq \frac{|\langle\chi_S,f\rangle|^2}{\langle\chi_S,\chi_S\rangle}\]
$\langle f,f\rangle$ is the autocorrelation\footnote{
Since $f$ is real the autocorrelation is just the convolution of $f(\tau)$ and $f(-\tau)$.}
at $0$ and hence a measure of the chance of getting a matching pair,
$\langle \chi_S,f\rangle$ is the probability of the sample being in $S$ and $\langle \chi_S,\chi_S\rangle$ is the size of $S$,
\\

Assume the set $S$ that is more favoured then it would be under the uniform distribution:
\[\langle f,\chi_S\rangle > \frac{|S|}{|X|}\]
Then:
\[\langle f,f\rangle > \left(\frac{|S|}{|X|}\right)^2\frac{1}{|S|} \geq \frac{1}{|X|}\]
As required.

If I wanted to do the continuous case better\footnote{
Such as considering samples that are within $\epsilon$ of $0$ to better quantify the chance of a pair} 
I'd consider characteristic functions instead of density functions because if $\phi_X(t)$ is the characteristic function of $X$ then the characteristic function of the difference of two samples would be:
\[|\phi_X(t)|^2\]
Which looks promising.
