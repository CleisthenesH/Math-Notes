% Copyright 2023 Kieran W Harvie. All rights reserved.

\section{Interest Identities}
Let $P$ be the principle invested at a rate of $r$.
Consider four different investment scenarios:
\begin{itemize}
	\item Not invested: $P_0 = P$.
	\item Fully Invested at the beginning, one instalment at the end:
		\[P_1 = (1+r)P\]
	\item Continuously invested, continuous installments:
		\[P_2 = \lim_{n\rightarrow\infty}\sum_{k=0}^n\frac{P}{n}\left(1+\frac{r}{n}\right)^k = \frac{\exp(r)-1}{r}P\]
	\item Fully Invested at the beginning, continuous instalments: 
		\[P_3 = \lim_{n\rightarrow \infty}P\left(1+\frac{r}{n}\right)^n = \exp(x)P\]
\end{itemize}

Interestingly the relative size of $P_2$ and $P_1$ depend on $r$.
$P_1$ starts on $P_2$ but switches as $r$ increases.\\

$P_3$ is always the best, the proof for $P_1$ and $P_0$ are obvious.
$P_3 > P_2$ follows from:
\[0 < \int_0^rt\exp(t)\,dt = \big[(t-1)\exp(t)\big]_0^r = (r-1)\exp(r)+1\] \\

The following interesting identities hold:
\[P_3 = rP_2+P_0\]
\[P_3-P = r(P_2-P) + (P_1-P)\]
The breaks first neatly breaks $P_3$ into a nice linear sum.
The second does similar for the profit of the investment, total yield minus principle.
