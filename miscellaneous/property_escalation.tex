% Copyright 2025 Kieran W Harvie. All rights reserved.

\section{Property Escalation}
I found a simple olympiad problem that really demonstrated `property escalation'.
And by `property escalation' I mean how a simple property can prove another property,
which then proves a more complex property,
and eventually the solution.

\subsubsection{Problem statement:}
Find all functions that satisfy:
\[f(f(n)+m) = f(m)+n\]
For all integers $n$ and $m$.

\subsubsection{$f$ takes arbitrary values:}
Fix $m$ and set $n=k-f(m)$:
\[f(f(k-f(m))+m) = f(m)+k-f(m) = k\]

\subsubsection{$0$ is a fixed point of $f$:}
Set $n=0$:
\[f(f(0)+m)=f(m)\]
If $f(0)\neq0$ then $f$ is periodic with a period of $f(0)$,
this contradicts $f$ taking arbitrary values.

\subsubsection{$f$ is idempotent:}
Set $m=0$:
\[f(f(n)+0) = f(0) = n\]

\subsubsection{$f$ is additive:}
Set $n=f(k)$:
\[f(f(f(k))+m) = f(k)+f(m)\]

\subsubsection{$f$ is the identity function:}
Since we know $f$ is additive we can expand the original equation:
\[f(n)+m=f(m)+n\]
Setting $m=0$ gives:
\[f(n)=n\]
