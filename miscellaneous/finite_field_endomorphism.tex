% Copyright 2023 Kieran W Harvie. All rights reserved.

\section{Finite Field Endomorphism}
Let $F_n$ be and finite field and let $f$ be an endomorphism,
then $f$ can be written as a polynomial of degree $n-1$ or less.
\\

Construct an indicator function in the normal way:
\[\chi_y(x) = \prod_{z\in F_n/\{y\}}\frac{x-z}{y-z}\]
Observe that this is a $n-1$ degree polynomial.
Now use the indicator function in the normal way:
\[f(x) = \sum_{y\in F_n}\chi_y(x)f(y)\]

\subsection{Low Hanging Fruit}
\[f(x) = yx\]
For non-zero $y$,
this is a permutation of $F_n$ that only fixes $0$.
\[f(x) = x+y\]
Is a permutation that doesn't fix anything.
Combining the two gives:
\[f(x) = y(x-x_0)+x_0\]
A permutation that only fixes $x_0$.

\subsection{Elementary Symmetric Function}
Let $e_n(S)$ be the $n$th elementary symmetric function of the set $S$ with element $y$.
\[e_0(S) = 0\]
\[e_n(S/\{y\})=e_n(S)-ye_{n-1}(S/\{y\})\]
\[e_n(S/\{y\})=\sum_{k=0}^ne_k(S)(-y)^{n-k}\]
Now recall:
\[x^{n}-x = \prod_{y\in F_n}(x-y)\]
Which gives the classic:
\[e_m(F_n) = \begin{cases} 1 & m = 0\text{ or } m = n-1\\ 0 & \text{ else}\\\end{cases}\]
\\

The first simplification we can make is to notice that:
\begin{equation*}
\begin{aligned}
\prod_{z\in F_n/\{y\}}\frac{1}{y-z} =& \frac{1}{e_{n-1}(F_n/\{0\})}\\
=& \frac{1}{e_{n-1}(F_n)-0e_{n-2}(F_n/\{0\})}\\
=&1\\
\end{aligned}
\end{equation*}
Hence:
\[\chi_y(x) = \prod_{z\in F_n/\{y\}}(x-z)\]
Going further:
\begin{equation*}
\begin{aligned}
e_m(F_n/\{y\})=&\sum_{k=0}^me_k(F_n)(-y)^{m-k}\\
=&(-y)^m+[m\geq n-1](-y)^{m-(n-1)}\\
\end{aligned}
\end{equation*}

\begin{equation*}
\begin{aligned}
\chi_y(x) =& \prod_{z\in F_n/\{y\}}(x-z)\\
=&\sum_{k=0}^{n-1}x^k(-1)^{n-k-1}e_{n-k-1}(F_n/\{y\})\\
=&\sum_{k=0}^{n-1}x^k(-1)^{n-k-1}(-y)^{n-k-1}+(-1)^{n-1}\\
=&\sum_{k=0}^{n-1}x^ky^{n-k-1}+(-1)^{n-1}\\
\end{aligned}
\end{equation*}
Which I guess makes a lot of sense when looked at like a geometric progression with
Frobenius to the numerator.
But that this more formal argument is cool.

\subsection{Second Attempt}
The last attempt with elementary polynomials had errors and was generally sloppy,
but it's enough to work backwards from:
\[\chi_y(x) = \frac{1}{n-1}\sum_{k=0}^{n-2}x^ky^{n-1-k}+y^{n-1}\frac{x^{n-1}-1}{n-1}+(1-x^{n-1})(1-y^{n-1})\]
When $y=0$ we have:
\[\chi_y(x) = 1-x^{n-1}\]
When $y\neq 0$ we have:
\[\chi_y(x) = \frac{1}{n-1}\sum_{k=0}^{n-2}(xy^{-1})^k+\frac{x^{n-1}-1}{n-1}\]
Gives Fourier vibes, might try again with generators.
