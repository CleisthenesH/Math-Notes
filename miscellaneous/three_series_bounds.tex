% Copyright 2023 Kieran W Harvie. All rights reserved.

\section{Bounds of Three Series}
Even though I'm sick I found an interesting problem on the internet.
Given positive $a_1,b_1,c_1>0$ and:
\begin{equation*}
\begin{aligned}
	a_{n+1} =& b_n+\frac{1}{c_n}\\
	b_{n+1} =& c_n+\frac{1}{a_n}\\
	c_{n+1} =& a_n+\frac{1}{b_n}\\
\end{aligned}
\end{equation*}

Show that at least one of $a_{800},b_{800},c_{800}$ is greater than 40 and that all sequences are unbounded.

\subsection{Lemma: Closed form for $x_{n+1} = x_n+\frac{k}{x_n}$} 
The closed form is this:
\[ x_{n+1}^2 = x_1^2+2nk+k^2\sum_{l=1}^n\frac{1}{x_l^2}\]
This result is nothing special, 
just a lot of algebra,
the plan is to first get a result for the product of successive terms,
then get perform induction on the product,
and finally cleanup to get the final form.

\begin{equation*}
\begin{aligned}
	x_{n+2}x_{n+1} =& \left(x_{n+1}+\frac{k}{x_{n+1}}\right)x_{n+1}\\
	=& \left(x_n+\frac{k}{x_n}+\frac{k}{x_{n+1}}\right)x_{n+1}\\
	=& x_{n+1}x_n + k +k\frac{x_{n+1}}{x_n} \\
	=& x_{n+1}x_n + k +k\frac{1}{x_n}\left(x_n+\frac{k}{x_n}\right) \\
	=& x_{n+1}x_n + 2k +\frac{k^2}{x_n^2} \\
\end{aligned}
\end{equation*}

Induction gives:
\[ x_{n+2}x_{n+1} = x_2x_1+2nk+k^2\sum_{l=1}^n\frac{1}{x_l^2}\]

Then you clean up by using the initial relation on $x_{n+2}$ and $x_2$.

\subsection{Unbounded:}
Let $s_n = a_n+b_n+c_n$, 
from the HM-AM inequality we have:
\[\frac{1}{a_n}+\frac{1}{b_n}+\frac{1}{c_n} \geq \frac{9}{s_n}\]
With equality if and only if $a_n=b_n=c_n$.
\\

From the series definition we have:
\[s_{n+1} = s_n + \frac{1}{a_n}+\frac{1}{b_n}+\frac{1}{c_n} \geq s_n+\frac{9}{s_n}\]

Since $a_n=b_n=c_n \Rightarrow a_{n+1}=b_{n+1}=c_{n+1}$ we have $s_n$ is bounded by:
\[ s_{n+1}^2 \geq s_1^2+18n+81\sum_{l=1}^n\frac{1}{s_l^2}\]
Hence $s_n$ grows at least as much as $3\sqrt{2n}$,
which is unbounded.
\\

If the sum is unbounded at least one of the series is unbounded.
And if at least one series is unbounded the "swapping" nature of the series means that they all are.

\subsection{$800^\text{th}$ terms:}
Consider the bound sum:
\[ s_{n+1}^2 \geq s_1^2+18n+81\sum_{l=1}^n\frac{1}{s_l^2}\]
If $s_n \leq 1$ then $s_{n+1} > 1$ hence at least half of the time $\frac{1}{s_l^2}$ must be greater than one.
Giving the new bound:
\[ s_{n+1}^2 > 18n+81\left\lfloor\frac{n}{2}\right\rfloor\]
(This isn't a particularly good bound, we could consider $s_1$ separately, but if works the result we need)
Substituting in $n=799$ gives:
\[s_{800}^2 > 18\times799+81\times398 > 18\times800 = 3^2\times40^2\]
Hence:
\[a_{800}+b_{800}+c_{800} = s_{800} > 3\times40\]
From pigeon holing at least one LHS term is greater than 40.

\subsection{Remarks}
This is yet another example of this Olympian style problem where I had to be prompted to consider the HM-AM inequality.

Although I was working with inequalities between elementary polynomial sums so wasn't that far away myself,
and the other advice that was given was just straight up wrong, 
so I should probably just remember to use the inequality means in the future and let it go.
