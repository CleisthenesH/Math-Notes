% Copyright 2023 Kieran W Harvie. All rights reserved.

\section{Discontinuities in a Non-decreasing Function}

Let $f$ be a non-decreasing function.

Define the jump function $J$ as:
\[ J(x) = \inf\{f(t) | t > x\} - \sup\{f(t) | t < x\}\]

This function is well defined since the sets are appropriately bound by $f(x)$. And it is clear that $J(d) \neq 0$ iff $d$  is a discontinuity and that $J$ is non-negative.
\\

Let $U = (x_0,x_1)$. For $d_n \in U$ with $n < m \Rightarrow d_n < d_m$ we have:
\[f(x_1)-f(x_0) \geq \sum_{k}J(d_k)\]

Proof:
\begin{equation*}
\begin{aligned}
	&\sum_{k}J(d_k) \\
	=& \inf\{f(t) | t > d_n\} - \sup\{f(t) | t < d_0\} + \sum_k\big[\inf\{f(t) | t > d_{k-1}\} - \sup\{f(t) | t < d_k\}\big] \\
	\leq & f(d_n)-f(d_0) + \sum_k\big[f(d_{k-1})-f(d_k)\big] \\
	\leq & f(d_n)-f(d_0)\\
	\leq & f(x_n)-f(x_0)\\
\end{aligned}
\end{equation*}

Let $ S_n = \{d \in U | J(d) > \frac{1}{n}(f(x_1)-f(x_0)) \}$
From the previous inequality there are at most $n$ elements in $S_n$.
Hence:
\[\{d\in U | J(d) > 0\} = \bigcup_{n}\left\{d \in U | J(d) > \frac{1}{n}(f(x_1)-f(x_0))\right\}\]
Is countable, hence the number of discontinuities of $f$ on $U$ is countable.
\\

By corollary the discontinuities of a non-decreasing function on $\mathbb{R}$ are countable:

Let $f$ be a non-decreasing function on $\mathbb{R}$.
Let $X_n = (n-1,n+1)$, clearly $\mathbb{R} = \bigcup_{n}X_n$\footnote{Having them overlap simplify the proof by avoiding literal edge-cases.}.
Assume $f$ has an uncountable number of discontinuities then at least one $X_n$ contains uncountable discontinuities.
Otherwise there would be a countable set of countable sets of discontinuities, making them countable.
But $f$ being non-decreasing function and having an uncountable number of discontinuities in $X_n$ is a contradiction.
