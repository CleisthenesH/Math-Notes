% Copyright 2024 Kieran W Harvie. All rights reserved.

\section{Group Action}
Let $G$ be a group and $S$ an arbitrary set,
the function $\cdot:G\times S\rightarrow S$ is called a "group action" if for all $s\in S$, $g,h\in G$, and $1$ being the unit of $G$, we have:
\[\begin{aligned}
	1\cdot s =& s\\
	gh\cdot s =& g\cdot h\cdot s\\
\end{aligned}\]
Given an element $s$ there are two natural sets to consider,
first is the subset of $G$ that leaves $s$ unchanged:
\[H_s=\{g\in G:g\cdot s = s\}\]
This subset is called the stabilizer of $s$,
second is a subset of $S$ consisting all elements accessible from $s$:
\[O_s =\{g\cdot s:g\in G\}\]
This subset is called the orbit of $s$.

\subsection{Structure}
Despite $S$ only being a set,
both of these subsets have a lot of structure.
Proofs of this structure is largely just expanding definitions while using our ability to invert arbitrary elements,
but I guess most introductory group theory proofs are like this.

\subsubsection{The Set of Sets $\{O_s:s\in S\}$ Partitions $S$:}
It's trivial that every element of $S$ is in at least of set as $s\in O_s$.
If $O_s$ and $O_{s'}$ share an element we have $g,g'\in G$ such that:
\[g\cdot s = g'\cdot s'\]
We can use this to bridge between the two orbits for any element $h$:
\[h\cdot s = hg^{-1}g\cdot s = hg^{-1}g'\cdot s'\]
Meaning if $O_s$ and $O_{s'}$ share a single element they share all elements.


\subsubsection{$H_s$ is a Subgroup of $G$:}
Similarly we just need to prove that if $h\in H_s$ then so is $h^{-1}$:
\[H_s\Rightarrow  h^{-1}s = h^{-1}\cdot (h\cdot s) = (h^{-1}h)\cdot s=s\]
And that if $h,h'\in H_s$ so is $hh'$:
\[(hh')\cdot s = h\cdot h'\cdot s = h\cdot s = s\]

\subsection{Theorems}
These two bits of structure give us two important theorems.

\subsubsection{The Orbit-Stabilizer Theorem:}
Consider the following chain of equivalences:
\[\begin{aligned}
	g\cdot s = h\cdot s \Leftrightarrow& s=g^{-1}h\cdot s\\
	\Leftrightarrow& g^{-1}h\in H_s\\
	\Leftrightarrow& gH_s = g(g^{-1}h)H_s = hH_s\\
\end{aligned}\]
This chain allows us to define a bijection between $O_s$ and  cosets of $H_s$ mapping the coset $X$ to $x\cdot s$ where $x\in X$:
\[X\mapsto x\cdot s\]
Reading the chain backwards shows this map is well-defined,
$x\cdot s$ doesn't depend on choice of $x$,
reading the chain forward shows the function in injective.
The mapping is surjective since any $g$ is in the coset $gH_s$:
\[gH_s\mapsto g\cdot s\]
This bijection means the index $|G:H_s|$,
the number of cosets of $H_s$ in $G$,
is equal to the order of $O_s$.
When $G$ is finite the Lagrange's theorem gives:
\[|G| = |H_s||O_s|\]

\subsubsection{The Counting Theorem:}
Let $I\subseteq S$ index the set of set $\{O_s:s\in S\}$ such that for all $i,j\in I$:
\[ O_i \neq O_j\]
Then partitioning means:
\[\begin{aligned}
	\sum_{s\in S}f(|O_s|) =& \sum_{i\in I}\sum_{s\in O_i}f(|O_s|)\\
	=&\sum_{i\in I}|O_i|f(|O_i|)\\
\end{aligned}\]
In particular when $S$ is finite and $f(x)=1$ we have:
\[|S| = \sum_{i\in I}|O_i|\]
This result is called the counting theorem.
\\

Note that when $G$ is finite we can turn a function $g(|O_s|$,$|H_s|)$ into just $f(|O_s|)$ by use of the Orbit-Stabilizer Theorem:
\[f(x) = g(x,|G|x^{-1})\]

\subsection{Finite Subgroups of $SO(3)$}
The three dimensional special orthogonal group,
denoted $SO(3)$,
is the systematic name of the group of rotation in three dimensional rotations where the group operations is composition,
more extensive descriptions are readily available elsewhere.
\\

We wish to classify the finite subgroups of $SO(3)$ and will do so by considering it's action on a sphere $S^2$ where the action will be the natural interpretation of rotations on the sphere.
Instead of the whole of $S^2$ we will considering the set of `poles' $P$ of $G$ which are the elements of $S^2$ with a stabilizer order greater than or equal to $2$.
\\

We'll start with a pure group theoretic result proving that $G$ acts on $P$,
then apply application specific knowledge to get an expression of $|G|$ in terms of stabilizers,
and finally applying group theory to simplify and apply this expression to the classification of finite subgroups.

\subsubsection{$G$ Acts on $P$:}
To prove $G$ acts on $P$ we only need to prove closure, 
the rest is inherited from from $G$ as a subgroup of $SO(3)$ acting on $S^2$.
A pole's, $\rho$, stabilizer's order being greater than or equal to $2$ is equivalent to saying there exits a no-unit element $g\in H_s$.
Let $g$ be the non-unit element such that:
\[q\cdot \rho = \rho\]
Now let $h$ be an arbitrary element in $G$ and consider the following chain of equalities:
\[(hgh^{-1})\cdot (h\cdot \rho) = h\cdot g\cdot (h^{-1}h)\cdot \rho = h\cdot g\cdot \rho = h\cdot \rho \]
By contradiction,
when $g$ in non-unit so is $hgh^{-1}$ as:
\[hgh^{-1}=1\Rightarrow hg=h \Rightarrow g=1\]
So the chain shows that for arbitrary $h$ the element $h\cdot\rho$'s stabilizer contains the non-unit element $hgh^{-1}$.
Hence we have closure and hence $G$ acts on $P$.

\subsubsection{Double Counting Rotations:}
Let the identity rotation be the rotation around any pole by a multiple of $2\pi$,
essentially no rotation,
and let a proper rotation be any rotation that isn't the identity rotation.
Every proper rotation fixes {\em exactly} two points of the sphere,
the points where the pole of rotation intersects the sphere.
The stabilizer of a pole $\rho$ includes the identity rotation,
hence $|H_\rho|-1$ is the number of proper rotations that fix $\rho$ and the sum:
\[\sum_{\rho\in P}(|H_\rho|-1)\]
Double counts the number of proper rotations in $G$.
Since $G$ includes the identity rotation,
the full expression becomes:
\[|G| = 1+\frac{1}{2}\sum_{\rho\in P}(|H_\rho|-1)\]
If you need help visualizing the double counting you can pair up a pole with it's antipodal point.
\\

By applying the generalization in Counting Theorem then the Orbit-Stabilizer Theorem to this sum we get:
\[\begin{aligned}
|G| =& 1+\frac{1}{2}\sum_{i\in I}|O_i|(|H_i|-1)\\
=& 1+\frac{|G|}{2}\sum_{i\in I}(1-|H_i|^{-1})\\
\end{aligned}\]

\subsubsection{Bound on $|I|$:}
Dividing through the previous sum by $|G|$ we get:
\[1 = |G|^{-1}+\frac{1}{2}\sum_{i\in I}(1-|H_i|^{-1})\]
Both the terms are greater than zero and sum to $1$ hence each term must be less than $1$:
\[1>\frac{1}{2}\sum_{i\in I}(1-|H_i|^{-1})\]
By definition,
each $|H_i|\geq 2$ meaning: 
\[1>\frac{1}{2}\sum_{i\in I}(1-|H_i|^{-1}) \geq \frac{1}{2}\sum_{i\in I}\frac{1}{2} = \frac{|I|}{4}\]
Hence we get the following bound on $|I|$:
\[|I|\leq 3\] 

\subsubsection{Potential Subgroups}
When $|I|=1$ then the whole of $G$ is in the same orbit hence $|O_1|=|G|$ and by the Orbit-Stabilizer Theorem $|H_1|=1$ meaning there are no proper rotations and $G$ is the trivial subgroup.

When $|I|=2$ we get:
\[\begin{aligned}
1 =& \frac{1}{|G|} + \frac{1}{2}\bigg(\left(1-\frac{1}{|H_1|}\right)+\left(1-\frac{1}{|H_2|}\right)\bigg)\\
=& \frac{1}{|G|} + \frac{1}{2}\bigg(2-\frac{1}{|H_1|}-\frac{1}{|H_2|}\bigg)\\
0=& \frac{1}{|G|} - \frac{1}{2}\bigg(\frac{1}{|H_1|}+\frac{1}{|H_2|}\bigg)\\
0=& 1- \frac{1}{2}\bigg(|O_1|+|O_2|\bigg)\\
2=& |O_1|+|O_2|\\
\end{aligned}\]
Since $|O_i|$ are integers the only solution is $|O_i|=1$ and by the Orbit-Stabilizer Theorem $|G| = |H_1| = |H_2|$,
hence:
\[|H_1| = |H_2| =|G|=n\]
Are potential subgroups.
\\

Similar algebra applied to $|I|=3$ gives
\[\frac{1}{|H_1|}+\frac{1}{|H_2|}+\frac{1}{|H_3|}=1+\frac{2}{|G|}  > 1\]
Meaning there's only so large $|H_i|$ can ge before being larger than $1$ can't be meet,
hence we can work though potential subgroups out through cases.
For example at least one of $|H_i|$ must be $2$ otherwise:
\[\frac{1}{3}+\frac{1}{3}+\frac{1}{3} =1\]

When $|H_1|=|H_2| = 2$ we have:
\[\frac{1}{|H_3|} = \frac{2}{|G|}\]
Hence:
\[|H_1| = |H_2| = 1,\,|H_3| = n,\, |G|=2n\]
Are potential subgroups.
\\

When $|H_1|=2,\,|H_2|=3$ we have:
\[\frac{1}{2}+\frac{1}{3}+\frac{1}{6} =1\]
Hence only:
\[\begin{matrix}
	|H_1| = 2,\,&|H_2|=3,\,&|H_3|=3,\,&|G|=12\\
	|H_1| = 2,\,&|H_2|=3,\,&|H_3|=4,\,&|G|=24\\
	|H_1| = 2,\,&|H_2|=3,\,&|H_3|=5,\,&|G|=60\\
\end{matrix}\]
Are possible subgroups.
\\

When $|H_1|=2,\,|H_2|=4$ we have:
\[\frac{1}{2}+\frac{1}{4}+\frac{1}{4} =1\]
Hence there are no more possible subgroups.

\subsubsection{Actual Classification:}
So far these are only potential subgroups,
to finish classification we still need to find a finite subgroup of $SO(3)$ matching these conditions and show that they are the only ones to do so.
This second part is out of the scope,
but rest assure that the theory around classification of finite groups of such a low order is mature and can easily be searched.
\\

For the $|I|=1$ case we got the trivial subgroup which corresponds to just the identity rotation.
For the $|I|=2$ case we got the family of subgroups corresponding to rotation around a singular pole of $n$ points spread out along the equator of the pole.
For the $|I|=3$ family of subgroups we got the same points as before but an extra rotation of flipping the two poles.
For the other $|I|=3$ cases we got the rotation groups of the regular polyhedra,
which must be a subgroup by their regularity.
The corresponds polyhedra for each group can be found by comparing elements and is presented in the table bellow:
\[\begin{matrix}
	|I|&|H_1|&|H_2|&|H_3|&|G|\\
	1&1& & &0  &\text{Trivial}\\
	2&n&n& &n  &\text{Equator Family}\\
	3&2&n&n&2n&\text{Equator and Flip Family}\\
	3&2&3&3&12&\text{Tetrahedron}\\
	3&2&3&4&24&\text{Octahedron and Cube}\\
	3&2&3&5&60&\text{Icosahedron and Dodecahedron}\\
\end{matrix}\]
