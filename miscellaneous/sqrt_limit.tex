% Copyright 2023-2024 Kieran W Harvie. All rights reserved.

\section{Square Root Limit}
\label{showcase:sqrt_limit}
While working on something else and the Stack Overflow and the "Hot Network Questions" brought up the following limit:
\[\lim_{x\rightarrow \infty} \bigg(\sqrt{x^2+x+1} - x\bigg) = \frac{1}{2}\]
The questioner asked why the limit wasn't $0$ as $\sqrt{x^2+o(x^2)}\rightarrow x$.
The simple answer is that you can't split up a sum in a limit like that when each term isn't finite.
\\

An interesting point in this presentation is that $x^2+x+1$ is algebraically manipulated in three different ways, 
if you include not manipulating it as a way,
and each manipulation has a different analysis step.
Pretty cool to see the algebra-analysis relationship like that.

\subsubsection{Algebra Answer}
This answer uses the algebraic observation:
\[\sqrt{a^2+b}-a =\frac{(\sqrt{a^2+b})^2-a^2}{\sqrt{a^2+b}+a}=\frac{b}{\sqrt{a^2+b}+a}\]
Manipulating the limit into this form gives:
\begin{equation*}
\begin{aligned}
\sqrt{x^2+x+1} - x=& \sqrt{\left(x+\frac{1}{2}\right)^2+\frac{3}{4}}-\left(x+\frac{1}{2}\right)+\frac{1}{2}\\
=&\frac{3}{4}\frac{1}{\sqrt{\left(x+\frac{1}{2}\right)^2+\frac{3}{4}}+\left(x+\frac{1}{2}\right)}+\frac{1}{2}\\
\end{aligned}
\end{equation*}
Observe that the denominator grows without bound but the numerator is constant.
Hence the fraction goes to $0$ meaning only $\frac{1}{2}$ remains, 
as expected.

\subsubsection{L'Hôpital's Rule Answer}
This answer isn't as careful with the early algebra and gets less cancellation.
Instead it uses L'Hôpital's Rule in a kind of cool way.
First perform the algebra:
\begin{equation*}
\begin{aligned}
\sqrt{x^2+x+1} - x=&\frac{x^2+x+1-  x^2}{\sqrt{x^2+x+1} + x} \\
=&\frac{x+1}{\sqrt{x^2+x+1} + x} \\
\end{aligned}
\end{equation*}
Both numerator and denominator grow without bound,
so applying L'Hôpital's rule gives:
\[\sqrt{x^2+x+1} - x\rightarrow\left(\frac{2x+1}{2\sqrt{x^2+x+1}}+1\right)^{-1}=(f(x)+1)^{-1}\]
Where:
\[f(x)=\frac{2x+1}{2\sqrt{x^2+x+1}}\]
Similarly, 
both numerator and denominator grow without bound,
so applying L'Hôpital's rule gives:
\[f(x)\rightarrow \frac{2}{2\frac{2x+1}{2\sqrt{x^2+x+1}}} = \frac{1}{f(x)}\]
Tidying up gives:
\[\lim_{x\rightarrow \infty} f(x) = 1\]
Hence
\[\sqrt{x^2+x+1} - x\rightarrow=(f(x)+1)^{-1} = \frac{1}{2}\]
As expected.

\subsubsection{Holomorphic Answer}
Let $a$ satisfy:
\[a^2-a+1=0\]
Then let us write:
\[\sqrt{x^2+x+1} - x = \sqrt{(x+a)^2+(1-2a)(x+a)}-(x+a)+a\]
Unfortunately $a\not\in \mathbb{R}$ meaning we are now using complex analysis.
\footnote{From $(a+1)(a^2-a+1)=a^3+1=0$ we know $a$ is an odd $6^\text{th}$ root of unity but not $-1$}
\\
Define:
\[h=\frac{1-2a}{x+a}\]
Simple algebra gives:
\[\sqrt{x^2+x+1} - x =(1-2a)\frac{\sqrt{1+h}-1}{h}+a\]
Observe that as $|h|\rightarrow 0$ we've constructed the derivative of $\sqrt{\cdot}$ at $1$ giving:
\[\lim_{|h|\rightarrow 0}\bigg((1-2a)\frac{\sqrt{1+h}-1}{h}+a\bigg) = (1-2a)\frac{1}{2\sqrt{1}}+a = \frac{1}{2}\]
As expected.

\subsubsection{Geometric Progression}
If we want the general $n^\text{th}$-root consider the geometric progression:
\[(a-b)\sum_{k=0}^{n}a^kb^{n-k} = a^{n+1}-b^{n+1}\]
The difference of two squares we used earlier is the $n=1$ case.
\\
\[\frac{(x+h)^\frac{1}{n}-x^\frac{1}{n}}{h} = \frac{1}{\sum_{k=0}^{n-1}(x+h)^\frac{k}{n}x^\frac{n-1-k}{n}}\]
Both numerator and denominator are finite in the $h\rightarrow 0$ limit, giving:
\[\lim_{h\rightarrow 0}\frac{(x+h)^\frac{1}{n}-x^\frac{1}{n}}{h} = \frac{1}{nx^\frac{n-1}{n}}\]
As expected.
\\
\\
We might be apple to apply a similar argument to a more general problem.
Let $f,g$ be functions such the limits $f-g$ and $f^p-g^p$ are finite then we get a cool limit:
\[\sum_{k=0}^{p-1}f^kg^{p-1-k}\]
This is cool since the limit of $f^p-g^p$ looks like it could be related to a cool metric.
\\
\\
Similarly if $f-g$ is finite but both $f^p-g^p$ and $\sum_{k=0}^{p-1}f^kg^{p-1-k}$ are not we could apply L'Hôpital's rule to get an interesting result.
The general case has too much algebra for me to care about now, 
but I might come back to it.
