% Copyright 2025 Kieran W Harvie. All rights reserved.

\section{Dilogarithm}
I saw the following integral:
\[\int\frac{\ln(ax)}{x-1}\,dx=\Li_2(x)+\ln(ax)\ln(1-x)\]
And noticed that the final term can be expressed as an integral very similar to the left hand side:
\[\int\frac{\ln(ax)}{x-1}\,dx=\Li_2(x)+\ln(ax)\int\frac{1}{x-1}\,dx\]
So the dilogarithm function is like the difference between $\frac{\ln(at)}{1-t}$ varying continuously or discretely.
But to analyse it further it makes sense to move from indefinite to definite integrals,
which requires me to do some revision.

\subsubsection{The Polylogarithm Function:}
The polylogarithm function is defined as the series:
\[\Li_n(x) = \sum_{k=1}^\infty\frac{x^k}{k^n}\]
For $n\leq 1$ the function should be familiar to anyone that has studied generating functions:
\[\begin{aligned}
	\Li_1(x) &= \sum_{k=1}^\infty\frac{x^k}{k} = -\ln(1-x) \\ 
	\Li_0(x) &=\sum_{k=1}^\infty x^k =\frac{x}{1-x} \\
	\Li_{-1}(x) &=\sum_{k=1}^\infty kx^k = \frac{x}{(1-x)^2} \\
\end{aligned}\]
And the method for incrementing $n$ should be equally familiar:
\[\begin{aligned}
	\Li_{n+1}(x) &= \sum_{k=1}^\infty\frac{x^k}{k^{n+1}}\\
	&= \sum_{k=1}^\infty\int_0^x\frac{t^{k-1}}{k^{n}}\,dt\\
	&= \int_0^x\frac{1}{t}\sum_{k=1}^\infty\frac{t^k}{k^{n}}\,dt\\
	&= \int_0^x\frac{\Li_n(t)}{t}\,dt\\
\end{aligned}\]
We'd intuitively expect the singularity at $t=0$ to be fine as the series for $\Li_n(t)$ has no constant term and can easily `absorb' the $t^{-1}$:
\[\frac{\Li_n(t)}{t} = \sum_{k=0}^\infty\frac{t^k}{(k+1)^n}\]
Meaning we can use this integral to expand the domain of $\Li_n(x)$ beyond $|x|<1$ by analytic continuation.
For example consider the $n=2$ integral,
which is naturally called the dilogarithm function:
\[\Li_2(x)=-\int_0^x\frac{\ln(1-t)}{t}\,dt\]
%= \int^1_{1-z}\frac{\ln(t)}{1-t}\,dt\]
The absorption from before looks like:
\[-\frac{\ln(1-t)}{t} = \frac{1}{t}\sum_{k=1}^\infty\frac{t^k}{k}= \sum_{k=0}^\infty\frac{t^k}{k+1}\]
And can easily be made rigorous through L'Hôpital's rule.

\subsubsection{Reflection Formula:}
%\[\begin{matrix}
%\int_a^bu(t)v'(t)\,dt+\int_a^bu'(t)v(t)\,dt=\left[u(t)v(t)\right]_a^b\\
%v(t)=\ln(t)\quad\quad u(t)=\ln(1-t)
%\end{matrix}\]
An important and useful result of the dilogarithm function is the reflection formula.
To derive it start with the following change of variables:
\[\begin{aligned}
	\left[\ln(t)\ln(1-t)\right]_a^b &= \int_a^b\frac{\ln(1-t)}{t}\,dt-\int_a^b\frac{\ln(t)}{1-t}\,dt\\
	&=-\int_b^a\frac{\ln(1-t)}{t}\,dt-\int_{1-b}^{1-a}\frac{\ln(1-t)}{t}\,dt\\
	&=\Li_2(a)-\Li_2(b)+\Li_2(1-a)-\Li_2(1-b)\\
\end{aligned}\]
Which shows that:
\[\Li_2(x)+\Li_2(1-x)+\ln(x)\ln(1-x)\]
Is constant for all $x$.
To get this constant consider the limit $\lim_{x\rightarrow 0^+}$ as the values of the dilogarithm terms directly follow from the series definition:
\[\Li_2(0) = 0\quad\quad\Li_2(1) = \frac{\pi^2}{6}\]
And the logarithmic product is given by applying L'Hôpital's rule twice:
\[\begin{aligned}
	\lim_{x\rightarrow 0^+}\ln(x)\ln(1-x) &=\lim_{x\rightarrow 0^+}\frac{\ln(x)}{\ln^{-1}(1-x)}\\
	&=\lim_{x\rightarrow 0^+}\frac{\ln^2(1-x)(1-x)}{x}\\
	&=\lim_{x\rightarrow 0^+}-2\ln(1-x)-\ln^2(1-x)\\
	&=0\\
\end{aligned}\]
Hence:
\[\Li_2(x)+\Li_2(1-x)+\ln(x)\ln(1-x) = \frac{\pi^2}{6}\]

\subsubsection{Original Integral:}
With the reflection formula it's easy to calculate the original integral:
\[\begin{aligned}
	\int_a^b\frac{\ln(ct)}{1-t}\,dt &=\int_a^b\frac{\ln(t)+\ln(c)}{1-t}\,dt \\
	&= \Li_2(1-b)-\Li_2(1-a)-\ln(c)\ln(1-b)+\ln(c)\ln(1-a)\\
	&= \Li_2(a)-\Li_2(b)+\ln(ca)\ln(1-a)- \ln(cb)\ln(1-b)\\
\end{aligned}\]
This is already a continuous versus discrete integral with $ln$ over $t\in[a,b]$ versus $c$,
but if we consider the following identity:
\[\ln(\lambda_0)\ln(\lambda_1)+\ln(\lambda_2)\ln(\lambda_3) = \frac{1}{2}\left(\ln(\lambda_0\lambda_2)\ln(\lambda_1\lambda_3)+\ln\left(\frac{\lambda_0}{\lambda_2}\right)\ln\left(\frac{\lambda_1}{\lambda_3}\right)\right)\]
We can set $c$ to the reciprocal of the geometric average of $a$ and $b$:
\[
\int_a^b\frac{\ln(t)-\ln(\sqrt{ab})}{1-t}\,dt =\Li_2(a)-\Li_2(b)+\ln\left(\sqrt{\frac{a}{b}}\right)\ln\left(\frac{1-a}{1-b}\right) \]
Which feels nice and clean.
And,
since there's a lot of identities for the sum of two dilogarithm functions,
I suspect there's even cleaner identities with specific choices of $a$ and $b$. 

\subsubsection{Rogers L-Function:}
The Rogers L-Function is defined\footnote{There's an alternative definition without the $\frac{6}{\pi^2}$ factor.} as:
\[L(x) = \frac{6}{\pi^2}\left(\Li_2(x)+\frac{1}{2}\ln(x)\ln(1-x)\right)\]
This form is can be motivated the reflection formula for the dilogarithm function and gives Rogers L-Function a particularly pleasing analog:
\[L(x)+L(1-x) = 1\]

\subsubsection{Another Integral:}
Another interesting integral follows from our original by a simple change of variables and cancellation:
\[\begin{aligned}
\int\frac{\ln(x)}{x-a}\,dx&=\int\frac{\ln(at)}{t-1}\,dt\\
&= \Li_2(t)+\ln(at)\ln\left(1-t\right)\\
&= \Li_2\left(\frac{x}{a}\right)+\ln(x)\ln\left(1-\frac{x}{a}\right)\\
\end{aligned}\]
