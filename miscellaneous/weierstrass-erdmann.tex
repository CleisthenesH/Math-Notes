% Copyright 2023 Kieran W Harvie. All rights reserved.

\section{Weierstrass-Erdmann Corner Conditions}
I saw an enlightening Calculus of Variation problem that involved some of the more interesting results from the field.

Find $x(t)$ such that:
\[I=\int_0^2(1-\dot{x})^2\dot{x}^2\,dt\]
Is minimized with endpoints $x(0)=0$ and $x(2)=1$.
\\

By inspection $F(t,x,\dot{x})=(1-\dot{x})^2\dot{x}^2\geq 0$ making $I\geq 0$ with the $F$ vanishing iff $\dot{x} = 0,1$.
Hence any piecewise combinations of lines with gradients $0$ or $1$ from $(0,0)$ to $(2,1)$ works.
For example,
the solution of a straight line from $(0,0)$ to $(1,1)$ then to $(2,1)$ is a minimum.
This solution is continuous but not differentiable.

Now lets use more standard methods.
First calculate a table of useful properties of $F$:
\begin{equation*}
\begin{aligned}
	F_x =& 0 \\
	F_{\dot{x}} =& 2\dot{x}(1-\dot{x})(1-2\dot{x})\\
	F_{\dot{x}^2} =& 12\dot{x}^2-12\dot{x}+2\\
	\frac{d}{d\,t}F_{\dot{x}} =& \ddot{x}F_{\dot{x}^2} \\
	=& \ddot{x}(12\dot{x}^2-12\dot{x}+2)\\
\end{aligned}
\end{equation*}

The Euler-Lagrange equation,
combined with the first and last line,
means that all extremals take the form:
\[x(t) = at+b\]
These are lines, 
but a line directly between the endpoints is:
\[x(t) = \frac{1}{2}t\]
Meaning:
\[F= \frac{1}{16}>0\]

This is clearly greater than the piecewise solutions,
but lets pretend we don't know that and use some other results. 
\\

\subsection{Only solution is not a Minimum}
The Legendre-Clebsch condition says that in order for $x_0(t)$ to minimize $\int_U F(t,x,\dot{x})\,dt$ we require:
\[ t\in U \Rightarrow F_{\dot{x}^2}(t,x_0,\dot{x_0}) \geq 0\]
But we have:
\[x(t)=\frac{1}{2}t \Rightarrow F_{\dot{x}^2} = -1\]
Meaning the only solution of the Euler-Lagrange equation isn't a minimum.
Does that mean there are none?
No, just none in the set of functions that match the Euler-Lagrange equations assumptions.
(At least continuously differentiable, but the set may be bigger).
But clearly minimal solutions exits from the beginning.

\subsection{Corners} 
Continue to pretend we don't know the piecewise solutions.
The Weierstrass-Erdmann corner condition states that for first-order derivate discontinuities,
or "corners", 
at $t_0$ the functions on either side,
$x_0$ and $x_1$ must satisfy:
\[F_{\dot{x}}(t_0,x_0,\dot{x_0}) = F_{\dot{x}}(t_0,x_1,\dot{x_1})\]
and
\[F_{\dot{x}}(t_0,x_0,\dot{x_0})\dot{x_0}(t_0)-F(t_0,x_0,\dot{x_0}) = F_{\dot{x}}(t_0,x_1,\dot{x_1})\dot{x_1}(t_0)-F(t_0,x_1,\dot{x_1})\]

Our $F$ is pretty simple, 
only put a restrictions on $\dot{x_n}$, 
\begin{equation*}
\begin{aligned}
	\dot{x_0}(1-\dot{x_0})(1-2\dot{x_0}) =& \dot{x_1}(1-\dot{x_1})(1-2\dot{x_1}) \\
	\dot{x_0}^2(1-\dot{x_0})(1-3\dot{x_0}) =& \dot{x_1}^2(1-\dot{x_1})(1-3\dot{x_1}) \\
\end{aligned}
\end{equation*}
This means $\dot{x_0},\dot{x_1}\in\{0,1\}$.
It's hard to see this from algebra alone,
or maybe it's just that I'm tired,
but plotting $x(1-x)(1-2x)$ and $x^2(1-x)(1-3x)$ shows it pretty well.
\\

Observe that this is just the piecewise solutions we got from first inspection.
Pretty cool.

\subsection{Next Steps}
There were other goodies in the place I found this problem.

First there are some notes here for implicit boundaries:
\[S(x,y)=0\]
Which might be useful for shader stuff,
but haven't pieced it together yet.
\\

There is also an iterative method of solving Calculus of Variation problems.
It might be interesting to inspect this iteration and see the differential solutions converge to the piecewise solutions.
Or maybe something else cooler!!
