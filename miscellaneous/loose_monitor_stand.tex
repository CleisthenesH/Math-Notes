% Copyright 2025 Kieran W Harvie. All rights reserved.

\section{Loose Monitor Stand}
My monitor stand at work is loose.
By this I mean the monitor rotates freely on its stand such that when I pick up the by the stand moving it to changes the monitors angle to me:
\begin{center}
\begin{tikzpicture}
	\foreach \angle in {0,-35,-70}
	{\begin{scope}[rotate=\angle]
		\draw[->] (0,0)--(0,4) node[midway,above left] {$\mathbf{R}$};
		\draw[dashed,->] (0,4)--(0,5) node[midway,above left] {$\mathbf{d}$};
		\draw[blue,thick,|-|] (0,4)+(180-0.4*\angle:1)--+(-0.4*\angle:1) coordinate (C);

		\draw[blue,dashed] (1,4) coordinate(A) -- (0,4) coordinate (B)
			pic[fill=blue!50,draw=blue] {angle};
	\end{scope}}
\end{tikzpicture}

The blue bar is the monitor.

$\mathbf{R}$ is the vector from my seat to the center of mass of the monitor.

$\mathbf{d}$ is the vector from the center of mass of the monitor to the monitor's stand where the stand in collinear with $\mathbf{R}$ and slightly behind.
\end{center}
I wanted to quantify how the monitor's angle to me is effected by moving the stand.
And I instantly thought of the following equation from the conservation of the orbit and spin angular momentum: 
% Better brace positioning but unable to use mathclap in underbrace
%\[\overbrace{\omega_OM|R|^2
%	+\rlap{$\underbrace{\phantom{\omega_OM|d|^2+\omega_SM|d|}}_\text{Center of Mass Correction}$}
% \omega_OM|d|^2}^\text{Orbit}+\overbrace{\omega_SM|d|^2+\omega_SI}^\text{Spin}=0\]
\[
\rlap{$\overbrace{\phantom{\omega_OM|R|^2+\omega_OM|d|^2}}^\text{Orbit}$}
\omega_OM|\mathbf{R}|^2+\underbrace{
 \omega_OM|\mathbf{d}|^2+\omega_SM|\mathbf{d}|^2
 }_{\mathclap{\text{Correction for the offset from the Center of Mass}}}+\omega_SI
\llap{$\overbrace{\phantom{\omega_SM|d|^2+\omega_SI}}^\text{Spin}$}
=0
\]
And was struck by how far removed this is from the first principles.
That for a closed system of point masses $m_n$ located at $\mathbf{x}_n$ the angular momentum\footnote{Using discrete sums because I don't want to deal with Jacobians and the physics is largely the same between discrete sums and integrals.} given by:
\[\mathbf{L} = \sum_nm_n\mathbf{x}_n\times\dot{\mathbf{x}}_n\]
Is conserved.
To get the former from the latter start let $M$ be the total mass and $\mathbf{R}$ be the center of mass as defined by:
\[M=\sum_nm_n,\quad M\mathbf{R}=\sum_nm_n\mathbf{x}_n\]
We can now consider $\mathbf{x}_n$ as offset from $\mathbf{R}$ by defining $\mathbf{r}_n$ as:
\[\mathbf{r}_n = \mathbf{x}_n-\mathbf{R}\]
And with simple algebra we get:
\[\begin{aligned}
	\sum_nm_n\mathbf{r}_n &= \sum_nm_n(\mathbf{x}_n-\mathbf{R})\\
	&= M\mathbf{R}-M\mathbf{R} = 0\\
\end{aligned}\]
Which lets us use the bilinearity of the cross product to simplify $\mathbf{L}$:
\[\begin{aligned}
	\mathbf{L}&=\sum_nm_n\mathbf{x}_n\times\dot{\mathbf{x}}_n\\
	&=\sum_nm_n(\mathbf{R}+\mathbf{\mathbf{r}}_n)\times(\dot{\mathbf{R}}+\dot{\mathbf{r}}_n)\\
	&=\sum_nm_n\mathbf{R}\times\dot{\mathbf{R}}+\sum_nm_n\mathbf{R}\times\dot{\mathbf{r}}_n+\sum_nm_n\mathbf{\mathbf{r}}_n\times\dot{\mathbf{R}}+\sum_nm_n\mathbf{\mathbf{r}}_n\times\dot{\mathbf{r}}_n\\
	&=M\mathbf{R}\times\dot{\mathbf{R}}+\mathbf{R}\times\left(\sum_nm_n\dot{\mathbf{r}}_n\right)+\left(\sum_nm_n\mathbf{\mathbf{r}}_n\right)\times\dot{\mathbf{R}}+\sum_nm_n\mathbf{\mathbf{r}}_n\times\dot{\mathbf{r}}_n\\
	&=M\mathbf{R}\times\dot{\mathbf{R}}+\sum_nm_n\mathbf{\mathbf{r}}_n\times\dot{\mathbf{r}}_n\\
\end{aligned}\]
And through similar arguments we also obtain:
\[\sum_nm_n(\mathbf{\mathbf{r}}_n-\mathbf{d})\times(\dot{\mathbf{r}}_n-\dot{\mathbf{d}})=M\mathbf{d}\times\dot{\mathbf{d}}+\sum_nm_n\mathbf{\mathbf{r}}_n\times\dot{\mathbf{r}}_n\]
Which when combined gives:
\[\mathbf{L}=M(\mathbf{R}\times\dot{\mathbf{R}}-\mathbf{d}\times\dot{\mathbf{d}})+\sum_n m_n(\mathbf{\mathbf{r}}_n-\mathbf{d})\times(\dot{\mathbf{r}}_n-\dot{\mathbf{d}})\]
The two orbit terms have successfully been serrated from the spin term.
The next step is to show that the orbit terms have the same angular velocity magnitude $|\omega_O|$.
This equality follows from $\mathbf{R}$ and $\mathbf{R+d}$ having the same angular velocity,
as $\mathbf{R+d}$ is my arm and $\mathbf{R}$ is the center of mass above my wrist,
meaning they form similar triangles with their velocity vectors:

\begin{center}
\begin{tikzpicture}
	\draw[->] (0,0)--(4,0) node[midway,below] {$\mathbf{R}$};
	\draw[->] (4,0)--(6,0) node[midway,below] {$\mathbf{d}$};
	\draw[->] (4,0)--(4,2) node[midway,right] {$\dot{\mathbf{R}}$};
	\draw[->] (6,0)--(6,2) node[midway,right] {$\dot{\mathbf{R}}$};
	\draw[->] (6,2)--(6,3) node[midway,right] {$\dot{\mathbf{d}}$};

	\draw[gray,dashed] (0,0)--(6,3);
	\draw[gray,dashed] (4,2)--(6,2);
\end{tikzpicture}
\end{center}
Hence $\mathbf{d}$ and $\dot{\mathbf{d}}$ is also similar triangle.
For a more algebraic argument consider that $\mathbf{R}$ and $\mathbf{R+d}$ having the same angular velocities means:
\[\begin{aligned}
	|\omega_O||\mathbf{R}|&=|\dot{\mathbf{R}}| \\
	|\omega_O||\mathbf{R}+\mathbf{d}|&=|\dot{\mathbf{R}}+\dot{\mathbf{d}}| \\
\end{aligned}\]
But since $\mathbf{R}$ and $\mathbf{R+d}$ are collinear with each other,
and so are their derivatives,
we can use the reverse triangle inequality to give:
\[\begin{aligned}
	|\mathbf{d}| = |\mathbf{R+d-R}| = ||\mathbf{R+d}|-|\mathbf{R}||\\
	|\dot{\mathbf{d}}| = |\dot{\mathbf{R}}+\dot{\mathbf{d}}-\dot{\mathbf{R}}| = ||\dot{\mathbf{R}}+\dot{\mathbf{d}}|-|\dot{\mathbf{R}}||\\
\end{aligned}\]
Hence:
\[\begin{aligned}
	|\dot{\mathbf{d}}| 
	&=||\dot{\mathbf{R}}+\dot{\mathbf{d}}|-|\dot{\mathbf{R}}||\\
	&=||\omega_O||\mathbf{R+d}|-|\omega_O||\mathbf{R}||\\
	&=|\omega_O|||\mathbf{R+d}|-|\mathbf{R}||\\
	&=|\omega_O||\mathbf{d}|\\
\end{aligned}\]
%\[d=\alpha R\]
%Since
%\[d+R=(1+\alpha)R\]
%From circular 
%\[\dot{d}+\dot{R}=(1+\alpha)\dot{R}\]
%Hence:
%\[\dot{d} =\alpha\dot{R}\]
%result:
%\[|d\times\dot{d}| = |\alpha^2R\times R| = |\alpha^2|R\times R| = \alpha|w_O| = w_1|d|^2\hat{z}\]
%\[d\times\dot{d} = \alpha^2R\times R = w_1(\alpha|R|)^2\hat{z} = w_1|d|^2\hat{z}\]
Using either method to establish equality we can now introduce the spin angular velocity $\omega_S$ and reduce our previous expression of $\mathbf{L}$ to:
\[\omega_OM|\mathbf{R}|^2+\omega_OM|\mathbf{d}|^2+\omega_S\sum_n m_n|\mathbf{\mathbf{r}}_n-\mathbf{d}|^2\]
Where we took advantage of the motions being both circular and planar to only consider the component out of the plane.
\\

The final simplification is to use the parallel axis theorem to change the right most term into a more standard form.
Let $I$ be the monitor's moment of inertia around its own center of mass and express $\mathbf{r}_n$ in terms of coordinates parallel $x_n$ and perpendicular $y_n$ to $\mathbf{d}$:
\[\begin{aligned}
	\sum_n m_n|\mathbf{\mathbf{r}}_n-\mathbf{d}|^2 =&\sum_nm_n((x_n-|\mathbf{d}|)^2+y_n^2)\\
	=&\sum_nm_n(x_n^2+y_n^2)-2|\mathbf{d}|\sum_nm_nx_n+|\mathbf{d}|^2\sum_nm_n\\
	=&I+M|\mathbf{d}|^2\\
\end{aligned}\]
Substitution completes the derivation.

\subsubsection{Bonus: Calculation of $I$:}
Assume the monitor was was a flat rectangle with with the axis of rotation going through its center and lying in its plane.
Then the monitor would have the same moment of inertia as a mental rod of the same mass perpendicular and centered to the axis:
\[I = \frac{1}{12}mL^2\]
This is because moving mass along the axis of rotation doesn't effect the moment of inertia.
\\

To calculate the moment of inertia for a rod of mass $M$ perpendicular to the axis of rotation at signed distances $a$ and $b$ away from the axis define the linear density as:
\[\rho = \frac{M}{b-a}\]
Then simply calculate the integral as normal:
\[\int_a^b\rho x^2\,dx = \rho\frac{b^3-a^3}{3} = M\frac{b^2+ba+a^2}{3}\]
Where $a=\frac{1}{2}L,\,b=-\frac{1}{2}L$ is the monitor subcase.

