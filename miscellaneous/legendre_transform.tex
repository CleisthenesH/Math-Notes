% Copyright 2023 Kieran W Harvie. All rights reserved.

\section{Legendre Transform}
I've used the Legendre transform a lot and always understood that it transformed between conjugate variables but never understood the motivation of its formal definition. 
Well I've thought of a good motivated argument.
\\

Let $f$ be a strictly convex differentiable function in $x$.
Then $\frac{\partial f}{\partial x}$ is a strictly increasing function.
This means $\frac{\partial f}{\partial x}$ is injective and we can defined a function $g$ such that:
\[g\left(\frac{\partial f}{\partial x}\bigg|_{x_0}\right) = x_0\]
And we can further define a new function $f^*$,
up to an additive constant,
such that:
\[\frac{\partial f^*}{\partial p} = g(p)\]
This is the Legendre Transform.
\\

This transforms usefulness comes from looking at the functions differential.
Let $f(x,y)$ and consider a point $(x_0,y_0)$ where:
\[\frac{\partial f}{\partial x}\bigg|_{(x_0,y_0)} = p,\quad\frac{\partial f}{\partial y}\bigg|_{(x_0,y_0)} = v \]
Giving the differential:
\[df = pdx+vdy\]
Now consider a $f^*(p,y)$ where $f^*$'s first argument is Legendre transformed from $f$'s, 
and the second is unchanged.
This gives:
\[df^* = xdp+vdy\]
You can see that the coefficient and differential of the first term has swapped,
pretty useful.
\\

This argument seems more motivated to me, 
but the domain of the transform can be widened to merely require $f$ being convex on $D$ by defining:
\[f^*(p) = \sup_{x\in D}\big[px-f(x)\big]\]
To see that this definition extends the strictly convex differentiable case derive the inner function by $x$ and set to zero:
\[p-f'(x)=0\]
But $f$ is injective,
so we can defined $g$ as before and obtain:
\[f'(g(p)) = p\]
meaning the maximal value of $x$ is $g(p)$, 
giving:
\[f^*(p) = pg(p)-f(g(p))\]
Deriving by $p$ gives:
% f^*'(p) doesn't work, don't feel like fixing
\[\frac{d}{d p}f^*(p) = g(p)+pg'(p)-g'(p)f'(g(p))\]
Substituting in $f'(g(p)) = p$ gives:
\[\frac{d}{d p}f^*(p) = g(p)+pg'(p)-g'(p)p = g(p)\]
As expected.
