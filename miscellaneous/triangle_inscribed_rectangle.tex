% Copyright 2025 Kieran W Harvie. All rights reserved.

\section{Triangle Inscribed Rectangle}
Consider a triangle $OAB$ with points $A'$ on $OA$, $B'$ on $OB$, and $P$ on $AB$ such that $OA'PB'$ is a rectangle:
\begin{center}
\begin{tikzpicture}
	\draw (0,0) node[below left] {$O$} -- (8,0) node[below right]{$A$} -- (0,6)  node[above left] {$B$}-- cycle;
	\draw (5.3333333,0) node[below]{$A'$} -- (5.3333333,2) node[above right]{$P$} -- (0,2) node[left] {$B'$};
\end{tikzpicture}
\end{center}
Given $|AB|$, $|A'P|$, and $|B'P|$ find the length of the other segments.

\subsubsection{Algebraic Reduction:}
Through the shared angles the triangles $OAB$, $A'AP$, and $B'PB$ are similar.
Let the ratio of $A'AP$ to $OAB$ be $S_A$ and likewise for $S_B$.
From the Pythagorean theorem we have:
\[ (|PA'|S_A)^2+(|PB'|S_B)^2=|AB|^2\]
And by $P$ lying on $AB$ we get:
\[\begin{aligned}
	|AB| &= |AP|+|PB|\\
	&=\frac{|AB|}{S_A}+\frac{|AB|}{S_B}\\
\end{aligned}\]
These equations reduce to solving:
\[\begin{matrix}
	(ax)^2+(by)^2=c^2\\
	x+y=xy\\
\end{matrix}\]
For $x$ and $y$ given $a$, $b$, and $c$.

\subsubsection{Substitution:}
The elliptical condition suggest the following substitution:
\[x = \frac{c}{a}\cos(\phi),\quad y = \frac{c}{b}\sin(\phi)\]
Which, 
after cleaning up,
gives:
\[\frac{b}{\sqrt{a^2+b^2}}\cos(\phi)+\frac{a}{\sqrt{a^2+b^2}}\sin(\phi)=\frac{c}{2\sqrt{a^2+b^2}}\sin(2\phi)\]
This suggests a second substitution of:
\[\cos(\theta_0) = \frac{a}{\sqrt{a^2+b^2}},\quad \sin(\theta_0) = \frac{b}{\sqrt{a^2+b^2}}\]
In order to combine the terms with the angle sum formula:
\[\sin(\phi+\theta_0) = \frac{c}{2\sqrt{a^2+b^2}}\sin(2\phi)\]
Which is where my ability to simplify ends.
Inputing into a computer algebra system didn't turn up anything up either.

\subsubsection{Square Case:}
When $a=b$ we can divide it out and use the hyperbolic condition to get a quadratic in $x+y$:
\[\begin{aligned}
	x^2+y^2&=\left(\frac{c}{a}\right)^2\\
	(x+y)^2&=\left(\frac{c}{a}\right)^2+2(x+y)\\
\end{aligned}\]
For which the positive solution is:
\[x+y = 1+\sqrt{1+\left(\frac{c}{a}\right)^2}\]
Using a similar substitution and double angle technique as before gives:
\[x=\cos(\phi),\quad y=\sin(\phi)\]
\[\sin\left(\phi+\frac{\pi}{4}\right)=\frac{\sqrt{2}}{2}\left(1+\sqrt{1+\left(\frac{c}{a}\right)^2}\right)\]

\subsubsection{Loose Ends:}
Here's some loose ends that I found trying to solve this problem:
\begin{itemize}
	\item Prosthaphaeresis: Was an algorithm used in the 16$^{th}$ century that approximated multiplication and division using trigonometric identities similar to how logarithms are used with slide rules. 
	It's very similar to how substitutions where used here.
	\item Phasor Addition: There's an formula used in the addition of phasors that I thought might be useful:
	\[\sum_kA_k\cos(\omega t+\delta_k) = A\cos(\omega t+\delta)\]
	Where:
	\[A^2 = \sum_k\sum_jA_kA_j\cos(\delta_k-\delta_j)\quad\text{ and }\quad\tan(\delta) = \frac{\sum_k A_k\sin(\delta_k)}{\sum_k A_k\cos(\delta_k)}\]
	\item Symmetric Matrix Conics: There's a method to intersect conics using symmetric $3\times3$ matrices which seems useful, but I'd need to learn it and comeback to this problem.
	\item Factoring the Hyperbolic Condition: 
	\[x+y=xy\Leftrightarrow (1-x)(1-y)=1\]
	And:
	\[x+y=xy\Leftrightarrow(x+y)^2-(x-y)^2=4xy=4(x+y)\]
	\item Sum-to-Product Diagram: If you look up the diagram for how the sum-to-product diagram is proven you'll see some geometry similar to the square case.
	\item Integrator: We have the ability to solve the square case and the equations defining a solution look pretty smooth.
	So the investigation of numeric solutions might be interesting.
	In particular there are integrators that preserve `symplectic' forms which I'm pretty sure one or both of the conditions are.
	Although it may only be useful to preserve the hyperbolic condition as the elliptical one should change,
	another point for further study.
\end{itemize}
