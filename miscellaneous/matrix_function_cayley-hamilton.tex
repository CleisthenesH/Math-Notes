% Copyright 2023 Kieran W Harvie. All rights reserved.

\section{Matrix Function Evaluation using Cayley-Hamilton}
(This is something I'm 70\% sure I've seen before but I saw it again today).
(Assume the field is $\mathbb{R}$, don't feel like dealing with others).
\\

The Cayley-Hamilton theorem states that a matrix is the root of it's own characteristic equation.
A cool application of this is simplifying the evaluation of (analytic) matrix functions.
\\

Let $M$ be a matrix with characteristic polynomial $p$,
and eigenvalues $\lambda_i$ with multiplicity $m_i$.
Let the analytic function $f$ be given by:
\[f(x) = \sum_{k=0}^\infty f_kx^k\]
Define a new function $r$ that is interpolated on $\lambda_i$ such that:
\[0\leq k< m_i \Rightarrow\left.\frac{d^k f}{d x^k}\right|_{\lambda_i} =\left.\frac{d^k r}{d x^k}\right|_{\lambda_i}\]
This means that for some analytic function $q$ we have:
\[f(x) = q(x)p(x)+r(x)\]

Now convert these analytic functions to analytic matrix functions in the natural way.
From Cayley-Hamilton theorem we have:
\[f(M) = q(M)p(M)+r(M) = r(M)\]

\subsection{Worked Example}
Let $M$ be a $2\times 2$ matrix with two distinct eigenvalues.
Let $f(x) = \exp(tx)$ then:
\[r(x) = (x-\lambda_0)\frac{\exp(t\lambda_1)}{\lambda_1-\lambda_0}+(x-\lambda_1)\frac{\exp(t\lambda_0)}{\lambda_0-\lambda_1}\]
Hence:
\[f(M) = r(M) =(M-\lambda_0I)\frac{\exp(t\lambda_1)}{\lambda_1-\lambda_0}+(M-\lambda_1I)\frac{\exp(t\lambda_0)}{\lambda_0-\lambda_1} \]

\subsection{Original Application}
The original problem was to minimize:
\[f(X) = \sum_{k=0}^n\left|\left|\exp\left(\frac{2\pi}{k}B\right)A-X\right|\right|^2\]

Through some procedure similar to the above reasoning it was obtained:
\[\exp(\alpha B) = I+ \sin(\alpha)B+(1-\cos(\alpha))B^2\]

Hence:
\begin{equation*}
\begin{aligned}
&\sum_{k=0}^n\left|\left|\exp\left(\frac{2\pi}{k}B\right)A-X\right|\right|^2\\
=&\sum_{k=0}^n\left|\left|\left(I+\sin\left(\frac{2\pi}{k}\right)B+\left(1-\cos\left(\frac{2\pi}{k}\right)\right)B^2\right)A-X\right|\right|^2\\
=&\sum_{k=0}^n\left|\left|\bigg((I+B^2)A-X\bigg)+\left(\sin\left(\frac{2\pi}{k}\right)B+\cos\left(\frac{2\pi}{k}\right)B^2\right)A\right|\right|^2\\
\end{aligned}
\end{equation*}
The left term independent of $k$ and the right term is independent of $X$.
This means each term of the sum is minimized at:
\[X=(I+B^2)A\]
And since each term is minimized so is the whole sum.

\subsection{Generalization of the Worked Example}
Let $M$ be a matrix with characteristic polynomial $p$, distinct eigenvalues $\lambda_i$, and corresponding eigenvectors $v_i$.
The Lagrange functions $\ell_i(x)$ on $\{\lambda_i\}$ are given by:
\[\ell_i(x) = \prod_{j\neq i} \frac{x-\lambda_j}{\lambda_i-\lambda_j}\]
And have the standard desirable property that:
\[\ell_i(x) = \begin{cases} 1&x=\lambda_i\\0&x=\lambda_j \text{ where } i\neq j\\\end{cases}\]
From the distinctness of $\lambda_i$:
\[p(x) | \ell_i(x)(x-\lambda_i)\]
(Where we don't necessarily have equality because of the denominators in $\ell_i(x)$).
From Cayley-Hamilton theorem we have:
\[\ell_i(M)(M-\lambda_iI) = 0 \Rightarrow \ell_i(M)M=\lambda_i\ell_i(M)\]
Which is pretty cool by itself,
since $\ell_i$ is like an eigenmatrix,
but also gives:
\begin{equation*}
\begin{aligned}
	\ell_i(M)Mv_j =& \lambda_i\ell_i(M)v_j \\
	\Rightarrow\lambda_j\ell_i(M)v_j =& \lambda_i\ell_i(M)v_j \\
\end{aligned}
\end{equation*}
Hence either $\ell_i(M)v_j =0$ or $\lambda_i=\lambda_j$,
but the eigenvalues are unique meaning $\ell_i(M)v_j = 0$ or $i=j$.
\\

This property is useful when we consider the standard remainder function $r$ written as:
\[r(x) = \sum_if(\lambda_i)\ell_i(x)\]
Giving:
\begin{equation*}
\begin{aligned}
	f(M) =& \sum_if(\lambda_i)\ell_i(M)\\
	\Rightarrow f(M)\sum_ju_jv_j =& \sum_ju_jf(M)v_j \\
	=& \sum_ju_j\sum_if(\lambda_i)\ell_i(M)v_j\\
	=& \sum_ju_jf(\lambda_j)\ell_j(M)v_j\\
\end{aligned}
\end{equation*}

Hence the value of the product,
over the space spanned by the eigenvalues of $A$,
is a dot product of a term only dependent on the vector $u$ and a term only dependent on the matrix and function.
\\

An enlightening example of this is when $f(M)=\exp(tM)$,
which has a natural interpretation as the continuous application of $M$ for a period of $t$ time,
and gives an initial term,
a time-dependent term,
and a symmetry of $M$ term:
\[\exp(tA)u = \sum_ju_j\exp(t\lambda_j)\ell_j(M)v_j\]

\subsection{Derivatives Generalization:}
I want to generalize the previous results for eigenvalues with multiplicities.
(I just want to quickly write this down and come back latter if I feel like it,
it has not been proof read)
This will be with easier with a definition and two lemma:.

Let $M$ be a matrix with characteristic polynomial $p$, eigenvalues $\lambda_i$, corresponding eigenvectors $v_i$ and multiplicities $m_i$.
\subsubsection{Modified Lagrange Function:}
Define $\ell_i(x)$ as:
\[\ell_i(x)=\prod_{j\neq i}\left(\frac{x-\lambda_j}{\lambda_i-\lambda_j}\right)^{m_j}\]
We still have:
\[\ell_i(x) = \begin{cases} 1&x=\lambda_i\\0&x=\lambda_j \text{ where } i\neq j\\\end{cases}\]
But also for $n < m_j$ and $i\neq j$:
\[\ell_i^{(n)}(\lambda_j) = 0\]
this can be verified from the product rule.

\subsubsection{Lemma:}
From induction on $n$ we have:
\[(M-\lambda_iI)^nv_j = (\lambda_i-\lambda_j)^nv_j\]
Recall that:
\[p(x) |\ell_i(x)(x-\lambda_i)^{m_i}\]
Hence:
\[\ell_i(M)(M-\lambda_iI)^{m_i} = 0\]
And:
\[(\lambda_j-\lambda_i)^{m_i}\ell_i(M)v_j = 0\]
Hence either $\ell_i(M)v_j =0$ or $\lambda_i=\lambda_j$.
\\

\subsubsection{Lemma: Interpolation}
Define:
\[r(x) = \sum_i\ell_i(x)\sum_{k=1}^{m_i}p_n\frac{(x-\lambda_i)^n}{n!}\]
Where $p_n$ satisfy:
\[f^{(d)}(\lambda_i) = \sum_{k=0}^{d}p_k\ell_i^{(d-k)}(\lambda_i)\]
(Note that this is just a system of linear equations and can easily be solved through a linear equation).

\subsubsection{Result:}
\[f(M) = r(M) = \sum_i\ell_i(M)\sum_{k=1}^{m_i}p_n\frac{(M-\lambda_iI)^n}{n!}\]
In particular let $u=\sum_ju_jv_j$
\begin{equation*}
\begin{aligned}
	f(M)u =& \sum_i\ell_i(M)\sum_{k=1}^{m_i}p_n\frac{(M-\lambda_iI)^n}{n!}u\\
	=& \sum_ip_0u_i\ell_i(M)v_i\\
\end{aligned}
\end{equation*}
Since if $n>0$ either $\ell_i(M)v_j=0$ or $(M-\lambda_iI)^nv_j = 0$.

