% Copyright 2024 Kieran W Harvie. All rights reserved.

\section{Direct Sum and Tensor Product}
(While doing something else I got the direct sum and tensor product confused again.
So this section is my penance to try and not do that again.)
\\

Lets say we have two vector spaces $V$ and $W$ over the shared field $F$ and want to construct a new vector space based on them,
how would we do that?
Well the most general way would be to consider the formal sum of $V\times W$ over $F$.
\\

The formal sum of the set $S$ over the ring $R$ is a $R$-submodule of standard $R$-module of functions from $S$ to $R$ where the elements are the function  with a finite number of non-zero terms\footnote{As expected, a formal series drops this condition}.
For example the function:
\[f(s) = \begin{cases} 1_R & s=s_0\\ 2_R & s=s_1\\ 0_R & \text{ otherwise} \end{cases}\]

Corresponds to the formal sum:
\[s_0+2_Rs_1\]

In our case $S=V\times W$ and $R$ is the shared underlying field $F$,
upgrading the module to a vector space.
While this construct {\em is} a vector space based on $V$ and $W$,
it doesn't actually use them being vector space in anyway beyond having a set of underlying elements,
and is hence too general for many applications.
\\

We can enrich the structure by the use of quotient spaces.
For example,
let $S$ be the subset of elements of {\bf either} of the following forms:
\begin{equation*}
\begin{aligned}
	k(v,w) -& (kv,kw)\\
	(v_1,w_1)+(v_2,w_2) -& (v_1+v_2,w_1+w_2)\\
\end{aligned}
\end{equation*}

The quotient space of the main space by the span of this set is called the direct sum and is denoted $V\oplus W$ and similarly the equivalent class containing the element $\sum_nk_n(v_n,w_n)$ is denoted $\sum_nk_n(v_n\oplus w_n)$.
\\

The equivalent classes are linear.
The proof is basically just expanding notation but I'll demonstrate the scaling property,
just for future me rereading this in 10 years.
\begin{enumerate}
	\item $k(v,w)-(kv,kw)$ can be written in above form as:
		\begin{equation*}
		\begin{aligned}
			k_0=k,&\quad v_0=v,\quad w_0=w\\
			k_1=-1,&\quad v_1=kv,\quad w_1=kw\\
		\end{aligned}
		\end{equation*}
	\item This means it's in the $k(v\oplus w) - (kv)\oplus(kw)$ equivalent class.
	\item Since the construction of quotient spaces takes $k(v,w)-(kv,kw)$ to $0$ we have:
		\[k(v\oplus w) - (kv)\oplus(kw) = 0\quad\square\]
\end{enumerate}
A similar argument works for:
\[(v_1\oplus w_1)+(v_2\oplus w_2) = (v_1+v_2)\oplus (w_1+w_2)\]

For the equivalent classes to be bilinear instead consider the following forms:
\begin{equation*}
\begin{aligned}
	k(v,w) -& (kv,w)\\
	k(v,w) -& (v,kw)\\
	(v_1,w)+(v_2,w) -& (v_1+v_2,w)\\
	(v,w_1)+(v,w_1) -& (v,w_1+w_2)\\
\end{aligned}
\end{equation*}
Replace uses of $\oplus$ with $\otimes$ and "direct sum" with "tensor product".

\subsection{Important Lemmas}
There's two useful lemmas I want to get out there but their proof is bloated.
Either I define two new functions that I won't use again or juggle a lot of constants.
Since I don't like either I'm just going to let them have a messy subsection to themselves.
\\

Let $S$ be the subset of forms from the linear case above, $f$ be linear, and $s \in \spn(S)$.
We can write $s$ as:
\[s =\sum_nK_n(k_n(v_n,w_n)-(k_nv,k_nw))+\sum_nK_n'((v_n',w_n')+(v_n'',w_n'')-(v_n'+v_n'',w_n'+w_n''))\]
If we consider these values in the following expression:
\begin{equation*}
\begin{aligned}
	&\sum_nK_n(k_nf(v_n,w_n)-f(k_nv,k_nw))+\sum_nK_n'(f(v_n',w_n')+f(v_n'',w_n'')-f(v_n'+v_n'',w_n'+w_n'')) \\
	=&\sum_nK_n\cdot0+\sum_nK_n'\cdot0 \\
	=&0\\
\end{aligned}
\end{equation*}

We see that if take the values out of $s$ and plug them into a linear function then we get zero.
If the reader feels so inclined they can define a function from the formal sums to range of $f$ to formalize this argument but as stated I don't want to do that,
and instead will hope you simply get that it works here and with the bilinear case.

\subsection{Universal Properties}
$V\oplus W$ and $V\otimes W$ have important applications as bookkeeping for function arguments.
To see what I mean let $X$ be an unknown vector space and consider a linear function $f:V\times W \rightarrow X$:
\begin{equation*}
\begin{aligned}
	kf(v,w) =& f(kv,kw)\\
	f(v_1,w_1)+f(v_2,w_2) =& f(v_1+v_2,w_1+w_2)\\
\end{aligned}
\end{equation*}

We define a new function $\tilde{f}:V\oplus W \rightarrow X$ by:
\[\tilde{f}\left(\sum_nk_n(v_n\oplus w_n)\right) = \sum_nk_nf(v_n,w_n)\]

By inspection we can see that $\tilde{f}$ is defined for all members of $V\oplus W$ and that those values are completely defined by $f$\footnote{All members of $V\oplus W$ can be written as $\sum_nk_nf(v_n\oplus w_n)$ and the LHS is defined solely in terms of $f$.},
so to prove well-definedness of {\em the} function we only need to show that all ways to write the same direct sum give the same value, i.e. that things like the following hold:
\[f((v_1\oplus w_1)+(v_2\oplus w_2)) = f((v_1+v_2)\oplus (w_1+w_2))\]
\\

By definition of the quotient space direct sums are equivalent class meaning they are equal iff their formal sums differ by an element in $\spn(S)$:
\begin{equation*}
\begin{aligned}
	&\sum_nk_n(v_n\oplus w_n) = \sum_nk_n'(v_n'\oplus w_n')\\
	\Rightarrow&\sum_nk_n(v_n,w_n) - \sum_nk_n'(v_n',w_n') \in \spn(S)\\
\end{aligned}
\end{equation*}
From the important lemmas section, 
if we input the values of these formal sums into $f$ it evaluates to $0$.
\[\sum_nk_nf(v_n,w_n) - \sum_nk_n'f(v_n',w_n') = 0 \]

Hence:
\begin{equation*}
\begin{aligned}
	\tilde{f}\left(\sum_nk_n(v_n\oplus w_n)\right) =& \sum_nk_nf(v_n,w_n)\\
	=& \sum_nk_n'f(v_n',w_n')\\
	=&\tilde{f}\left(\sum_nk_n'(v_n'\oplus w_n')\right) \\
\end{aligned}
\end{equation*}
Hence $\tilde{f}$ is well-defined and unique for any given $f$.
Similar arguments work for bilinear functions and the tensor product.

\subsection{Interpretation}
Consider archetypal linear and bilinear functions that occur when $V=W=F$:
\begin{equation*}
\begin{aligned}
	\text{Linear: }&f(v,w) = v+w\\
	\text{Bilinear: }&f(v,w) = vw\\
\end{aligned}
\end{equation*}
It's easy to verify these functions are linear and bilinear respectfully and that,
down to constants,
are the only such functions.
Hence we expect expect the direct some resemble the addition of vector from two different spaces and for the tensor product to resemble their multiplication.
