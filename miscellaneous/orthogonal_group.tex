% Copyright 2023 Kieran W Harvie. All rights reserved.

\section{Orthogonal Group}
The orthogonal group $O(n)$ has two equivalent definitions:
\begin{quote}
	The group of endomorphisms of the vector space $\mathbb{R}^n$, under compositions, such that the endomorphisms:

	\quad $\circ$\,Preserve the Euclidean Norm.
	
	{\centering Or}

	\quad$\circ$\,Takes orthonormal bases to orthonormal bases.
\end{quote}
To prove equivalence we will identify endomorphisms with matrices,
taking cation that for the second definition the matrix base $b_k$ will need to be orthonormal:

Preserving the Euclidean Norm:
\[\|Mv\| = \|v\|\]

Orthonormal bases to orthonormal bases:
\[MM^T=M^TM=I\]

\subsubsection{Preserving the Euclidean Norm $\Rightarrow$ Orthonormal bases to Orthonormal bases:}
The inner product can be written in terms of the Norm:
\[\langle u,v\rangle = \frac{\|u+v\|-\|v\|-\|u\|}{2\sqrt{\|u\|\|v\|}}\]
Hence:
\begin{equation*}
\begin{aligned}
\langle Mb_i,Mb_j\rangle =&\frac{\|Mb_i+Mb_j\|-\|Mb_j\|-\|Mb_i\|}{2\sqrt{\|Mb_i\|\|Mb_j\|}}\\
 =&\frac{\|b_i+b_j\|-\|b_j\|-\|b_i\|}{2\sqrt{\|b_i\|\|b_j\|}}\\
 =&\langle b_i,b_j\rangle\\
\end{aligned}
\end{equation*}
As required.

\subsubsection{Orthonormal bases to Orthonormal bases $\Rightarrow$ Preserving the Euclidean Norm:}
This follows directly from Parseval's identity:
\begin{equation*}
\begin{aligned}
\|Mv\| =& \left\|a_k\sum_kMb_k\right\| \\
=& \sum_k|a_k|^2\\
=& \|v\|\\
\end{aligned}
\end{equation*}

\subsection{Parameterization of $O(2)$}
The elements of $O(2)$ are pretty easy to parameterize: 

\subsubsection{Matrix Parametrization:}
We need to find a parametrization for the following elements:
\[
\begin{bmatrix}
	m_{1,1}&m_{1,2} \\
	m_{2,1}&m_{2,2} \\
\end{bmatrix}
\]
Start by multiplying the standard basis: 
\[
\begin{bmatrix}
	m_{1,1}&m_{1,2} \\
	m_{2,1}&m_{2,2} \\
\end{bmatrix}
\begin{bmatrix}
	1\\0\\
\end{bmatrix}
=
\begin{bmatrix}
	m_{1,1} \\ m_{2,1}\\
\end{bmatrix}
\quad
\begin{bmatrix}
	m_{1,1}&m_{1,2} \\
	m_{2,1}&m_{2,2} \\
\end{bmatrix}
\begin{bmatrix}
	0\\1\\
\end{bmatrix}
=
\begin{bmatrix}
	m_{1,2} \\ m_{2,2}\\
\end{bmatrix}
\]
Since orthonormal bases are sent to orthonormal bases the following set is an orthonormal basis:
\[
\left\{
\begin{bmatrix}
	m_{1,2} \\ m_{2,1}\\
\end{bmatrix}
,
\begin{bmatrix}
	m_{1,2} \\ m_{2,2}\\
\end{bmatrix}
\right\}
\]
Hence the norm of the first element is $1$, meaning:
\[m_{1,1}^2+m_{2,1}^2=1\]
Hence the first column is given by, for some $\theta$:
\[m_{1,1} = \cos(\theta)\,,\quad m_{2,1} = \sin(\theta)\]
\\
The second column needs to be perpendicular to the first.
The perpendicular space to $(\cos(\theta),\sin(\theta))$ has dimension $2-1 = 1$ and is given by:
\[
t
\begin{bmatrix}
	-\sin(\theta) \\ \cos(\theta)\\
\end{bmatrix}
\]
However the second column needs to be normalized:
\[
\left\|
t
\begin{bmatrix}
	-\sin(\theta) \\ \cos(\theta)\\
\end{bmatrix}
\right\|
= t^2
\]
Hence the second column is one of:
\[
\begin{bmatrix}
	\mp\sin(\theta) \\ \pm\cos(\theta)\\
\end{bmatrix}
\]

\subsubsection{Geometric Parametrization:}
Let $A$ and $B$ be the end points orthogonal unit vectors with shared origin $O$.
Let $A'$ and $B'$ be the transforms of $A$ and $B$ respectfully.
Given $A'$ where can the $B'$ be located?
\begin{center}
\begin{tikzpicture}[every node/.style={black}]
	\coordinate (O) at (0,0);
	\coordinate (A) at (3,0);
	\coordinate (B) at (0,3);
	\coordinate (A') at (-2.4,1.8);

	\begin{scope}[-stealth]
		\draw (O) node[below]{$O$}-- (A) node[right] {$A$};
		\draw (O) -- (B) node[above] {$B$};
		\draw (O) -- (A') node[left] {$A'$};
	\end{scope}

	\pic[draw] {right angle = A--O--B};
\end{tikzpicture}
\end{center}
From normality all of $A,B,A'$ and $B'$ are on the same circle centered at $O$.
From orthogonality $B'$ is located on the intersection of this circle with the perpendicular line to $A'O$ at $O$ (Here labeled $P$ and $Q$):
\begin{center}
\begin{tikzpicture}[every node/.style={black}]
	\coordinate (P) at (1.8,2.4);
	\coordinate (Q) at (-1.8,-2.4);

	\draw[dashed,gray] (O) circle (3);

	\begin{scope}[-stealth]
		\draw (O) node[below]{$O$}-- (A) node[right] {$A$};
		\draw (O) -- (A') node[left] {$A'$};
	\end{scope}

	\draw (3.6,4.8) -- (-3.6,-4.8);
	\filldraw (Q) node[left] {$Q$} circle (2pt);
	\filldraw (P) node[right] {$P$} circle (2pt);
	\pic[draw] {right angle = A'--O--P};
\end{tikzpicture}
\end{center}
These operations have easy interpretations as rotation and reflection.
This can easily be seen by angle counting and introducing an angular bisector of $\angle AOA'$:
\begin{center}
\begin{tikzpicture}[every node/.style={black}]
	\draw[dashed,gray] (O) circle (3);

	\begin{scope}[-stealth]
		\draw (O) node[below]{$O$}-- (A) node[right] {$A$};
		\draw (O) -- (A') node[left] {$A'$};
		\draw (O) -- (B) node[above] {$B$};
		\draw (O) -- (Q) node[left] {$B'$};
	\end{scope}

	\pic[draw=red, ->] {angle = A--O--A'};
	\pic[draw] {right angle = A--O--B};
	\pic[draw] {right angle = A'--O--Q};
\end{tikzpicture}
\begin{tikzpicture}[every node/.style={black}]
	\draw[dashed,gray] (O) circle (3);

	\begin{scope}[-stealth]
		\draw (O) node[below]{$O$}-- (A) node[right] {$A$};
		\draw (O) -- (A') node[left] {$A'$};
		\draw (O) -- (B) node[above] {$B$};
		\draw (O) -- (P) node[right] {$B'$};
	\end{scope}

	\draw[red] (-0.9,-3) -- (0.9,3);
\end{tikzpicture}
\end{center}

\subsection{Group Order:}
This result is the main reasion I was thinking about the Orthogonal group.
\\

Let:
\[
R_\theta=\begin{bmatrix}
	\cos(\theta) & -\sin(\theta) \\
	\sin(\theta) & \cos(\theta) \\
\end{bmatrix}
\, ,\quad
M_\theta=\begin{bmatrix}
	\cos(\theta) & \sin(\theta) \\
	\sin(\theta) & -\cos(\theta) \\
\end{bmatrix}
\]
Through angle sum formulae we can derive:
\[R_\theta R_\phi = R_{\theta+\phi}\,,\quad M_\theta M_\phi = R_{\theta -\phi}\]
As a corollary we have:
\[M_\theta^2 = I\]
And 
\[R_\theta^n = I \text{ if and only if } \frac{n\theta}{2\pi}\in\mathbb{Z}\]

Hence, despite $M_{0}$ and $M_{\sqrt{2}\pi}$ having finite order $2$ their product:
\[M_{\sqrt{2}\pi}M_0 = R_{\sqrt{2}\pi}\]
Doesn't have a finite order, since the angle is irrational.
\\

This result can also be proved geometrically by considering a vector collinear with the mirror and that two reflections is a rotation (through considering order).
