%\documentclass[12pt]{article}
%\usepackage{amsmath}
%\usepackage{amssymb}

\section{Wave Equation}
\subsection{Symmetry}
Given some function $u(\vec{x},t)$ on space and time we want to understand it's dynamics under the following assumptions:
\begin{enumerate}
	\item {\textbf{Reversible}}, the dynamics should be symmetric in respect to reversing time.
	\item {\textbf{Isotropic}}, the function should be symmetric in respect to all directions.
	\item {\textbf{Relative}}, the absolute value doesn't matter, only changes.
	\item {\textbf{Perturbation}}, the changes should be small.
\end{enumerate}

The wave equation is a natural conclusion from from these constraints:
\[\frac{\partial^2}{\partial t^2}u = c\sum_i \frac{\partial^2}{\partial x_i^2} u\]

Because the change change is small we start by considering a Taylor expansion and try to get the lowest order terms.
Because it's relative we ignore the $0^\text{th}$ order terms.
Because it's reversible we ignore the $1^\text{st}$ order terms
Giving:
\[\frac{\partial^2}{\partial t^2}u = \sum_{i,j} w_{i,j}\frac{\partial^2}{\partial x_i \partial x_j} u\]
Because it's isotropic all the $\frac{\partial^2}{\partial x_i^2}$ coefficient need to be equal.
And without loss of generality we can assume a basis where the cross terms vanish, giving:
\[\frac{\partial^2}{\partial t^2}u = \sum_i cu\frac{\partial^2}{\partial x_i^2} = c\sum_i \frac{\partial^2}{\partial x_i^2} u\quad \square\]

\subsection{Old Symmetry Attempt}
(Not sure what I was doing here, lol. Seems like I went on the wrong path by not including $t$)
Lets start with with a function of space $u(\vec{x})$.
Lets assume it is continuous such that:
\[u(\vec{x} +\vec{r}) = u(\vec{x}) + \sum_{i}r_i\frac{\partial}{\partial x_i}u(\vec{x})+\frac{1}{2}\sum_{i,j}r_ir_j\frac{\partial^2}{\partial x_i\partial x_j}u(\vec{x}) \]
Lets remove all terms that aren't reversible:
\[u(\vec{x} +\vec{r}) = u(\vec{x}) + \frac{1}{2}\sum_{i}r_i^2\frac{\partial^2}{\partial x_i^2}u(\vec{x}) \]
Lets impose localization:
\[c^2r_0^2 = \sum_{i>0}r_i^2\]
Not sure exactly how this part works but we get the sign we need one example is to only keep terms that have $c^2r_0^2 = r_i^2$

\subsection{1-D}
I need to remind myself how to solve the 1-D case as a stepping stone for something else.
Reminder that the 1-D form, with unitary speed, is:
\[ \frac{\partial^2}{\partial t^2} u = \frac{\partial^2}{\partial x^2}u\]
Observe that plane waves where angular frequency and wave number have the same magnitude solve this:
\[ \exp(ik(x\pm t))\]
This suggests the use of Fourier transform, where the $\pm$ is used to fit the solution to initial value and derivative:
\[ u = \frac{1}{\sqrt{2\pi}}\int_\mathbb{R}(a_+(k)e^{ikt}+a_-(k)-e^{-ikt})e^{ikx}\,dk\]
Assume $f(x) = u(x,0)$ and $h(x) = \left[\frac{\partial}{\partial t}u\right](x,0)$.
Equating Fourier transforms gives:
\begin{equation*}
\begin{aligned}
	\hat{f}(k) =& a_+(k)+a_-(k) \\
	\hat{h}(k) =& ik(a_+(k)-a_-(k)) \\
\end{aligned}
\end{equation*}
Hence:
\[ u = \frac{1}{\sqrt{2\pi}}\int_\mathbb{R}(\hat{f}(k)\cos(kt)+\hat{h}(k)k^{-1}\sin(kt))e^{ikx}\,dk\]
\\

This can be processed further through the convolutions theorem.
(Because of the table of transforms I have it will be easier $\sin$ is turned into normalized $\sinc$ first):
\[ u = \frac{1}{\sqrt{2\pi}}\int_\mathbb{R}\left(\hat{f}(k)\cos(kt)+\hat{h}(k)t\sinc\left(\frac{kt}{\pi}\right)\right)e^{ikx}\,dk\]

\begin{equation*}
\begin{aligned}
	u =& \left[f(\tau)\ast\frac{1}{2}(\delta(\tau-t)+\delta(\tau+t))\right](x) + \frac{|t|}{2}\left[h(\tau)\ast\rect\left(\frac{\tau}{2t}\right)\right](x)\\
	 =&  \frac{1}{2}(f(x-t)+f(x+t)) + \frac{|t|}{2}\left[h(\tau)\ast\rect\left(\frac{\tau}{2t}\right)\right](x)\\
\end{aligned}
\end{equation*}

Which is such a cool equation!
It shows the reversibility and isotropic nature of the solution but the evenness of $x$ and $t$.
It shows the limit on how fast information travels by the convolution with $\rect$.
