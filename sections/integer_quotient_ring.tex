% Copyright 2023 Kieran W Harvie. All rights reserved.

\section{Isomorphisms of the quotient rings of $\mathbb{Z}[X]$ }
While working elsewhere I thought that it's actually really easy to prove:
\[\gcd(a,d)=\gcd(b,d) \Rightarrow \exists z \in\mathbb{Z} \text{ such that } z\frac{a}{d}-\frac{b}{d} \in \mathbb{Z}\]
Using Bézout's identity.
\\

\subsection{The proof}
From Bézout's identity we know there exits $r,s$ in $\mathbb{Z}$ such that:
\[ra+sd = \gcd(a,d) = \gcd(b,d)\]
From the definition of $\gcd$ there exists $d'\in\mathbb{Z}$ such that:
\[b'\gcd(b,d) = b\]
Multiplying the first equation by $b'$ gives:
\[(rb')a+(sb')d = b'\gcd(b,d) = b\]
Hence:
\[(rb')\frac{a}{d}-b\frac{b}{d} = -(sb')\]
As required.
\\

Note that we can actually control the size, and hence sign, of $z$ by using the:
\[ra+sd = a(r+bk)+b(s-ak)\]
Trick, meaning we can force $z>0$ and hence $z\in\mathbb{N}$.

\subsection{The questions}
So this proof brings up some questions:
\begin{enumerate}
	\item Doesn't this prove that $\mathbb{Z}[X]/(dX-a) \cong \mathbb{Z}[X]/(dX-b)$?
	\item Couldn't this be generalized to Bézout Domains?
	\item What about the converse?
\end{enumerate}

\subsection{Question 1}
Let:
\[A = \mathbb{Z}[X]/(dX-a),\quad B =\mathbb{Z}[X]/(dX-b)\]
We can say $A$ and $B \subset \mathbb{Q}$ where we identity $X$ with $\frac{a}{d}$ and $\frac{b}{d}$ respectfully. 
But since we can find a linear relation between the two $X$s in terms, we need two relations but we can avoid division completely, of the sub-field $\mathbb{Z}$ I'd expect an isomorphism.
\\

Assume the $\gcd$ condition is meet and let $z_0$ and $z_1$ be the integers such that:
\[z_1\frac{a}{d}-\frac{b}{d} = -z_0\]
Equally written as:
\[d(z_1a+z_0)=b\]

\subsubsection{Attempt 1}
Define $\phi: B \rightarrow A$.
A ring homomorphism fixes $1$ and hence acts identically on the set generated by $1$,
that is $\mathbb{Z}$.
Since $a\in\mathbb{Z}$ hence:
\begin{equation*}
\begin{aligned}
	a =& \phi(a) \\
	=&\phi(dX) \\
	=&d\phi(X) \\
\end{aligned}
\end{equation*}
Suggesting a form for $\phi(X)$:
\[\phi(X) = z_1X+z_0\]
This is all the degrees of freedom we get since being a homomorphism requires:
\[\phi\left(\sum_np_nX^n\right) = \sum_n\phi(p_n)\phi(X)^n\]
Hence the suggested form is:
\[\phi\left(\sum_np_nX^n\right) = \sum_np_n(z_1X+z_0)^n\]

But dealing with showing homomorphism when dealing the canceling term isn't something I want to do.

\subsubsection{Attempt 2}
Define a function $\phi: \mathbb{Z}[X] \rightarrow B$ where:
\[\phi\left(\sum_np_nX^n\right) = \sum_np_n(z_1X+z_0)^n\mod (dX-b)\]
Then we can use the first isomorphism theorem of rings and only need to show that $\phi$ is surjective and has a kernel of  $\langle dX-a \rangle$.
