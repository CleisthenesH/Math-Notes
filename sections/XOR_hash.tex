% Copyright 2023 Kieran W Harvie. All rights reserved.

\section{XOR hash}
Today I was presented with the following problem:
\\

Given an array with all integers $[1,100]$ with a single integer removed.
Assuming memory is limited, how do you efficiently determine the missing integer?
\\

The solution is to XOR all the elements of the array together.
Then XOR this with the known value for ALL the numbers between $[1,100]$.
The result will be the missing number.
\\

This problem seemed like a good introduction to hashes, like Zobrist, that XOR a bunch of info together.
The benefit of this approach is that the components can be added/removed by a XOR:
\[h(\{a_0,a_1\}) = h(\{a_0\})\XOR h(\{a_1\})\]

\subsection{Determining the constant}
It turns out its easy to calculate the constant for the original question by hand.
First define $T$ as:
\[T(n) = n \XOR T(n-1),\quad T(1) = 1\]
Then:
\[T(n) = \begin{cases}n&n \mod 4 = 0 \\1& n\mod 4 = 1\\ n+1 &n\mod 4 = 2\\ 0& n\mod 4 = 3\end{cases}\]
This can easily be proved by induction and remembering that for even $n$ we have:
\[n\XOR 1 = n+1\]

Because of the modularity you would expect there to be an expression for $T$ involving powers of the fourth root of unity ($i$): 
\[T(n) = \frac{1}{2} +\frac{1}{2}(1+(-1)^n)+\frac{1}{4}((i-1)(-i)^n-(i+1)i^n)\]
(Done on scrap paper, not double checked, close enough to see the form).
\\

While the hybrid function is probably a more useful form the last one involves complex numbers in a way I didn't expect.

\subsection{Homomorphic Hashing}
Those with mathematic training will notice that the property that make this approach work:
\[h(\{a_0,a_1\}) = h(\{a_0\})\XOR h(\{a_1\})\]
Has the same form as a homomorphism\footnote{"same form" as a "homomorphism", math pun intended} and would investigate homomorphic hash functions. 

Well the most influential definition of such a function happened in 2004 Krohn, Freedman and Mazieres proposed a definition of homomorphic hash function as function $H: V\rightarrow G$ such that:
\begin{itemize}
\item $H$ is collision resistant. Meaning we are unlikely to find $x$ and $y$ such that $H(x) = H(y)$.
\item $H$ is a homomorphism. Meaning $H(x+y) = H(x)+H(y)$ for all $x$ and $y$.
\end{itemize}

The main problem is defining $V$ that is compatible with the regiments on $H$.
Since we want to use the set union operator then the natural choice is $V = 2^S$ for some set $S$.
But our hash only has the same form as a homomorphism\footnote{haha} when the input sets are mutually exclusive.
In general we have:
\[H(S_0 \cup S_1) = H(S_0)\XOR H(S_1) \XOR H(S_0 \cap S_1)\]

We can make a collision resistant hash though.
Make $H$ uniformly distributed on $\mathbb{F}_2^{\log_2|S|}$ for $S$.
Then by induction and $x \XOR y = 0$ iff $x=y$ you can show that it stays uniform for $2^S$.
Meaning the chance of a collision is $\frac{1}{|S|}$

(This is actually a bound for when $|S|$ is a power of $2$, but you can figure out the rest)
\\

Homomorphic functions are still useful in cryptography though.
