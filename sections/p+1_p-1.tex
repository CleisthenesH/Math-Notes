% Copyright 2023 Kieran W Harvie. All rights reserved.

\section{Divisibility of $(p+1)(p-1)$}
It is well known that if $p$ is prime than either $p$ is a divisor of $6$ or has the remainder $1$ or $5$ when divided by $6$.
This can be verified by noting that a remainder of $0$,$2$, or $4$ would make $p$ divisible by $2$ and likewise for $0$ and $3$ for $3$.
\\

A cool corollary I saw today was that $p$ is a divisor of $6$ or $24$ is a divisor of $(p+1)(p-1)$.
This is also verified by cases.
If the remainder is $1$ then $p-1$ is divisible by $6$ and $p+1$ by $4$.
And if the remainder is $5$ then $p-1$ is divisible by $4$ and $p+1$ by $6$.
\\

The converse isn't true.
Since $24\cdot 3 = 72$,
but $(7-1)(7+1) = 48 < 72$ and $(11-1)(11+1) = 120 > 72$.
