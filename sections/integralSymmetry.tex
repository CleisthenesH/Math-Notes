% Copyright 2023 Kieran W Harvie. All rights reserved.

\section{Integrals and symmetry}
Start with change of variables:
\[\int_{\phi(U)}f(\mathbf{v})\,d\mathbf{v} = \int_Uf(\phi(\mathbf{u}))|\det\phi'(\mathbf{u})|\,d\mathbf{u}\]
Assume the function has the following properties:
\begin{equation*}
\begin{aligned}
	\phi(U) &= U \\
	|\det\phi'(\mathbf{u})| &= c \\
\end{aligned}
\end{equation*}
Basically a linear automorphism, a symmetry, on $U$.

We have:
\[\int_{U}f(\mathbf{v})\,d\mathbf{v} = c\int_Uf(\phi(\mathbf{u}))\,d\mathbf{u}\]
Hence:
\[0=\int_{U}\big(f(\mathbf{u})-cf(\phi(\mathbf{u}))\big)\,d\mathbf{u}\]
Which is interesting when we:
\begin{equation*}
\begin{aligned}
f(\mathbf{u}) = \frac{1}{2}\big(f(\mathbf{u})+cf(\phi(\mathbf{u}))+\frac{1}{2}\big(f(\mathbf{u})-cf(\phi(\mathbf{u}))\big) \\
\int_Uf(\mathbf{u})\,d\mathbf{u} = \frac{1}{2}\int_U\big(f(\mathbf{u})+cf(\phi(\mathbf{u}))\,\mathbf{u}+\frac{1}{2}\int_U\big(f(\mathbf{u})-cf(\phi(\mathbf{u}))\big)\,\mathbf{u} \\
= \frac{1}{2}\int_U\big(f(\mathbf{u})+cf(\phi(\mathbf{u}))\,\mathbf{u}\\
\end{aligned}
\end{equation*}

Part of the integral vanishes.

For example take:
\[\phi(x) = -x \]
\[U = [-1,1]\]

We get the familiar:
\[\int_{-1}^{1}f(x)\,dx = \frac{1}{2}\int_{-1}^{1}\big(f(x)+f(-x)\big)\,dx\]

The utility can be seen by applying this relation to the basis of a class of function.
Let $V = \langle 1,x,x^2 \rangle$ parabolas.
Notice that $x$ base element vanishes, simplify evaluating integrals.

====================================================

Odiously the symmetries act like a group and with functions being a vector space we can use group representations.
I don't remember much about representations atm though.

=======================================================

Recall that for two dimensional change of variables:
\[ (x,y) = \phi(u,v) \]
We have:
\[|\det\phi'(\mathbf{v})| = \frac{\partial x}{\partial u}\frac{\partial y}{\partial v} - \frac{\partial x}{\partial v}\frac{\partial y}{\partial u} \]

Rotation for the triangle is:
\[(x,y) = \frac{1}{2}(-u-\sqrt{3}v,\sqrt{3}u-v)\]
\end{document}
