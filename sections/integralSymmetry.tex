% Copyright 2023 Kieran W Harvie. All rights reserved.

\section{Integrals and Symmetry}

\subsection{Domain Symmetry} 
Let $U$ be a subset of $\mathbb{R}^n$ and let $\phi: U \rightarrow U$ be a function such that:
\begin{equation*}
\begin{aligned}
	\phi(U) &= U \\
	|\det\phi'(\mathbf{u})| &= 1 \\
\end{aligned}
\end{equation*}
Basically $\phi$ is a linear permutation\footnote{
	Note that $\phi$ being a permutation requires that if the magnitude of the determinate is constant it must be unity, this can be seen by setting $f$ to a constant.
}
on $U$, this is a symmetry in the most direct sense.
We obtain the following:
\\
\begin{equation*}
\begin{aligned}
	\int_{U}f(\mathbf{v})\,d\mathbf{v} =& \int_{\phi(U)}f(\mathbf{v})\,d\mathbf{v} \\
	=& \int_Uf(\phi(\mathbf{u}))|\det\phi'(\mathbf{u})|\,d\mathbf{u} \\
	=& \int_Uf(\phi(\mathbf{u}))\,d\mathbf{u} \\
\end{aligned}
\end{equation*}
\\

In particular we get:
\[0=\int_{U}\big(f(\mathbf{u})-f(\phi(\mathbf{u}))\big)\,d\mathbf{u}\]

This integral is important since a function can be split into a vanishing and non-vanishing part:
\begin{equation*}
\begin{aligned}
	f(\mathbf{u}) =& \frac{1}{2}\big(f(\mathbf{u})+f(\phi(\mathbf{u}))+\frac{1}{2}\big(f(\mathbf{u})-f(\phi(\mathbf{u}))\big) \\
	\int_Uf(\mathbf{u})\,d\mathbf{u} =& \frac{1}{2}\int_U\big(f(\mathbf{u})+f(\phi(\mathbf{u}))\,d\mathbf{u}+\frac{1}{2}\int_U\big(f(\mathbf{u})-f(\phi(\mathbf{u}))\big)\,d\mathbf{u} \\
	=& \frac{1}{2}\int_U\big(f(\mathbf{u})+f(\phi(\mathbf{u}))\,d\mathbf{u}\\
\end{aligned}
\end{equation*}
\\

For example, consider the classic odd function on an integral centered at $0$.
\[\phi(x) = -x \]
\[U = [-1,1]\]

We get the familiar:
\[\int_{-1}^{1}f(x)\,dx = \frac{1}{2}\int_{-1}^{1}\big(f(x)+f(-x)\big)\,dx\]
\\

The utility of this relation can be seen by applying it to the basis of a class of function.
Let $V = \langle 1,x,x^2 \rangle$, this is a basis for all parabolas.
Notice that $x$ base element vanishes, simplify the evaluation of integrals.
\\

For a 2-D example recall that for two dimensional change of variables:
\[ (x,y) = \phi(u,v) \]
We have:
\[|\det\phi'(\mathbf{v})| = \frac{\partial x}{\partial u}\frac{\partial y}{\partial v} - \frac{\partial x}{\partial v}\frac{\partial y}{\partial u} \]

The rotation symmetry for a regular triangle is:
\[(x,y) = \frac{1}{2}(-u-\sqrt{3}v,\sqrt{3}u-v)\]
\\

Obviously the symmetries act like a group and with functions being a vector space we can use group representations.

\subsection{Function Symmetry}
Let $f$ and $\phi$ be functions such that:
\[f(t) = \phi'(t)f(\phi(t))\]
Then for arbitrary $x_0$ and $x_1$ we have:
\begin{equation*}
\begin{aligned}
	\int^{x_1}_{x_0}f(t)\,dt =& \int^{\phi(x_1)}_{\phi(x_0)}\phi'(t)f(\phi(t))\,dt \\ 
	=& \int^{\phi(x_1)}_{\phi(x_0)}f(t)\,dt \\ 
	=& \int^{\phi(x_1)}_{x_1}f(t)\,dt+\int_{\phi(x_0)}^{x_1}f(t)\,dt \\ 
	\int^{\phi(x_0)}_{x_0}f(t)\,dt =& \int^{\phi(x_1)}_{x_1}f(t)\,dt \\ 
\end{aligned}
\end{equation*}
Hence the integral value is independent of $x_n$, in particular if $\phi$ has a fixed point then the integral is zero.
