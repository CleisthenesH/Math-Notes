% Copyright 2023 Kieran W Harvie. All rights reserved.

\section{Convex}
TODO: Flesh out, copy edit, and maybe include pictures for the geometric intuition in the cord section.
\\

In the section of Jensen's Inequality I defined a convex function $\phi$ on a set $X$ as one such that for $a_n\in X$ with $w_0+w_1 =1$ and $w_n \geq 0$ then we have:
\[\phi(w_0a_0+w_1a_1) \leq w_0\phi(a_0)+w_1\phi(a_1)\]
I've thought of another proof of the previous section but iterates over functions instead of points.
And we can use some geometry intuition (cords).
\\

\subsection{Cords}
First let me define a new function called the cord function that takes on an interval $[a,b] \subseteq X$:
\[C_{[a,b]}(x) = \phi(a)+(x-a)\frac{\phi(b)-\phi(a)}{b-a}\]
The convex property can be changed to:
\[x\in [a,b] \Rightarrow \phi(x) \leq C_{[a,b]}(x)\]

{\bf Lemma:}, if $x_0 \leq x_1 \leq x_2$ then:
\[ x\in [x_0,x_1] \Rightarrow C_{[x_0,x_1]}(x) \leq C_{[x_0,x_2]}(x)\]
and:
\[ x\in [x_1,x_2] \Rightarrow C_{[x_1,x_2]}(x) \leq C_{[x_0,x_2]}(x)\]
{\bf Proof:} Use $C_{[a,b]}(a) = \phi(a)$ or $C_{[a,b]}(b) = \phi(b)$ and $x_1 \in [x_0,x_2]$ with the definition of convexity.
\\

{\bf Lemma:}, if $a_0 \leq a_1 \leq b_1 \leq b_0$ then:
\[x\in [a_1,b_1] \Rightarrow C_{[a_1,b_1]}(x) \leq C_{[a_0,b_0]}(x)\]
{\bf Proof:} use the previous lemma with the triples form the quad inequality, then chain together the newer inequalities.
\\

Let:
\[\phi_{[A,B]}(x) = \begin{cases} C_{[A,B]}(x) & x\in [A,B] \\ \phi(x) &\text{else} \end{cases}\]
$\phi_{[A,B]}(x)$ is convex.
Prove this using the previous lemmas with the three cases (points same side, both sides, one in one out).

\subsection{The Cool Proof}
We want to show that for all $\sum_{i=1}^nw_i'=1$ and $w_i' \geq 0$:
\[\phi\left(\sum_{i=1}^nw_i'a_i\right) \leq \sum_{i=1}^nw_i'\phi(a_i)\]

Assume it works for $n-1$ points.
Take two points $(w_0,a_0)$ and $(w_1,a_1)$ and make a new convex function using the method from the previous section with $[x_0,x_1]$, then remove the two points and replace with $(w_0+w_1, \frac{w_0a_1+w_1a_1}{w_1+w_0})$.
Now we have a $n-1$ points,
and if you expand the algebra you get the correct form,
hence by induction you prove the initial result.
