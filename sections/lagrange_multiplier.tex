% Copyright 2023 Kieran W Harvie. All rights reserved.

\section{Lagrange Multiplier}
Recall that local extrema $x$ of the function $f :\mathbb{R}^n \rightarrow \mathbb{R}$ subject to contrasts $g_i$ satisfy:
\[\nabla f(x) = \sum_i \lambda_i \nabla g_i(x)\]

The core observation is that if $\nabla f$ has a component outside the span of ${\nabla g_i}$ then you can move in that direction while keeping $g_i$'s constant, contradicting the point being an extrema.
\\

But the actual constants $\lambda_i$ have a useful interpretation as the rate the value of $f$ at the extrema changes as the constant $g_i$ changes.
To see this pick a particular $g_j$ and construct a $d$ such that:
\[d\cdot \nabla g_i = D\delta_{i,j}\]
You can achieve this by iteratively removing components in some matter like the following:
\[d_0 = \nabla g_0,\, d_{n+1} = d_n -d_n\cdot\nabla g_n\]

Now scale $d$ down such that functions around the extrema can be approximated through targets\footnote{Those so inclined are free to chase $\epsilon - \delta$'s}.
We have:
\begin{equation*}
	\begin{aligned}
		g_i(x+d) =& g_i(x)+d\cdot\nabla g_i(x)\\
		=& g_i(x) + D\delta_{i,j} \\
		f(x+d) =& f(x)+d\cdot\nabla f(x) \\
		=& f(x) + \sum_i \lambda_i d \cdot \nabla g_i(x) \\ 
		=& f(x) + D\lambda_j \\
		\nabla f(x+d) =& \nabla(f(x)+D\lambda) \\
		=& \nabla f(x) \\
		=& \sum_i \lambda_i \nabla g_i(x) \\
		=& \sum_i \lambda_i \nabla \big(g_i(x+d) - D\delta_{i,j}\big) \\
		=& \sum_i \lambda_i \nabla g_i(x+d)\\
	\end{aligned}
\end{equation*}
From the these equation we can see that $x+d$ satisfy the requirement to be an extrema.
We can also see that a change of $D$ in $g_i$ created a change of $\lambda_j D$ in the value at the extrema, hence giving a rate of change of $\lambda_j$.
\\

To-Do: Add and example of minimizing height when the two contrasts are parabolic and linear.
(You will need to use logs to get the change for the parabola to be a change in it's width and not height.)
