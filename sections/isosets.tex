% Copyright 2023 Kieran W Harvie. All rights reserved.

\section{Isosets}
Consider the function $f: \mathbb{R}^n \rightarrow \mathbb{R}$ let the isosets\footnote{Not the proper name, but I can't recall the proper name right now.} be sets $S$ in the domain of $f$ such that $f(S)$ is constant.
\\

Given some isoset $S$ and point $x \in \mathbb{R}^n$ of $f$ we want some kind of function that returns some type of measure $d \in \mathbb{R}$ of the distance of $x$ to $S$.
This function will be used in fragment shader rendering, which is why the mission statement is so vague.
We only need some general measure since it's better to efficiently get that measure and tweak coefficients then to get something perfectly accurate.
\\

A particular application is the $n=2$ case where we are looking to find the distance for the contour line.

\subsection{Naïve Solution}
The first idea is to compare the distance of $f(x)$ to $f(S)$ to $d$:
\[ d = |f(x)-f(S)|  \]
The problem with this solution is that the measure changes as a function of $|\nabla f(x)|$.
That is to say that the faster $f$ changes at $x$ the closer $x$ has to be to $S$ to get the same value of $d$.
\\

This method might work for some shader effects but not others.
For example it won't work to draw a constant width contour as the width would be inversely proportional to $|\nabla f(x) |$

\subsection{Better Solution}
If the underestimation is proportional to $|\nabla f(x)|$ the obvious solution is to divide by $|\nabla f(x)|$:
\[ d = \frac{|f(x)-f(S)|}{|\nabla f(x)|}\]
This is the solution currently used but deserves more analysis.
\\

For starters consider the case that $x$ is near the point $s \in S$.
Then $\nabla f(s)$ points away from $S$, that is that if a tangent to $S$ exists at $s$ then $\nabla f(s)$ is at orthogonal to the tangent, by definition of $S$ being a set such that $f$ is constant.
The combination of orthogonality and closeness lets us recover the original expression by use of the tangent surface:
\begin{equation*}
\begin{aligned}
	f(x) =& f(s)+(x-s)\cdot \nabla f(s) \\
	|f(x)-f(s)| =& |(x-s) \cdot \nabla f(s) |\\
	=& |x-s||\nabla f(s)| \\
	\frac{|f(x)-f(s)|}{|\nabla f(s) |} =& |x-s| \\
\end{aligned}
\end{equation*}
\\

Now consider the case where $x$ is not close to $S$ and the use case of drawing a constant width contour line.
Under what conditions do we avoid a false positive?
(That is the function thinks $x$ is closer than it is.)
Well we need some constraints on the rate at which $\nabla f(x)$ can grow.
To see this consider a point far away from $S$ but whose rate of change is very slow between most of $x$ and $s$, so that $|f(x) - f(s)|$ is small, but suddenly increases at $x$, such that $|\nabla f(x)|$ is large.
This causes their ratio to be small despite $|x-s|$ being large, false saying $x$ should be colored as part of the contour line.
If we reverse the set up, rate of change is large at first then slow, we will get a false negative.
(That the point is further than we think it is) 
\\

To see how a constraint would be useful, consider the following one: 
\[ |\nabla f(x+d)| \leq k|d||\nabla f(x)|\]
\begin{equation*}
\begin{aligned}
	|f(x)-f(s)| =& \left|\int_{0}^{1}\nabla f(x + (s-x)t) \cdot (s-x) \,dt\right| \\
	=& \left|(s-x)\cdot\int_{0}^{1}\nabla f(x + (s-x)t) \,dt\right| \\
	\leq& |(s-x)|\left|\int_{0}^{1}\nabla f(x + (s-x)t) \,dt\right| \\
	&\text{Cauchy-Schwartz} \\
	\leq& |(s-x)|\int_{0}^{1}\left|\nabla f(x + (s-x)t) \right|\,dt \\
	&\text{ML Bound} \\
	\leq& |(s-x)|\int_{0}^{1}k|(s-x)t||\nabla f(x)|\,dt \\
	\frac{|f(x)-f(s)|}{|\nabla f(x)|} \leq& \frac{k}{2}|s-x|^2 \\
\end{aligned}
\end{equation*}

This avoids a false negative as $x$ must be at least $ \sqrt{\frac{2d}{k}}$ away.

To-do: we need a inequities like $d \geq p(|x-s|)$ to get a bound on false positives.
\\
The condition:
\[ |\nabla f(x+d)| \leq k|d|+|\nabla f(x)|\]
Gives:
\[d \leq |x-s|\left(1+\frac{|x-s|}{|\nabla f(x)|}\right)\]
