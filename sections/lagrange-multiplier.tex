% Copyright 2023 Kieran W Harvie. All rights reserved.

\section{Lagrange Multiplier}
Recall that local extrema $x$ of the function $f :\mathbb{R}^n \rightarrow \mathbb{R}$ subject to contrasts $g_i$ satisfy:
\[\nabla f(x) = \sum_i \lambda_i \nabla g_i(x)\]

The core observation is that if $\nabla f$ has a component outside the span of ${\nabla g_i}$ then you can move in that direction while keeping $g_i$'s constant, contradicting the point being an extrema.
\\

But the actual constants $\lambda_i$ have a useful interpretation as the rate the extrema changes as the constant changes.
To see this pick a particular $g_j$ and construct a $d$ such that:
\[d\cdot \nabla g_i = D\delta_{i,j}\]
You can achieve this by induction with:
\[d_0 = \nabla g_0,\, d_{n+1} = d_n -d_n\cdot\nabla g_n\]

Now scale $d$ down such that functions around the extrema can be approximated through targets.
We have:
\[g_i(x+d) = g_i(x)+d\cdot\nabla g_i(x) = g_i(x) + D\delta_{i,j}\]
That all the constants are the same, except for $j$
\\

\begin{equation*}
	\begin{aligned}
		f(x+d) =& f(x)+d\cdot\nabla f(x) \\
		=& f(x) + \sum_i \lambda_i d \cdot g_i(x) \\ 
		=& f(x) + D\lambda_j
	\end{aligned}
\end{equation*}
Hence the rate of change of the constant to the extrema is:
\[(D\lambda_i)/D = \lambda_i\]
\\

The only loose end is to show $\nabla f(x+d)$ is in the span of ${\nabla g_i }$.
This follows from the previous equation.

