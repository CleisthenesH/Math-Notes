% Copyright 2023 Kieran W Harvie. All rights reserved.

\section{Generated Subfield of Finite Fields}
Let the finite field $F$ have a set of elements $S$.
The set $\langle S \rangle$ whose elements are the sums of the products of powers of $S$:
\[\langle S \rangle = \left\{\sum_n\prod_i s_i^{p_{i,n}}\,|\,s_i\in S \text{ and } p_{i,n}\in \mathbb{Z}\right\}\cup\{0\}\]
For example if $S = \{s_0,s_1,s_2\}$ all of the following are elements of $\langle S \rangle$:
\[s_0+s_1+s_2,\quad s_0s_1s_2s_3,\quad 1+s_0+s_1s_2^7+s_0s_1s_2^2\]
(I use the $\langle \cdot \rangle$ notation because this is like generation,
one of the most irregular and abused words in algebra).
\\

The set inherits it's operations from $F$ and is clearly closed under addition and multiplication.
It also contains $1$ and its repeated sum, labeled $n$, by using $p_{i,n} = 0$:
\[n = \overbrace{(s_0s_1s_2)^0+(s_0s_1s_2)^0+\dots s(s_0s_1s_2)^0}^{n \text{ times}}\]

Given $q\in \langle S \rangle$ because there are a finite number of options for $nq$ the pigeonhole principle means there will eventually be $n > m+1$ such that:
\begin{equation*}
\begin{aligned}
nq =& mq\\
(n-m)q =& 0\\
q+(n-m-1)q=& 0\\
\end{aligned}
\end{equation*}
$(n-m-1)q$ which is the additive inverse of $q$.
Because $n-m-1 > 0$ it is an element of $\langle S \rangle$ and since $q$ is also an element by closure so is the inverse.
\\

Likewise for the multiplicative inverse:
\begin{equation*}
\begin{aligned}
q^n=&q^m\\
q\cdot q^{n-m-1}=&1
\end{aligned}
\end{equation*}
Hence $\langle S \rangle$ is a, not necessarily proper, subfield of $F$.

\subsection{$S=\{1\}$}
An important example is when $S=\{1\}$.
Let $n$ be the "lowest"\footnote{Yes I know I haven't defined that properly, but just figure it out.} where the pigeonhole principle whole applies:
\begin{equation*}
\begin{aligned}
	m + (n-m) =& n+(m-m)\\
	=& n \\
	=& m \\
\end{aligned}
\end{equation*}

The minimality\footnote{See previous footnote.} of $n$ makes $n-m$ non-zero making $n$ zero.
Hence $S\cong\mathbb{F}_{n}$.
\\

(This is sloppy even for quick notes)

\subsection{Dreams of a more rigor}
Let $R$ be a ring and let $\phi$ be a function from $\mathbb{Z}$ to $R$ such that:
\[\phi(1)=1_R,\quad \phi(n\pm 1) = \phi(n)\pm \phi(1_R)= \phi(n)\pm 1_R\]
$\phi$ is well-defined as you can find the value for any $n\in\mathbb{Z}$ by induction.
And the function only has one value at each $n$ since you can't change the value by "doubling back" on the induction:
\[\phi((n\pm 1)\mp 1) = \phi(n\pm 1)\mp 1_R = \phi(n)\pm 1_R\mp 1_R = \phi(n)\]

Induction on $n$ shows:
\begin{equation*}
\begin{aligned}
	\phi(n)+\phi(m) =& \phi(n\pm 1)\mp 1 +\phi(m) \\
	=& \phi(n\pm 1)+\phi(m\mp 1) \\
	=& \phi(n\pm 1 + m\mp 1) \\
	=& \phi(n+m) \\
\end{aligned}
\end{equation*}

Which can be used to further prove:
\begin{equation*}
\begin{aligned}
	\phi(n)\phi(m) =&(\phi(n\pm 1)\mp 1)\phi(m) \\
	=&\phi(n\pm 1)\phi(m)\mp\phi(m) \\
	=&\phi((n\pm 1)m)\mp\phi(m) \\
	=&\phi((n\pm 1)m\mp m) \\
	=&\phi(nm)\\
\end{aligned}
\end{equation*}

These identities,
combined $\phi$ being defined with $\phi(1)=1_R$,
shows that $\phi$ is a homomorphism.
\\

Which, 
by the first isomorphisms theorem for rings,
shows that the $\img\phi$ is a subring of $R$ and is isomorphic to $\mathbb{Z}/\ker \phi$.
This formalizes the concept of the $n$th sum of $1_R$,
and lets us manipulate it better.
\\

\textbf{Corollary:} If $R$ is finite then the additive group of $R$ is also finite.
Meaning $1_R$ has an order,
which we label $N$.
In this case $\ker\phi = N\mathbb{Z}$ giving:
\[\img\phi = \frac{\mathbb{Z}}{N\mathbb{Z}}\]
It's really easy to use Bézout's identity on these objects to show which ones are fields.

\subsection{Wedderburn's Little Theorem}
This is similar to Wedderburn's little theorem, 
that all finite domains are fields.
Wedderburn generalizes the subject to domains instead of field and thus has to use $q-q^n = 0$ and some polynomial analysis (Headache).
But reduces the results,
it doesn't discuss closure.
\\

Cool result,
I suggest looking it up.
