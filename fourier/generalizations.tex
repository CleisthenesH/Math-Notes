% Copyright 2023 Kieran W Harvie. All rights reserved.
\chapter{Generalizations}

\section{Pontryagin Duality}
Let $G$ be a locally compact group and let $\hat{G}$ be the group of continuous group homomorphism from $G$ to the circle group $\Hom(G,T)$. Where $\hat{G}$ is endowed with the compact-open topology.

Then there is a canonical isomorphism between $G$ and $\hat{\hat{G}}$.

Note that locally compact groups come with an interesting measure called the "Haar measure" which interacts well with this transform.

\section{Discrete Fourier transform over a ring}
\label{sec:ring}
Observe that many of the properties of used in the definition of the Discrete Fourier require only that the underlying set be a ring with a special element $\alpha$ such that $\alpha^n = 1$ and $\sum_{k=0}^{n-1}\alpha^{kj} = n\delta_{j,n}$.
This lets us expand discussion of the Discrete Fourier transform to any ring with this element.
And also explains why some prefer the non-unitary definition as a square root is a significant jump from a ring.

\subsection{Definition}
Let $a_k$ be a n-tuple of ring elements, $\alpha$ be an element of that ring with the above properties. 
For convenience let $n$ have a multiplicative inverse, and define a new tuple as:
\[A_k = \sum_{j=0}^{n-1}a_j\alpha^{jk}\]

Then we have:
\begin{equation*}
\begin{aligned}
	\frac{1}{n}\sum_{i=0}^{n-1}A_i\alpha^{-ik} =& 
	\frac{1}{n}\sum_{i=0}^{n-1}\left(\sum_{j=0}^{n-1}a_j\alpha^{ji}\right)\alpha^{-ik} \\ 
	=&\frac{1}{n}\sum_{j=0}^{n-1}a_j\sum_{i=0}^{n-1}\alpha^{i(j-k)} \\ 
	=&\frac{1}{n}\sum_{j=0}^{n-1}a_jn\delta_{j,k} \\ 
	=&a_k \\ 
\end{aligned}
\end{equation*}

\subsection{Basic Properties}
Other then the obvious linearity there are two other easy and useful properties of the ring transform.
Uniqueness is a corollary of linearity:
\[A_n=B_n \Rightarrow a_n-b_n = \frac{1}{n}\sum_k\alpha^{-kn}(A_k-B_k) = 0\]

A version of Plancherel's Theorem is also easy:
\begin{equation*}
\begin{aligned}
	\sum_k a_kb_k =& \sum_k \left(\frac{1}{n}\sum_iA_i\alpha^{-ki}\right) \left(\frac{1}{n}\sum_jB_j\alpha^{-kj}\right) \\
	=&  \frac{1}{n^2}\sum_i\sum_jA_iB_j \sum_k\alpha^{-k(i+j)}  \\
	=&  \frac{1}{n^2}\sum_i\sum_jA_iB_j \sum_k\alpha^{k((n-i)-j)}  \\
	=&  \frac{1}{n}\sum_iA_iB_{n-i}  \\
\end{aligned}
\end{equation*}

I'm sure similar results for convolution also follow.

\subsection{Integral Domains}
It's worth noting that in integral domains the requirements for $\alpha$ are simplified as:
\[(\alpha^k-1)\sum_{i=0}^{n-1}\alpha^{ik} = \alpha^{nk}-1 = 0\]
Means it is sufficient for $\alpha$ to be a primitive nth root of unity.
It's further worth noting that all fields are integral domains.

