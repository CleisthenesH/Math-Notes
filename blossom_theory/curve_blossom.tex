% Copyright 2024 Kieran W Harvie. All rights reserved.

\section{Blossom of a Quadratic Bézier Curve}
Consider the function $f:\mathbb{R}\rightarrow\mathbb{R}^m$ used to define the  Quadratic Bézier curve with control points $\{P_0,P_1,P_2\}$:
\[f(t)=(1-t)^2P_0+2t(1-t)P_1+t^2P_2\]
It has the following bivariate blossom:
\[\mathcal{B}[f](t_1,t_2) = (1-t_1)(1-t_2)P_0+((1-t_1)t_2+(1-t_2)t_1)P_1+t_1t_2P_2\]
$\mathcal{B}[f]$ is the archetypal blossom,
Bézier Curves are the context blossoms where made to tackle,
and it having argument symmetry and diagonality can be verified by inspection.
And affinity follows by coefficient-wise substitution, for example:
\[\begin{aligned}
	w_0(1-s_0)+w_1(1-s_1)=& w_0+w_1-(w_0s_0+w_1s_1)\\
	=&1-(w_0s_0+w_1s_1)\\
\end{aligned}\]
Observe that we reclaim the original control points when $(t_1,t_2)\in\{0,1\}^2$:
\[
	f(0,0) = P_1\quad
	f(0,1) = P_2\quad
	f(1,1) = P_3
\]
Also observe that the sum of all the coefficients is $1$:
\[(1-s)(1-t)+((1-s)t+(1-t)s)+st = 1\]
This means the tuple can be interpreted as \hyperref[appx:bary]{barycentric coordinates} of the curve:
\[\big((1-s)(1-t),(1-s)t+(1-t)s,st\big)\]

