% Copyright 2024 Kieran W Harvie. All rights reserved.

\section{Notation Variants}
There's two other common notation for the control points of a Bézier Triangle the reader is likely to encounter.

\subsubsection{(Symmetrized) Tensor Product:}
\[(\alpha s + \beta r)^2 = \alpha^2+2\alpha\beta + \beta^2\]
Quadratic Bézier curve:
\[(\lambda_0\mathbf{e}_0+\lambda_1\mathbf{e}_1)^2 = \lambda_0^2\mathbf{e}_0^2+2\lambda_0\lambda_1\mathbf{e}_0\mathbf{e}_1+\lambda_1^2\mathbf{e}_2^2\]
Quadratic Bézier curve blossom:
\[(\lambda_{0,0}\mathbf{e}_0+\lambda_{0,1}\mathbf{e}_1)(\lambda_{1,0}\mathbf{e}_0+\lambda_{1,1}\mathbf{e}_1)\]

Works because of the universal property of the tensor product.
This is probalby the best but like most math it requires a larger upfront cost.
It simplifies and unifies curves and triangles of different orders.

\subsubsection{Weighted}
I think ours links to theory (blossoms) better and this one reflects application better
$P_{i,j,k}$ since they add to $n$ we can see them as weighted 
Particularly when working with Point-Normal triangles.

\subsubsection{Example conversion:}
Still needs to identify corners.
Bloosom 

% I was originally going to include a comparison of the blossoming to the interpolation of a polynomial using the finite difference.
% 
% \section{Finite Difference}
% The finite difference operator $\Delta_h$ is defined as:
% \[\Delta_h[f](x) = f(x+h)-f(x)\]
% We similarly define 
% \begin{equation*}
% \begin{aligned}
% 	\Delta^n_h[f](x) =& \Delta_h[\Delta^{n-1}_h[f]](x)\\
% 	=& \Delta^{n-1}_h[f](x+h)- \Delta^{n-1}_h[f](x)\\
% \end{aligned}
% \end{equation*}
% Let $p_n(x) = x^n$ then:
% \begin{equation*}
% \begin{aligned}
% 	\Delta[p_n](x) =& (x+h)^n-x^n\\
% 	=&\sum_{k=0}^{n-1}\binom{n}{k}h^{n-k}x^k\\
% \end{aligned}
% \end{equation*}
