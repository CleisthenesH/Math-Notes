% Copyright 2023 Kieran W Harvie. All rights reserved.

\section{Description}

\subsubsection{Key Generation:}
\begin{enumerate}
\item Generate two large distinct prime numbers $p$ and $q$.
\item Calculate $\lambda(pq) = \lcm(p-1,q-1)$.
\item Calculate $e$ such that $e$ and $\lambda(pq)$ are coprime and $2<e<\lambda(pq)$.
\item Calculate $d$ such that $de \mod \lambda(pq) = 1$.
\end{enumerate}
The public key is $(pq,e)$ and the private key is $d$,
$\lambda(pq)$ can be discarded.
\\

\subsubsection{Analysis:}
There's a multiple applications for the RSA cryptosystem so instead of jumping into any particular one we'll focus on the key feature,
that for an arbitrary $0\leq m < pq$ we have:
\[m^{ed}\mod pq = m\]

This is because by construction $ed-1$ is a multiple of $p-1$ meaning:
\[m^{ed}\mod p = m^{ed-1}m\mod p = m \mod p\]

\subsubsection{Remark:}
Observe that the above proof doesn't really use any of $\lambda(pq)$'s \hyperref[appx:torient]{properties} beyond it being divisible by $p-1$ and $q-1$.
When $m\in(\mathbb{Z}/pq\mathbb{Z})^\times$ we can present a much cooler proof.
In this case $ed-1$ is a multiple of $\lambda(pq)$, 
hence:
\[m^{ed-1} \mod pq = 1\]
From which the main result follows.
\\

The above proof is often presented as the sole proof of correctness for RSA,
this is incorrect and probably done because it is much cooler.

In practice it is true most of the time since $p$ and $q$ are large and:
\[\frac{|(\mathbb{Z}/pq\mathbb{Z})^\times|}{|\mathbb{Z}/pq\mathbb{Z}|} = \frac{\phi(pq)}{pq} = 1-\frac{1}{p}-\frac{1}{q} +\frac{1}{pq}\]

