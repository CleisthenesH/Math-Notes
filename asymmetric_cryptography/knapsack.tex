% Copyright 2023 Kieran W Harvie. All rights reserved.

The knapsack problem is an easy to grasp yet complex to solve optimization problem best summarised through the archetypal example:

\begin{quote}
Given a knapsack with with a limited size and a collection of items of fixed size and value determine which selections of items with a combined size less than or equal to the knapsack has the maximum value.
\end{quote}

The practical applications are obvious and the problem has been thoroughly studied because of it.
Yet the problem remains difficult to compute for arbitrary items,
which makes the knapsack problem a candidate for producing a trapdoor function.
\\

For a trapdoor function to work we need a version of the problem which is hard without private information but easy with private information.
The Subset Sum Problem (SSP) is a special case of the knapsack problem where the items values are constant and we must find a collection that {\bf equals} the knapsack size, not just be less than.
With arbitrary item sizes the SSP is known to be hard,
but when the item sizes are superincreasing the problem is trivial.
\\

A sequence $\{w_k\}$ is superincreasing if we have:
\[w_n > \sum_{k=1}^{n-1}w_k\]
In our terms it means there is no combination of items with size less than a given item that have a combined size greater or equal to the first item.
This makes solving the superincreasing SSP easy because if we always take the largest item that would fit in the remaining knapsack size, since the remaining smaller items \hyperref[appx:SSP]{can't make up the difference}.
\\

Hence the plan is to construct some private key which can convert a public SSP to a private superincreasing SSP.
