% Copyright 2023 Kieran W Harvie. All rights reserved.

\section{Algebra Revision}
\subsection{Lagrange's Theorem}
\subsection{Cyclic Group}
\subsection{Bézout's Identity}
I've send multiple time "Calculate $x$ such that $yx \mod z = 1$

\subsection{Chinese Remainder Theorem}
The Chinese Remainder Theorem is an interesting bring between regular and abstract algebra and I'll be using both forms in this document.

\subsubsection{Regular:}
Let $n_1$ and $n_2$ be coprime integers,
from Bézout's identity there exists $q_1$ and $q_2$ such that:
\[q_1n_1+q_2n_2 \mod n_1n_2 = 1\]
Now consider another integer $x$ such that:
\[x\mod n_1 = a_1\, \text{ and }\, x\mod n_2 = a_2\]
Then the only $0\leq x < n_1n_2$ is given by
\[x = a_1n_2q_2+a_2n_1q_1\mod n_1n_2\]
Proof follows by substitution.
\\

Consider the corollary when $a=a_1=a_2$ then we have:
\begin{equation*}
\begin{aligned}
x =& a_1n_2q_2+a_2n_1q_1\mod n_1n_2\\
=& a(n_2q_2+n_1q_1)\mod n_1n_2\\
=& a\\
\end{aligned}
\end{equation*}

\subsubsection{Abstract:}

We can use the prior construction to define a function 
\begin{equation*}
\begin{aligned}
f:\mathbb{Z}/n_1\mathbb{Z}\times\mathbb{Z}/n_2\mathbb{Z} &\rightarrow \mathbb{Z}/n_1n_2\mathbb{Z}\\
(a_1,a_2)&\mapsto a_1n_2q_2+a_2n_1q_1\\
\end{aligned}
\end{equation*}
It's left as an excise that $f$ is an isomorphism, meaning:
\[\mathbb{Z}/n_1\mathbb{Z}\times\mathbb{Z}/n_2\mathbb{Z} \cong \mathbb{Z}/n_1n_2\mathbb{Z}\]

You can use induction to generalize this to:
\[\mathbb{Z}/N\mathbb{Z} \cong \prod_k\mathbb{Z}/n_k\mathbb{Z}\]
Where $n_k$ are pairwise relatively prime positive integers and:
\[N =\prod_k n_k \]
Of particular importance is the case where $n_k = p_k^{m_k}$ for distinct primes $p_k$ and integers $m_k$.

\subsection{Multiplicative group of integers modulo $n$}
The "multiplicative group of integer module $n$" is is the multiplicative group of the ring $\mathbb{Z}/n\mathbb{Z}$ and is denoted $(\mathbb{Z}/n\mathbb{Z})^\times$.

\subsubsection{Structure:}
For general rings $R$ and $S$ we have:
\[(R\times S)^\times \cong R^\times \times S^\times\]
This follows from the, some what awkward looking, function:
\begin{equation*}
\begin{aligned}
	f:(R\times S)^\times&\rightarrow R^\times \times S^\times\\
	(r,s)&\mapsto (r,s)\\
\end{aligned}
\end{equation*}
Applying this to the Chinese remainder theorem gives:
\[(\mathbb{Z}/n\mathbb{Z})^\times \cong \prod_k (\mathbb{Z}/p_k^{m_k}\mathbb{Z})^\times\]
Where $n$ is given by the distinct primes $p_k$ as:
\[n =\prod_k p_k^{m_k} \]
When $p\neq 2$ we have: 
\[(\mathbb{Z}/p^{m}\mathbb{Z})^\times = C_{p^{m}-p^{m-1}}\]
And when $p=2$ we have:
\[(\mathbb{Z}/2^{m}\mathbb{Z})^\times =\begin{cases} 
	C_2\times C_{2^{m-2}} & m\geq 2\\
	C_1 & m = 1\\
\end{cases}\]


The general $m$ case is outside the scope of this document,
but can be found in most undergrad algebra textbooks.
The $m = 1$ case is pretty tractable:

\subsubsection{The $m = 1$ case:}

\subsubsection{Cyclic Corollary:}
\label{appx:cycle-multiplicative-group}
A corollary of the structure of $(\mathbb{Z}/n\mathbb{Z})^\times$ is that it cyclic if and only if:
\[n = 1,2,4,p^k,2p^k\]
For some odd prime $p$.
\\

This is because the product of cyclic groups is cyclic if and only if all but one are of size $1$.

\subsubsection{Example $(\mathbb{Z}/9\mathbb{Z})^\times$ table:}
Let $g\in(\mathbb{Z}/9\mathbb{Z})^\times$,
bellow is a table of important properties:

\begin{center}
\begin{tabular}{|c|}
	\hline
	$g$\\
	\hline
	1\\2\\4\\5\\7\\8\\
	\hline
\end{tabular}
\begin{tabular}{|c|c|c|c|c|c|}
	\hline
	$g^2$&$g^3$&$g^4$&$g^5$&$g^6$\\
	\hline
	1&1&1&1&1\\
	4&8&7&5&1\\
	7&1&4&7&1\\
	7&8&4&2&1\\
	4&1&7&4&1\\
	1&8&1&8&1\\
	\hline
\end{tabular}
\begin{tabular}{|c|}
	\hline
	$\ord(g)$\\
	\hline
	1\\6\\3\\6\\3\\2\\
	\hline
\end{tabular}
\begin{tabular}{|c|}
	\hline
	$g^{-1}$\\
	\hline
	1\\5\\7\\2\\4\\8\\
	\hline
\end{tabular}
\end{center}


\subsection{Totient Functions}
\label{appx:torient}
A totient of an integer $n$ is a number less than $n$ that is relatively prime to $n$.
Totient are important to cryptography because they are, by definition, the elements of $(\mathbb{Z}/n\mathbb{Z})^\times$.
\\

From this group we define two functions,
Euler's totient function:
\[\phi(n) = |(\mathbb{Z}/n\mathbb{Z})^\times|\]
and the Carmichael function\footnote{I have not idea why Euler gets to own his function but Carmichael does not.}:
\[\lambda(n) = \lcm\bigg\{\ord(k)\,|\, k\in(\mathbb{Z}/n\mathbb{Z})^\times\bigg\}\]

These functions are often studied in their own right

\subsubsection{Euler's Product Formula:}
\subsubsection{Carmichael's Theorem:}
\subsubsection{Carmichael's Theorem:}
\subsubsection{Remark:}
