% Copyright 2023 Kieran W Harvie. All rights reserved.

\section{Algebra Revision}
\subsection{Lagrange's Theorem}
\subsection{Cyclic Group}
\subsection{Bézout's Identity}
I've send multiple time "Calculate $x$ such that $yx \mod z = 1$

\subsection{Chinese Remainder Theorem}
The Chinese Remainder Theorem is an interesting bring between regular and abstract algebra and I'll be using both forms in this document.

\subsubsection{Regular:}
Let $n_1$ and $n_2$ be coprime integers,
from Bézout's identity there exists $q_1$ and $q_2$ such that:
\[q_1n_1+q_2n_2 \mod n_1n_2 = 1\]
Now consider another integer $x$ such that:
\[x\mod n_1 = a_1\, \text{ and }\, x\mod n_2 = a_2\]
Then the only $0\leq x < n_1n_2$ is given by
\[x = a_1n_2q_2+a_2n_1q_1\mod n_1n_2\]
Proof follows by substitution.
\\

Consider the corollary when $a=a_1=a_2$ then we have:
\begin{equation*}
\begin{aligned}
x =& a_1n_2q_2+a_2n_1q_1\mod n_1n_2\\
=& a(n_2q_2+n_1q_1)\mod n_1n_2\\
=& a\\
\end{aligned}
\end{equation*}

\subsubsection{Abstract:}

We can use the prior construction to define a function 
\begin{equation*}
\begin{aligned}
f:\mathbb{Z}/n_1\mathbb{Z}\times\mathbb{Z}/n_2\mathbb{Z} &\rightarrow \mathbb{Z}/n_1n_2\mathbb{Z}\\
(a_1,a_2)&\mapsto a_1n_2q_2+a_2n_1q_1\\
\end{aligned}
\end{equation*}
It's left as an excise that $f$ is an isomorphism, meaning:
\[\mathbb{Z}/n_1\mathbb{Z}\times\mathbb{Z}/n_2\mathbb{Z} \cong \mathbb{Z}/n_1n_2\mathbb{Z}\]

You can use induction to generalize this to:
\[\mathbb{Z}/N\mathbb{Z} \cong \prod_k\mathbb{Z}/n_k\mathbb{Z}\]
Where $n_k$ are pairwise relatively prime positive integers and:
\[N =\prod_k n_k \]
\subsection{Totient Functions}
\label{appx:torient}
A totient of an integer $n$ is a number less than $n$ that is relatively prime to $n$.
Totient are important to cryptography because they are, by definition, the elements the multiplicative group of the ring $\mathbb{Z}/n\mathbb{Z}$,
denoted $(\mathbb{Z}/n\mathbb{Z})^\times$.
\\

From this group we define two functions,
Euler's totient function:
\[\phi(n) = |(\mathbb{Z}/n\mathbb{Z})^\times|\]
and the Carmichael function\footnote{I have not idea why Euler gets to own his function but Carmichael does not.}:
\[\lambda(n) = \lcm\bigg\{\ord(k)\,|\, k\in(\mathbb{Z}/n\mathbb{Z})^\times\bigg\}\]

These functions are often studied in their own right

\subsubsection{Euler's Product Formula:}
\subsubsection{Carmichael's Theorem:}
\subsubsection{Cyclic Multiplicative group of integers modulo $n$:}
\label{appx:cycle-multiplicative-group}
\subsubsection{Carmichael's Theorem:}
\subsubsection{Remark:}
