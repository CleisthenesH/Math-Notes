% Copyright 2023-2024 Kieran W Harvie. All rights reserved.

\section{Algebra Revision}
\subsection{Extended Euclidean Algorithm}
I've send multiple time "Calculate $x$ such that $yx \mod z = 1$

\subsection{Structure of Integers Modulo $n$}
The Chinese Remainder Theorem is an interesting bring between regular and abstract algebra and I'll be using both forms in this document.

\subsubsection{Regular:}
Let $n_1$ and $n_2$ be coprime integers,
from Bézout's identity there exists $q_1$ and $q_2$ such that:
\[q_1n_1+q_2n_2 \equiv 1\mod n_1n_2\]
Now consider another integer $x$ such that:
\[x\equiv a_1 \mod n_1\, \text{ and }\, x\equiv a_2\mod n_2 \]
Then the only $0\leq x < n_1n_2$ is given by
\[x \equiv a_1n_2q_2+a_2n_1q_1\mod n_1n_2\]
Proof follows by substitution.

\subsubsection{Abstract:}
We can use the prior construction to define a function 
\[\begin{matrix}
\phi:\mathbb{Z}/n_1\mathbb{Z}\times\mathbb{Z}/n_2\mathbb{Z} \rightarrow \mathbb{Z}/n_1n_2\mathbb{Z}\\
(a_1,a_2)\mapsto a_1n_2q_2+a_2n_1q_1\\
\end{matrix}\]
Proof that $\phi$ is an isomorphism when $n_1$ and $n_2$ are coprime follows directly from the previous results, meaning:
\[\mathbb{Z}/n_1\mathbb{Z}\times\mathbb{Z}/n_2\mathbb{Z} \cong \mathbb{Z}/n_1n_2\mathbb{Z}\]
You can use induction to generalize this to:
\[\mathbb{Z}/n\mathbb{Z} \cong \prod_k\mathbb{Z}/n_k\mathbb{Z}\]
Where $n_k$ are pairwise relatively prime positive integers such that:
\[n =\prod_k n_k \]
Of particular importance is the case where $n_k = p_k^{m_k}$ for integers $m_k$ and distinct primes $p_k$ as this gives the structure of $\mathbb{Z}/N\mathbb{Z}$ in terms of of its prime factorization:
\[\mathbb{Z}/n\mathbb{Z} \cong \prod_k\mathbb{Z}/p_k^{m_k}\mathbb{Z}\]

\subsection{Structure of the Multiplicative Group of Integers Modulo $n$}
For general rings $R$ and $S$ we have:
\[(R\times S)^\times \cong R^\times \times S^\times\]
This follows from the following, some what awkward looking, function being an isomorphism:
\begin{equation*}
\begin{aligned}
	\phi:(R\times S)^\times&\rightarrow R^\times \times S^\times\\
	(r,s)&\mapsto (r,s)\\
\end{aligned}
\end{equation*}
Proof of which follows directly from the isomorphism theorem and is left as an exercise for the reader.
\\

Applying this to the structure of $\mathbb{Z}/n\mathbb{Z}$ previously shown gives:
\[(\mathbb{Z}/n\mathbb{Z})^\times \cong \prod_k (\mathbb{Z}/p_k^{m_k}\mathbb{Z})^\times\]
Where $n$ is given by the integers $m_k$ and distinct primes $p_k$ as:
\[n =\prod_k p_k^{m_k} \]
The factors can be further simplified too:
\[(\mathbb{Z}/p_k^{m_k}\mathbb{Z})^\times =\begin{cases} 
	C_{\phi\left(p_k^{m_k}\right)} & p_k \neq 2\\
	C_2\times C_{2^{m_k-2}} & m_k\geq 2\\
	C_1 & m_k = 1\\
\end{cases}\]
Proof of which are outside the scope of this document but can be found in most undergrad algebra textbooks.
\\

A corollary of the structure of $(\mathbb{Z}/n\mathbb{Z})^\times$ is that it cyclic if and only if:
\[n = 1,2,4,p^k,2p^k\]
For some odd prime $p$ as the product of cyclic groups is cyclic if and only if all but one are of size $1$.
\\

However,
proving just the subcase that $(\mathbb{Z}/p\mathbb{Z})^\times$ is that it cyclic when $p$ is prime is useful and tractable and is presented bellow. 

\subsection{The Multiplicative Group of Integers Modulo a Prime is Cyclic}
\label{appx:cycle-multiplicative-group}
The goal is to prove that:
\[(\mathbb{Z}/p\mathbb{Z})^\times\]
Is cyclic, the structure of this proof is the following:
\begin{enumerate}
	\item Prove there is some $d\in(\mathbb{Z}/p\mathbb{Z})^\times$ such that $g\in(\mathbb{Z}/p\mathbb{Z})^\times \Rightarrow g^{\ord(d)}=1$.
	\item Hence, the polynomial $f(x)=x^{\ord(d)}-1$ has $|(\mathbb{Z}/p\mathbb{Z})^\times|$ distinct roots.
	\item Hence, $\ord(d)=|(\mathbb{Z}/p\mathbb{Z})^\times|$.\footnote{This only works because $\mathbb{Z}/p\mathbb{Z}$ is a field.}
	\item Hence, $(\mathbb{Z}/p\mathbb{Z})^\times$ is cyclic with generator $d$.
\end{enumerate}
It takes a few steps to prove such a $d$ exists,
an important result to help us is that:
\[\ord(g^n) = \frac{\ord(g)}{\gcd(\ord(g),n)}\]
This result is pretty common in algebra textbooks, 
but a sketch of the proof is that by the minimality of $\ord(g)$ and $\ord(g^n)$ the relation:
\[(g^{n})^{\ord(g^n)} = g^{n\ord(g^n)}=1\] 
Means every factor of $\ord(g)$ not in $n$ has to come from $\ord(g^n)$.
\\
There is three corollaries from this result:

\subsubsection{Inverse Corollary:}
From the $n=-1$ case we have:
\[\ord(g^{-1})=\ord(g)\]

\subsubsection{Relatively Prime Corollary:}
Let $x,y\in(\mathbb{Z}/p\mathbb{Z})^\times$ with $\ord(x)$ and $\ord(y)$ being relatively prime,
then:
\[\ord(xy) = \ord(x)\ord(y)\]
Let $(xy)^n=1$, since $(\mathbb{Z}/p\mathbb{Z})^\times$ is Abelian we have:
\[x^ny^n = (xy)^n=1\]
From the previous corollary we have:
\[\ord(x^n) = \ord(y^n)\]
Giving:
\[\frac{\ord(x)}{\gcd(\ord(x),n)}=\frac{\ord(y)}{\gcd(\ord(y),n)}\]
The LHS is a factor of $\ord(x)$ and likewise for the RHS and $\ord(y)$.
Since they are relatively prime they can be equal only when they are both $1$,
meaning:
\[\ord(x) = \gcd(\ord(x),n),\quad\ord(y)=\gcd(\ord(y),n)\]
Hence $n$ is a multiple of both $\ord(x)$ and $\ord(y)$.
Since $\ord(x)$ and $\ord(y)$ are relatively prime the smallest such $n$ is:
\[n=\ord(x)\ord(y)\quad\square\]

\subsubsection{Lowest Common Denominator Corollary:}
Now let $x,y\in(\mathbb{Z}/p\mathbb{Z})^\times$ and let:
\[g=\gcd(\ord(x),\ord(y))\]
Then:
\[\ord(x^gy) = \lcm(\ord(x),\ord(y))\]
\\
This follows from $\ord(x^g)$ and $\ord(y)$ being relatively prime as:
\begin{equation*}
\begin{aligned}
	\ord(x^g)=&\frac{\ord(x)}{\gcd(\ord(x),g)}\\
	=&\frac{\ord(x)}{g}\\
\end{aligned}
\end{equation*}
Means $\ord(x^g)$ is $\ord(x)$ with all its common factors with $\ord(y)$ removed.
\\
\\
Hence we can use the previous corollary to get:
\begin{equation*}
\begin{aligned}
	\ord(x^gy)=&\ord(x^g)\ord(y)\\
	=&\frac{\ord(x)\ord(y)}{g}\\
	=&\lcm(\ord(x),\ord(y)) \quad \square\\
\end{aligned}
\end{equation*}

\subsubsection{Proving $d$ exists:}
From the preceding corollary for any $x,y\in(\mathbb{Z}/p\mathbb{Z})^\times$ there exits $z\in(\mathbb{Z}/p\mathbb{Z})^\times$ such that:
\[\ord(z) = \lcm(\ord(x),\ord(y))\]
Now $\lcm(\ord(x),\ord(y))$ is by definition a multiple of $\ord(x)$ and $\ord(y)$ hence:
\[z^{\ord(z)}=x^{\ord(z)}=y^{\ord(z)}=1\]
Hence a there exists $d\in(\mathbb{Z}/p\mathbb{Z})^\times$ such that:
\[g\in(\mathbb{Z}/p\mathbb{Z})^\times \Rightarrow g^{\ord(d)}=1\]
Since we can take two elements $x$ and $y$ and calculate $z=x^{\gcd(\ord(x),\ord(y))}y$,
then we take another element $w$ and calculate $z^{\gcd(\ord(z),\ord(x))}x^{\gcd(\ord(x),\ord(y))}y$.
We continue until all elements of $(\mathbb{Z}/p\mathbb{Z})^\times$ are included,
the end result of this process is $d$.
\subsubsection{Example $(\mathbb{Z}/9\mathbb{Z})^\times$ table:}
Let $g\in(\mathbb{Z}/9\mathbb{Z})^\times$,
bellow is a table of important properties that is useful as an example:

\begin{center}
\begin{tabular}{|c|}
	\hline
	$g$\\
	\hline
	1\\2\\4\\5\\7\\8\\
	\hline
\end{tabular}
\begin{tabular}{|c|c|c|c|c|c|}
	\hline
	$g^2$&$g^3$&$g^4$&$g^5$&$g^6$\\
	\hline
	1&1&1&1&1\\
	4&8&7&5&1\\
	7&1&4&7&1\\
	7&8&4&2&1\\
	4&1&7&4&1\\
	1&8&1&8&1\\
	\hline
\end{tabular}
\begin{tabular}{|c|}
	\hline
	$\ord(g)$\\
	\hline
	1\\6\\3\\6\\3\\2\\
	\hline
\end{tabular}
\begin{tabular}{|c|}
	\hline
	$g^{-1}$\\
	\hline
	1\\5\\7\\2\\4\\8\\
	\hline
\end{tabular}
\end{center}
