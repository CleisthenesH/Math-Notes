% Copyright 2023 Kieran W Harvie. All rights reserved.

\section{Algebra Revision}
\subsection{Lagrange's Theorem}
\subsection{Cyclic Group}
\subsection{Bézout's Identity}
\subsection{Totient Functions}
A totient of an integer $n$ is a number less than $n$ that is relatively prime to $n$.
Totient are important to cryptography because they are, by definition, the elements the multiplicative group of the ring $\mathbb{Z}/n\mathbb{Z}$,
denoted $(\mathbb{Z}/n\mathbb{Z})^\times$.
\\

From this group we define two functions,
Euler's totient function:
\[\phi(n) = |(\mathbb{Z}/n\mathbb{Z})^\times|\]
and the Carmichael function\footnote{I have not idea why Euler gets to own his function but Carmichael does not.}:
\[\lambda(n) = \lcm\bigg\{\ord(k)\,|\, k\in(\mathbb{Z}/n\mathbb{Z})^\times\bigg\}\]

These functions are often studied in their own right

\subsubsection{Chinese Remainder Theorem:}
Let $n_k$ be pairwise relatively prime positive integers and let:
\[N =\prod_k n_k \]
\[\mathbb{Z}/N\mathbb{Z} \cong \prod_k\mathbb{Z}/n_k\mathbb{Z}\]
\subsubsection{Euler's Product Formula:}
\subsubsection{Carmichael's Theorem:}
\subsubsection{Cyclic Multiplicative group of integers modulo $n$:}
\label{appx:cycle-multiplicative-group}
\subsubsection{Carmichael's Theorem:}
\subsubsection{Remark:}
