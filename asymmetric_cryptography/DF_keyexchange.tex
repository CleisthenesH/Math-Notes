% Copyright 2023 Kieran W Harvie. All rights reserved.

The Diffie-Hellman Key Exchange is a short but sweet demonstration of asymmetric cryptography.
The goal is for Alice and Bob to use a public channel to end with a shared private key.

\begin{enumerate}
\item Alice and Bob agree on an integer $n>1$ and a generating element $g$ of the cyclic group $C_n$ in private. 
\item Independently and in private,
Alice and Bob pick random integers $1<a<n$ and $1<b<n$ respectfully.
\item Alice and Bob let $g^a$ and $g^b$ be publicly known.
\item Alice and Bob calculate $(g^b)^a$ and $(g^a)^b$ from the public $g^b$ and $g^a$ respectfully.
\end{enumerate}

Since:
\[(g^b)^a = g^{ba} = g^{ab} = (g^a)^b\]
Alice and Bob agree on $g^{ab}$ while only making $n,g,g^a$ and $g^b$ public.
Under the discrete logarithm trapdoor assumption, 
a third party can not deduce $a$ or $b$ from this information meaning $g^{ab}$ is safe from attackers.
\\

Two important remarks.

Firstly,
if $g$ is publicly known then the exchange is susceptible to a man-in-the-middle attack.
Mallory is able to pretend to be Alice to Bob and send $g^m$ then Mallory and Bob will share the key $g^{mb}$ with Bob thinking the key was shared with Alice.

Secondly, 
The most direct implementation uses the $G=(\mathbb{Z}/p\mathbb{Z})^\times$ with prime $p$ and is called the Finite Field Diffie-Hellman.
This is because $\mathbb{Z}/p\mathbb{Z}$ is a finite field and $(\mathbb{Z}/p\mathbb{Z})^\times$ is always 
\hyperref[appx:cycle-multiplicative-group]{guaranteed to be a cyclic field}.
