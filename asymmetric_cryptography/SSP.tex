% Copyright 2023,2025 Kieran W Harvie. All rights reserved.

\section{Subset Sum Problem}
\label{appx:SSP}
The Subset Sum Problem (SSP) is the type of problem with a lot of defined and studied variants.
So it's worth formalizing our variant is:
\begin{quote}
	Given a nonincreasing positive sequence $(w_k)_{k=1}^n$ and an integer $c$ such that there exits a sequence $(m_k)_{k=1}^n$ with $m_k \in \{0,1\}$ and:
\[c = \sum_{k=1}^nm_kw_k\]
return such a sequence.
\end{quote}

We will formalize the greedy SSP algorithm as the recursive function:
\[gSSP(c,(w_k)_{k=1}^n) = \begin{cases}
	\emptyset & n = 0\\
	(1)^\frown gSSP(c-w_k,(w_k)_{k=1}^{n-1})& c \geq w_n \\
	(0)^\frown gSSP(c,(w_k)_{k=1}^{n-1})& c < w_n \\
\end{cases}\]

This algorithm is called greedy because at each call it reduces the first argument,
if it is able to.
{\bf This does not work in the general SSP case!}

For example, consider $gSSP(5,(4,3,2)) = (4)$,
the algorithm greedily takes $4$ but is then unable to make up the $5-4=1$ difference.
But if it has waited it would have found $2+3=5$.
\\

But I've claimed it does work then $(w_k)$ is superincreasing, 
this appendix is meant to back that up.

\subsection{Proof of Correctness}
Correctness equates to proving:
\[gSSP\left(\sum_{k=1}^nm_kw_k,(w_k)_{k=1}^n\right) =(m_k)_{k=1}^n\]
This will be done by induction, 
but first some useful lemmas.

\subsubsection{Superincreasing lemmas:}
If $\sum_{k=1}^nm_kw_k \geq w_n$ then $m_n = 1$.

Proof, assume $m_n = 0$ then:
\[\sum_{k=1}^nm_kw_k = \sum_{k=1}^{n-1}m_kw_k \leq \sum_{k=1}^{n-1}w_k < w_n\]

A contradiction.
\\
If $\sum_{k=1}^nm_kw_k < w_n$ then $m_n = 0$.

Proof, assume $m_n = 1$ then:
\[\sum_{k=1}^nm_kw_k \geq w_n\]

A contradiction.
\subsubsection{Base Case $n=1$:}

\begin{equation*}
\begin{aligned}
	gSSP\left(\sum_{k=1}^1m_kw_k,(w_k)_{k=1}^1\right) 
	=&gSSP\left(m_1w_1,(w_1)\right)\\
	=&\begin{cases} 
		(1)^\frown gSSP((m_1-1)w_1,\emptyset) & m_1w_1 \geq w_1\\
		(0)^\frown gSSP(m_1w_1,\emptyset) & m_1w_1 < 1_1\\
	\end{cases}\\
	=&\begin{cases} 
		(1)^\frown \emptyset & m_1w_1 \geq w_1\\
		(0)^\frown \emptyset & m_1w_1 < 1_1\\
	\end{cases}\\
	=&\begin{cases} 
	(1) & m_1w_1 \geq w_1\\
		(0) & m_1w_1 < 1_1\\
	\end{cases}\\
\end{aligned}
\end{equation*}
If $m_1=1$ we have:
\[m_1w_1\geq w_1 \Rightarrow gSSP(m_1w_1,(w_1)) = (1) = (m_1)\]
else if $m_1=0$ and we have:
\[m_1w_1 < w_1 \Rightarrow gSSP(m_1w_1,(w_1)) = (0) = (m_1)\]
As required 

\subsubsection{Induction:}
The $n$ case is given by the verbose:
\[gSSP\left(\sum_{k=1}^nm_kw_k,(w_k)_{k=1}^n\right) = 
	\begin{cases}
	(1)^\frown gSSP\left(\sum_{k=1}^nm_kw_k-w_k,(w_k)_{k=1}^{n-1}\right)& \sum_{k=1}^nm_kw_k \geq w_n \\
	(0)^\frown gSSP\left(\sum_{k=1}^nm_kw_k,(w_k)_{k=1}^{n-1}\right)& \sum_{k=1}^nm_kw_k < w_n \\
	\end{cases}
\]
It can be simplified using the superincreasing lemmas,
if $\sum_{k=1}^nm_kw_k \geq w_n $ we have $m_n = 1$ and:
\[\sum_{k=1}^nm_kw_k -w_k = \sum_{k=1}^{n-1}m_kw_k\]
and if $\sum_{k=1}^nm_kw_k < w_n $ we have $m_n = 0$ and:
\[\sum_{k=1}^nm_kw_k = \sum_{k=1}^{n-1}m_kw_k\]
Combining these results gives:
\[gSSP\left(\sum_{k=1}^nm_kw_k,(w_k)_{k=1}^n\right) = 
(m_n)^\frown gSSP\left(\sum_{k=1}^{n-1}m_kw_k,(w_k)_{k=1}^{n-1}\right)
\]
Which by the induction assumption:
\[gSSP\left(\sum_{k=1}^nm_kw_k,(w_k)_{k=1}^n\right) = (m_n)^\frown (m_k)_{k=1}^{n-1} = (m_k)_{k=1}^n\]
As required.

\subsection{Proof of Uniqueness}
For a given superincreasing sequence $(w_k)_{k=1}^n$ assume there are two distinct sequences $(n_k)_{k=1}^n$ and $(m_k)_{k=1}^n$ such that $m_k$ and $n_m$ are in $(0,1)$:
\[\sum_{k=1}^nm_kw_k = 	\sum_{k=1}^nn_kw_k\]
Without loss of generality assume that:
\[n_k \neq m_k\]
Let $K$ be the largest integer such that $m_k = 1$ or $n_k = 1$.
Without loss of generality assume that it is $m_K = 1$,
and have:
\[\sum_{k=1}^Km_kw_k = 	\sum_{k=1}^{K-1}n_kw_k\]
However:
\[\sum_{k=1}^Km_kw_k \geq w_K > \sum_{k=1}^{K-1}w_k \geq \sum_{k=1}^{K-1}n_kw_k\]
A contradiction,
hence $(n_k)_{k=1}^n$ and $(m_k)_{k=1}^n$ are not distinct.
\\

Uniqueness is an important property for our application since it ensures decryption always gives the original message.

\subsection{Digits}
Consider the formula for geometric progression:
\[\sum_{k=0}^nr^n = \frac{r^{n+1}-1}{r-1}\]
If $r>1$ this formula proves $w_k = r^{k-1}$ is a superincreasing function,
in fact we get the stronger:
\[r^{n+1} > (r-1)\sum_{k=0}^nr^n\]
With simple modifications the results for the superincreasing lemmas and uniqueness sections can be generalized to allow $m_k\in \{0,1,2,...,r-1\}$.
\\

It should have now hit you that under this generalized an expression $\sum_{k=1}^nm_kr^{k-1}$ is just the digit expansion.
And further that the $gSSP$ algorithm is just reading digits left-to-right.
So this whole section is proving something children implicitly understand once they start using digits,
that's math for ya.
