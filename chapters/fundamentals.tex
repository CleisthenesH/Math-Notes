% Copyright 2023 Kieran W Harvie. All rights reserved.
\chapter{Fundamentals}

Before moving towards applications of the Fourier transform it will be useful to collect some basic properties, theorems and symmetries.
As advertised in the introduction most of these will follow from the Fourier transform of the delta function.
\section{Basic Properties}
\subsection{Linearity}
It's almost so trivial as not need to be said, but the Fourier transform is linear,
Let:
\[h(x)=af(x)+bg(x)\]
Then:
\begin{equation*}
\begin{aligned}
\hat{h}(k) =& \int_{-\infty}^{\infty}h(x)\exp(-ikx)\,dx \\
=& \int_{-\infty}^{\infty}(af(x)+bg(x))\exp(-ikx)\,dx \\
=& a\int_{-\infty}^{\infty}f(x)\exp(-ikx)\,dx +b\int_{-\infty}^{\infty}g(x)\exp(-ikx)\,dx \\
=& a\hat{f}(k)+b\hat{g}(k) \\
\end{aligned}
\end{equation*}

\subsection{Duality}
Although other properties are more useful and obvious but learning about the duality first lets us double all coming properties.
Let:
\[g(x) = \hat{f}(x)\]
Then:
\begin{equation*}
\begin{aligned}
\hat{g}(k) =&\int_{-\infty}^{\infty}g(x)\exp(-ikx)\,dx \\
=&\int_{-\infty}^{\infty}\hat{f}(x)\exp(-ikx)\,dx \\
=& \int_{-\infty}^{\infty}\left(\int_{-\infty}^{\infty}f(t)\exp(-itx)\right)\,dt\exp(-ikx)\,dx \\
=& 2\pi\int_{-\infty}^{\infty} f(t) \left(\frac{1}{2\pi} \int_{-\infty}^{\infty} \exp(-ix(t+k))\,dx\right)\,dt \\
=& 2\pi\int_{-\infty}^{\infty} f(t) \delta(t+k)\,dt \\
=& 2\pi f(-k)
\end{aligned}
\end{equation*}
You can also show this by change of variable $x\mapsto -x$ and then apply the definition of inverse transform.
I don't like this because I fear getting the order of operations right in a partial/total derivative way.

\subsection{Translation}
A simple but useful property that we get immediately from definitions.
Let:
\[g(x) = f(x+x_0)\]
Then:
\begin{equation*}
\begin{aligned}
\hat{g}(k) =& \int_{-\infty}^{\infty}g(x)\exp(-ikx)\,dx\\
=& \int_{-\infty}^{\infty}f(x+x_0)\exp(-ikx)\,dx\\
=&\exp(ikx_0) \int_{-\infty}^{\infty}f(x+x_0)\exp(-ik(x+x_0))\,dx\\
=&\exp(ikx_0) \hat{f}(k)\\
\end{aligned}
\end{equation*}

\subsection{Scaling}
Let:
\[g(x) = f(ax) \]
Then:
\begin{equation*}
\begin{aligned}
	\hat{g}(k) =& \int_{-\infty}^{\infty}f(ax)\exp(-ixk)\,dx\\
	=& \int_{-\infty}^{\infty}f(x)\exp\left(-ix\left(\frac{k}{a}\right)\right)\,\frac{dx}{|a|}\\
	=& \frac{1}{|a|}\hat{f}\left(\frac{k}{a}\right) \\
\end{aligned}
\end{equation*}

Note how the special case of $a=-1$ means the transform commutative with changing the sign of the argument.

\subsection{Differentiation} 
Let:
\[g(x) = f'(x)\]
Then:
\begin{equation*}
\begin{aligned}
	\hat{g}(k) =& \int_{-\infty}^{\infty}f'(x)\exp(-ixk)\,dx\\
	=& \left[f(x)\exp(-ixk)\right]_{-\infty}^{\infty}-\int_{-\infty}^{\infty}f(x)(-ik)\exp(-ixk)\,dx\\
	=&ik\int_{-\infty}^{\infty}f(x)\exp(-ixk)\,dx\\
	=& ik\hat{f}(k) \\
\end{aligned}
\end{equation*}

The vanishing term on the second line will be explained in full by duality and the Riemann-Lebesgue lemma.

\subsection{Conjugation}
Let:
\[g(x) = \overline{f(x)}\]
Then:
\begin{equation*}
\begin{aligned}
	\hat{g}(k) =& \int_{-\infty}^{\infty}\overline{f(x)}\exp(-ikx)\,dx\\
	=& \overline{\int_{-\infty}^{\infty}f(x)\exp(-i(-k)x)\,dx}\\
	=& \overline{\hat{f}(-k)} \\
\end{aligned}
\end{equation*}

\section{Theorems}
\subsection{Convolution Theorem}
An immediate application of the Shifting Property is the Convolution Theorem.
Let:
\[h(x) = \int_{-\infty}^{\infty}f(t)g(x-t)\,dt\]
Then:
\begin{equation*}
\begin{aligned}
\hat{h}(k) =& \int_{-\infty}^{\infty}h(x)\exp(-ikx)\,dx\\
=& \int_{-\infty}^{\infty}\left(\int_{-\infty}^{\infty}f(t)g(x-t)\,dt\right)\exp(-ikx)\,dx\\
=& \int_{-\infty}^{\infty}f(t)\left(\int_{-\infty}^{\infty}g(x-t)\exp(-ikx)\,dx\right)\,dt\\
=& \int_{-\infty}^{\infty}f(t)\exp(-itk)\hat{g}(k)\,dt\\
=& \hat{g}(k)\int_{-\infty}^{\infty}f(t)\exp(-itk)\,dt\\
=& \hat{g}(k)\hat{f}(k)\\
\end{aligned}
\end{equation*}

\subsection{Poisson Summation Formula}
There is an interesting summation formula that follows from the \hyperref[appx:dirac-comb]{Dirac Comb}:
\begin{equation*}
\begin{aligned}
	\sum_{n=-\infty}^{\infty}\hat{f}(Dn) = & \int_{-\infty}^{\infty}\hat{f}(k)\Sha_D(k)\,dk \\
	= &\frac{1}{D} \int_{-\infty}^{\infty}\hat{f}(k)\left(\sum_{n=-\infty}^{\infty}\exp\left(ik \frac{2\pi n}{D}\right)\right)\,dk\\
	= &\frac{2\pi}{D} \sum_{n=-\infty}^{\infty}\frac{1}{2\pi}\int_{-\infty}^{\infty}\hat{f}(k)\exp\left(ik \frac{2\pi n}{D}\right)\,dk\\
	= &\frac{2\pi}{D} \sum_{n=-\infty}^{\infty}f\left(\frac{2\pi n}{D}\right)\\
\end{aligned}
\end{equation*}

Can apparently be proved through Pontryagin Duality, which sounds cool. 

\subsection{Real, Imaginary, Odd, and Even Components}
Consider a function $f$ that is both \hyperref[sec:conjugation-sym]{purely real} and \hyperref[sec:parity-sym]{odd}.
Then by combining results from the symmetry section $\hat{f}$ is odd and purely imaginary.
\\

Similar arguments for the other combinations gives the following table:

\begin{center}
\begin{tabular}{c|c}
	$f$ & $\hat{f}$ \\ \hline
	Real and Even & Real and Even \\
	Real and Odd & Imaginary and Odd \\
	Imaginary and Even & Imaginary and Even \\
	Imaginary and Odd & Real and Odd \\
\end{tabular}
\end{center}

Interestingly these four groups are one-to-one.
Combining this table with the unique \hyperref[appx:real-img-odd-even]{representation of $f$} we know that the real and odd components of $\hat{f}$ depends solely on the Imaginary and odd components of $f$.

Basically splitting the Fourier transforms from one complex function on $\mathbb{R}$ to four real functions on $\mathbb{R}_+$.

\section{Symmetries}
\subsection{Parity Symmetry}
\label{sec:parity-sym}
Let $f$ be an \hyperref[appx:real-img-odd-even]{even or odd} function then $\hat{f}$ is the same:
\[f(x)\pm f(-x) = 0 \Rightarrow \hat{f}(x)\pm\hat{f}(-x) = 0\]

\subsection{Conjugation Symmetry}
\label{sec:conjugation-sym}
Let $f$ be a \hyperref[appx:real-img-odd-even]{purely real} function then $\hat{f}$ is the Hermitian:
\[f(x) - \overline{f(x)} = 0 \Rightarrow \hat{f}(x)+\overline{\hat{f}(-x)} = 0\]

To my knowledge there is no name for the \hyperref[appx:real-img-odd-even]{purely imaginary} but a similar relation applies:
\[f(x) + \overline{f(x)} = 0 \Rightarrow \hat{f}(x)-\overline{\hat{f}(-x)} = 0\]

\subsection{Translation Symmetry}
Let $f(x)$ be periodic with period $P$.
Define $f_k = \int_{0}^{P}f(x)\exp(-ikx)\,dx$ and consider:
\begin{equation*}
\begin{aligned}
	\hat{f}(k) =& \int_{-\infty}^{\infty}f(x)\exp(-ixk)\,dx \\
	=& \sum_{n=-\infty}^{\infty}\int_{nP}^{(n+1)P}f(x)\exp(-ixk)\,dx \\
	=& \sum_{n=-\infty}^{\infty}\int_{0}^{P}f(x)\exp(-ik(x+nP))\,dx \\
	=& \left(\int_{0}^{P}f(x)\exp(-ikx)\,dx\right)\sum_{n=-\infty}^{\infty}\exp(-iknP) \\
	=& \frac{2\pi}{P}\left(\int_{0}^{P}f(x)\exp(-ikx)\,dx\right)\sum_{n=-\infty}^{\infty}\delta\left(k-n\frac{2\pi}{P}\right)\\
\end{aligned}
\end{equation*}

This is an interesting result by itself, that when $f$ is periodic $\hat{f}$ is mostly zero except for where $k$ is a multiple of $\frac{2\pi}{P}$.
But if we substitute this expressing into the inverse transform we get a long but interesting result:
\begin{equation*}
\begin{aligned}
	f(x) =& \frac{1}{2\pi}\int_{-\infty}^{\infty}\left(\frac{2\pi}{P}\left(\int_{0}^{P}f(t)\exp(-ikt)\,dt\right)\sum_{n=-\infty}^{\infty}\delta\left(k-n\frac{2\pi}{P}\right)\right)\exp(ixk)\,dk\\
	      =&\frac{1}{P}\sum_{n=-\infty}^{\infty}\int_{-\infty}^{\infty}\delta\left(k-n\frac{2\pi}{P}\right)\left(\int_{0}^{P}f(t)\exp(-ikt)\,dt\right)\exp(ixk)\,dk \\
	      =&\frac{1}{P}\sum_{n=-\infty}^{\infty}\left(\int_{0}^{P}f(t)\exp(-itn\frac{2\pi}{P})\,dt\right)\exp\left(ixn\frac{2\pi}{P}\right)\,dk
\end{aligned}
\end{equation*}

\subsection{Radial Symmetry}
Start with our standard 2-D Fourier integral:
\[F(k_x,k_y) = \int_{-\infty}^{\infty}\int_{-\infty}^{\infty}f(x,y)\exp(-i(k_x x + k_y y))\,dx\,dy\]
Assume we have radial symmetry such that:
\[f(r\cos(\theta),r\sin(\theta)) = g(r)\]
Make the following substitutions:
\begin{equation*}
\begin{aligned}
	x = r\cos(\theta),\quad k_x = k_r\cos(\phi) \\
	y = r\sin(\theta),\quad k_y = k_r\sin(\phi)\\
\end{aligned}
\end{equation*}
We get:
\begin{equation*}
\begin{aligned}
	&F(k_r\cos(\phi),k_r\sin(\phi)  \\
	=& \int_0^\infty\int_0^{2\pi}f(r\cos(\theta),r\sin(\theta))\exp(-ik_r r(\cos(\theta)\cos(\phi)+\sin(\theta)\sin(\phi)))r\,d\theta\,dr\\
	=& \int_0^\infty g(r)r\int_0^{2\pi}\exp(-ik_r r\cos(\theta-\phi))\,d\theta\,dr \\
\end{aligned}
\end{equation*}
Note that the inner integrand is a function of $\theta$ with period $2\pi$ which is the same as the integral bounds hence we can pick any $\phi$ in particular:
\[-\cos(\theta +\pi/2) = \sin(\theta)\]
Hence the total function has no dependence on $\phi$ giving:
\[F(k_r) = \int_0^\infty g(r)rJ_0(k_r r)\,dr\]

