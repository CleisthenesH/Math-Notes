% Copyright 2023 Kieran W Harvie. All rights reserved.

\chapter{Derivatives}
General
\section{A Tale of Two Derivative}
Partial and total.
Chain rule.
$\nabla$ operator maybe an appendix for vectors?
\[\frac{d\,f}{d\,t} = \sum_i \frac{\partial f}{\partial x_i}\frac{d\,x_i}{d\,t} = \nabla f \cdot \left(\frac{d\,x_0}{d\,t},\frac{d\,x_1}{d\,t},\frac{d\,x_2}{d\,t},\dots\right)\]
\section{Dual Numbers}
Like how we get Complex numbers from Real numbers by adjoining a new number $i$ such that $i^2=-1$ we get the Dual numbers by adjoining a new number $\epsilon$, 
called the infinitesimal, 
such that $\epsilon^2 = 0$.

Some properties aren't that special:
\[(x+x'\epsilon) + (y+y'\epsilon) = (x+y)+(x'+y')\epsilon\]

But consider what happens for multiplication and division:
\begin{equation*}
\begin{aligned}
(x+x'\epsilon)(y+y'\epsilon) =& xy+(x'y+xy')\epsilon+x'y'\epsilon^2 \\
=& xy+(x'y+xy')\epsilon \\
\frac{x+x'\epsilon}{y+y'\epsilon} =& \frac{(x+x'\epsilon)(y-y'\epsilon)}{(y+y'\epsilon)(y-y'\epsilon)} \\
 =& \frac{xy+(x'y-xy')\epsilon-x'y'\epsilon^2}{y^2-y'^2\epsilon^2} \\
 =& \frac{x}{y}+\frac{x'y-xy'}{y^2}\epsilon \\
\end{aligned}
\end{equation*}

Define:
$f\left(\dual{u}{u'},\dual{v}{v'}\right) = \dual{f(u,v)}{\nabla f(u,v) \cdot (u',v')}$

Because:
\begin{equation*}
\begin{aligned}
	f\left(\dual{u}{\frac{du}{dt}},\dual{v}{\frac{dv}{dt}}\right) =& \dual{f(u,v)}{\nabla f(u,v) \cdot \left(\frac{du}{dt},\frac{dv}{dt}\right)} \\
	=& \dual{f(u,v)}{\frac{d}{d\,t}f(u,v)} \\
\end{aligned}
\end{equation*}
(Because of the similar form this argument also works with partials).
\subsection{Historical Note}
This is actually close to how historic calculus 
Transcendental law of homogeneity only keep the lowest term.
Adequality.

\section{Automatic Differentiation}
\subsection{Forward Accumulation}
\[f(\dual{x}{1},\dual{y}{0}) = \dual{f(x,y)}{\frac{\partial}{\partial x}f(x,y)}\]

\subsection{Reverse Accumulation}
\[f\left(\dual{x}{\frac{dx}{df}},\dual{y}{\frac{dy}{df}}\right) = \dual{f(x,y)}{\frac{df}{df}} = \dual{f(x,y)}{1}\]
Adjoint: 
\[\bar{x} = \frac{\partial f}{\partial x}\]
Diamond with a forward value calculation and a reverse adjoint accumulation.
